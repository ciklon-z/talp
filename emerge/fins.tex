%	 *** Licence ***
% Cette œuvre est diffusée sous les termes de la license Creative Commons
% «~CC BY-NC-SA 3.0~», ce qui signifie que :

% Vous êtes libres :

%   * de reproduire, distribuer et communiquer cette création au public ;
%   * de modifier cette création.

% Selon les conditions suivantes :
    	
%  * Paternité - vous devez citer le nom de l'auteur original de la manière indiquée par l'auteur de l'œuvre ou le
%	         titulaire des droits qui vous confère cette autorisation (mais pas d'une manière qui suggérerait
%	         qu'ils vous soutiennent ou approuvent votre utilisation de l'œuvre).
%  * Pas d’utilisation commerciale - Vous n'avez pas le droit d'utiliser cette œuvre à des fins commerciales. 
%  * Partage des conditions initiales à l'identique - si vous transformez ou modifiez cette œuvre pour en créer une nouvelle,
% 						      vous devez la distribuer selon les termes du même contrat ou avec une
%					              licence similaire ou compatible.

% Comprenant bien que :

%  * Renoncement - N'importe quelle condition ci-dessus peut être retirée si vous avez l'autorisation du détenteur des droits.
%  * Domaine public - Là où l'œuvre ou un quelconque de ses éléments est dans le domaine public selon le droit applicable, ce statut
%		      n'est en aucune façon affecté par le contrat.
%  * Autres droits - d'aucune façon ne sont affectés par le contrat les droits suivants :
%    - Vos droits de distribution honnête ou d’usage honnête ou autres exceptions et limitations au droit d’auteur applicables;
%    - Les droits moraux de l'auteur;
%    - Droits qu'autrui peut avoir soit sur l'œuvre elle-même soit sur la façon dont elle est utilisée, comme la publicité
%      ou les droits à la préservation de la vie privée.

% === Note ===

% Ceci est le résumé explicatif du Code Juridique; La version intégrale du contrat est consultable ici:
% <http://creativecommons.org/licenses/by-nc-sa/3.0/legalcode>.

\PoemTitle{Coincoin la prune (14 Septembre 2012)}
  \begin{verse}
    Je suis le pitre intégral\\
    ébréché de symbales\\
    rondes et dorées à l’envie.
  \end{verse}
  \begin{verse}
    Je tombe j’éclaire\\
    je meurs je désespère\\
    de rien de presque rien\\
    -- tout en somme
  \end{verse}
  \begin{verse}
    oh poésie mensongère\\
    -- assignée en justice, contrefaçon\\
    autrement tournée que ce monde\\
    (je veux dire, que déraison)
  \end{verse}
  \begin{verse}
    Saoulerie sensas’\\
    démiurge allitérative --\\
    vermine explicative\\
    hors d’ici, hors de là, las!
  \end{verse}
  \begin{verse}
    Les légistes passent\\
    la ligne reprend son honnêt’té de cadavre,\\
    le vers n’est plus senti mais grouillant\\
    -- comme la fin de ceci, pour le moment.
  \end{verse}

\PoemTitle{Céhaiphe (12 Septembre 2012)}
  \begin{verse}
    Ce lieu nous est plaisant \& nous attaque\\
    et si l’on se repose survient la matraque\\
    \texttt{kwa 2 neuf?}
  \end{verse}
  \begin{verse}
    Tout achevé, il faut nous pendre\\
    au plus vite, au gibet de l’agitation\\
    pour taire à l’avance l’insatisfaction.
  \end{verse}
  \begin{verse}
    \textsc{Nouvelles du Soir}\\
    Trigorine a pondu un œuf aux lilas\\
    névrosé vénéneux à Nina,\\
    \qpass 80 morts en licenciement provisoire.
  \end{verse}
  \begin{verse}
    Cycle détestable de l’écrit,\\
    on t’amènera en Hyperborée\\
    si loin du travail tu peux être abhorré --
  \end{verse}
  \begin{verse}
    \textsc{Nouvelles du Midi}\\
    Quelle chance, le navire a coulé!
  \end{verse}

\PoemTitle{Décadré (19 Septembre 2012)}
  \begin{verse}
    Je te parlerai de mon ennemi le langage\\
    si tu en viens à dire «~infini~»\\
    je te piquerais, te jeterais au large\\
    cabot stupide grenouillant l’infamie --
  \end{verse}
  \begin{verse}
    ta chimère volée des chapelles,\\
    ton amour du langage-missel\\
    ta haine de la voluptueuse nature\\
    -- une chiée que tout cela, le langage est ordure;
  \end{verse}
  \begin{verse}
    si contre lui on s’exerce à napalmiser
    \footnote{Le néologisme n’est pas de moi. Le 21 Octobre, j’ai trouvé le mot en tant que participe chez \noun{Barthes} (\emph{Mythologies})}\\
    (en vain car tout acte ainsi fait est inutile)\\
    c’est pour la vie qu’en fin de compte il faut miser\\
    -- tout autre désir est futile.
  \end{verse}

\PoemTitle{Nexus (7 Octobre 2012)}
  \begin{verse}
    J’aime tes lèvres sangsues immobiles\\
    quand elles riment aux mots qui m’obnubilent\\
    et qu’elles sifflent les syntagmes volubiles
  \end{verse}
  \begin{verse}
    Paroles de Lourdaud, guère subtiles --\\
    poèmes, vers-proses: \hulu{À l’échafaud!}
  \end{verse}
