%	 *** Licence ***
% Cette œuvre est diffusée sous les termes de la license Creative Commons
% «~CC BY-NC-SA 3.0~», ce qui signifie que :

% Vous êtes libres :

%   * de reproduire, distribuer et communiquer cette création au public ;
%   * de modifier cette création.

% Selon les conditions suivantes :
    	
%  * Paternité - vous devez citer le nom de l'auteur original de la manière indiquée par l'auteur de l'œuvre ou le
%	         titulaire des droits qui vous confère cette autorisation (mais pas d'une manière qui suggérerait
%	         qu'ils vous soutiennent ou approuvent votre utilisation de l'œuvre).
%  * Pas d’utilisation commerciale - Vous n'avez pas le droit d'utiliser cette œuvre à des fins commerciales. 
%  * Partage des conditions initiales à l'identique - si vous transformez ou modifiez cette œuvre pour en créer une nouvelle,
% 						      vous devez la distribuer selon les termes du même contrat ou avec une
%					              licence similaire ou compatible.

% Comprenant bien que :

%  * Renoncement - N'importe quelle condition ci-dessus peut être retirée si vous avez l'autorisation du détenteur des droits.
%  * Domaine public - Là où l'œuvre ou un quelconque de ses éléments est dans le domaine public selon le droit applicable, ce statut
%		      n'est en aucune façon affecté par le contrat.
%  * Autres droits - d'aucune façon ne sont affectés par le contrat les droits suivants :
%    - Vos droits de distribution honnête ou d’usage honnête ou autres exceptions et limitations au droit d’auteur applicables;
%    - Les droits moraux de l'auteur;
%    - Droits qu'autrui peut avoir soit sur l'œuvre elle-même soit sur la façon dont elle est utilisée, comme la publicité
%      ou les droits à la préservation de la vie privée.

% === Note ===

% Ceci est le résumé explicatif du Code Juridique; La version intégrale du contrat est consultable ici:
% <http://creativecommons.org/licenses/by-nc-sa/3.0/legalcode>.

\PoemTitle{\texttt{merge branch} (14 Septembre 2012)}
  \begin{verse}
    Pour consommer cet amour\\
    il faudrait t’entr’ouvrir la gorge\\
    simplement d’un fin couteau\\
    et en extirper \hulu{Sloly} le boyau;
  \end{verse}
  \begin{verse}
    \hulu{Anatomiquement impossible?}
  \end{verse}
  \begin{verse}
    misère de la médecine: qu’importe l’œsophage\\
    si l’on peut en joie vivre l’anthropophage\\
    en nous qui s’exclame\\
    qui racle les os, les désarticulant,\\
    qui tombe, déambulant péremptoire\\
    hors de l’état de nuire de la Loire?
  \end{verse}
  \begin{verse}
    Que dira-t-on à toi, le poseur d’«~entre-guillemets~»\\
    sinon les accroches obscènes des firmamenteux\\
    ou bien les réclames sèches aux médicamenteux?
  \end{verse}
  \begin{verse}
    Plus tard? Plus tard c’est encore ailleurs --\\
    trainer du soufre des oxydés\\
    une charge de poudre amorcée\\
    sans doute, que faire ailleurs
  \end{verse}
  \begin{verse}
    dis? Où aller sinon plus loin\\
    sinon plus vieux plus bête\\
    bête à crever --\\
    cafards de la terre-mère:\\
    où donc vous enterrer?
  \end{verse}

\PoemTitle{Île d’Automne (16 Septembre 2012)}
  \begin{verse}
    Le temps même en ferait son affaire\\
    je n’ouvrirais pas le livre de l’enfer\\
    de ma mémoire nécrosée\\
    ouverte en fanion rouge sur le sommet;
  \end{verse}
  \begin{verse}
    Un aigle écarlate passe et me déchire\\
    le thorax comme le refus du pardon\\
    et la vie qui m’étire un sourire\\
    soudain ordonne «~Allons!~».
  \end{verse}
  \begin{verse}
    Un grand livre s’est ouvert sur le pire\\
    un royaume heureux sans roi, sans moi\\
    des armes vivantes dont on n’a l’empire\\
    qui cheminent sentant l’égoût criant l’effroi.
  \end{verse}
  \begin{verse}
    Premières neiges dissoutes et compassées,\\
    spectres aux milles mains efflûtées\\
    et même toi, l’infect regret\\
    à la bouche écorchée: où êtes-vous passés?
  \end{verse}

\PoemTitle{L’or des fous (10 Octobre 2012)}
  \begin{verse}
    Le silence je le crains n’existe pas\\
    même au bord bleu de la mer\\
    même au bord bleu de la littérature
  \end{verse}
  \begin{verse}
    tout est vacarme et discrète ordure,\\
    ruisseau mourant de plaintes amères\\
    à la fin, même ma tranquillité n’est pas.
  \end{verse}
  \begin{verse}
    Trouver une pépite dans le courant\\
    des replis de la terre et de la femme froufroutant\\
    c’est chercher l’or des fous et se perdre de tout son long\\
    c’est chercher l’espoir et le gagner pour de bon.
  \end{verse}

\PoemTitle{Ligne A vers Jules Vernes (26 Octobre 2012)}
  \begin{verse}
    Les rails du tram sont bruns et durs\\
    ainsi que de longs cheveux de femmes\\
    incurvés comme la main\\
    qui donne, et que cela dure!
  \end{verse}
  \begin{verse}
    Les graffitis grognent et grimacent en confettis\\
    une joie cynique dans les parages\\
    au tram se traine, et la ligne directrice
  \end{verse}
  \begin{verse}
    -- se perd
    dans le vert artificiel\\
    labellisé Fond de Marécage\\
    sans nuance, de multiples femmes stériles\\
    jouent les polymères imbéciles.
  \end{verse}
  \begin{verse}
    80 jours en ballon en rail\\
    en trainée de kérosène que ça dure\\
    dans l’arène urbaine -- clac!\\
    Les portes se ferment
  \end{verse}
  \begin{verse}
    mais pas la phrase, sans aucun doute;\\
    que de champs automobiles\\
    grisés de matière déjà les semailles,\\
    l’ironie d’fin de mois\\
    la faim dans l’armoire;
  \end{verse}
  \begin{verse}
    Société aérienne\\
    on est perdu quand on a le fil\\
    relié à tout, touchant le néant\\
    si peur de tomber sur de nouveaux rails\\
    qu’on a des pavés anti-glisse\\
    et ça lisse tant qu’on ne s’entend.
  \end{verse}
