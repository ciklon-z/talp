	%*** Licence ***

%Cette œuvre est diffusée sous les termes de la license Creative Commons
%«~CC BY-NC-SA 3.0~», ce qui signifie que :

%Vous êtes libres :

  %* de reproduire, distribuer et communiquer cette création au public ;
  %* de modifier cette création.

%Selon les conditions suivantes :
    	
  %* Paternité - vous devez citer le nom de l'auteur original de la manière indiquée par l'auteur de l'œuvre ou le
		%titulaire des droits qui vous confère cette autorisation (mais pas d'une manière qui suggérerait
		%qu'ils vous soutiennent ou approuvent votre utilisation de l'œuvre).
  %* Pas d’utilisation commerciale - Vous n'avez pas le droit d'utiliser cette œuvre à des fins commerciales. 
  %* Partage des conditions initiales à l'identique - si vous transformez ou modifiez cette œuvre pour en créer une nouvelle,
  						     %vous devez la distribuer selon les termes du même contrat ou avec une
						     %licence similaire ou compatible.

%{Comprenant bien que :

  %* Renoncement} - N'importe quelle condition ci-dessus peut être retirée si vous avez l'autorisation du détenteur des droits.
  %* Domaine public - Là où l'œuvre ou un quelconque de ses éléments est dans le domaine public selon le droit applicable, ce statut
		     %n'est en aucune façon affecté par le contrat.
  %* Autres droits - d'aucune façon ne sont affectés par le contrat les droits suivants :
    %- Vos droits de distribution honnête ou d’usage honnête ou autres exceptions et limitations au droit d’auteur applicables;
      %Les droits moraux de l'auteur;
    %- Droits qu'autrui peut avoir soit sur l'œuvre elle-même soit sur la façon dont elle est utilisée, comme la publicité
      %ou les droits à la préservation de la vie privée.

%=== Note ===

%Ceci est le résumé explicatif du Code Juridique; La version intégrale du contrat est consultable ici:
%<http://creativecommons.org/licenses/by-nc-sa/3.0/legalcode>.

%-------------------------------------------------------------------
\PoemTitle{Aller-retour (7 Juillet 2011)}
  \begin{verse}
    Les champs ont perdu de leur fraicheur,\\
    brulés à l’acide;
  \end{verse}
  \begin{verse}
    Les vallons que tu arbores,\\
    Ta beauté qui s’endort,\\
    Toi et moi, de même en rouges écrevisses,\\
    Toi et moi finirons par l’acide.
  \end{verse}
  \begin{verse}
    C’est maintenant qu’il faut récolter!\\
    Fauches, fauches – ou tais-toi !\\
    D’épi en épi nous nous éffilochons\\
    et l’acide remplace notre passion :
  \end{verse}
  \begin{verse}
    Grand caillou celeste et désolé\\
    tu t’accomodes mal de Morbidité\footnote{Synonyme de «~Humanité~»};\\
    et comme eux – mais tu ne le peux -\\
    durant toute la journée tu voudrais hurler!
  \end{verse}
  \begin{verse}
    Mais de vodka en pesticides\\
    pleuvent les seides des muets\\
    et couteau sous gorge, comme nous neutralisée,\\
    toute chose périra dans l’acide.
  \end{verse}
\PoemTitle{Le pas-piqué (1er Août 2011)}
  \begin{verse}
    J’ai la tête pleine de souffre\\
    et du fiel comblant ma bouche,\\
    pour rester sur la touche,
  \end{verse}
  \begin{verse}
    pour de vaines escarmouches…
  \end{verse}
  \begin{verse}
    Bouche d’égout, ouvres toi rouillée,\\
    boche d’égo, sublime meurtrier,\\
    masques en tout lieu pour combler\\
    ma p’tite livide vasque --
  \end{verse}
  \begin{verse}
    pour de vaines escarmouches…
  \end{verse}
  \begin{verse}
    Je veux ma dose, officier!\\
    plein le cogne de ton éther sirupeux,\\
    de tes corps nègres et démembrés;\\
    l’araignée réclame son miel officieux
  \end{verse}
  \begin{verse}
    pour de vaines escar-mouches…
  \end{verse}
  \begin{verse}
    Trouves-toi une promise pour te violer le ciboulot,\\
    une âme démise en quête de cases,\\
    la terreur sous un joli manteau\\
    pour te faire exploser au gaz
  \end{verse}
  \begin{verse}
    pour de vaines escarmouches…
  \end{verse}
%-------------------------------------------------------------------
\PoemTitle{Cafeïne sous interrogatoire (12 Septembre 2011)}
  \begin{verse}
    Le silence sage, la faim et le déseuvrement me donnent le haut-le-cœur\\
    et les livres s’ennuient d’absentes échardes;
  \end{verse}
  \begin{verse}
    Les conciliabules passent\\
    mais encore trop de monde au portillon\\
    -- ces sacs de viande remuent en spasmes\\
    -- encore là! -- Dégobillons.
  \end{verse}
  \begin{verse}
    Même ces arbres ont un aspect  hideux; à travers la vitre, les bois-singes
    singent des primates grossiers grotesque de marbre enracinés…\\
    même le ciel de grisaille se trouble de ces reproches glauques tournés à sa face.
  \end{verse}
%-------------------------------------------------------------------
\PoemTitle{Se dévissent les crânes (16 Septembre 2011)}
  \begin{verse}
    Madame a coincé sa jupe dans la porte du tram --\\
    quelle froussse et gémissant émoi!\\
    Cette ville mord sans entrain\\
    et ainsi plait aux épargnés dont moi.
  \end{verse}
  \begin{verse}
    Et toutes ces rues où courent sans entraves\\
    les vents de l’air et des femmes\\
    cultivent l’humour offrant\\
    et les sueurs de Madame.
  \end{verse}
  \begin{verse}
    Cohue-bohu dans le tram --\\
    controleur nazi en vue,\\
    un jeune a collé sur la tête \textsc{La Bévue} \\
    -- que cette ligne nous remue\\
    et emporte les rires de Madame.
  \end{verse}
  \begin{verse}
    On vous sourie sans s’en apercevoir\\
    -- les regards croisés en chemin se laissent choir,\\
    ce soir ne me donne que l’obséssion de boire;\\
    la masse dont Madame chemine vers l’espoir.
  \end{verse}
%-------------------------------------------------------------------
\PoemTitle{N’êst-ceu pas? (13 Septembre 2011)}
  \begin{verse}
    «~Salutations à vous mes vers solitaires\\
    si petits par habitude,\\
    écoutez moi,m’voyez, fraiche promotion\\
    faire celle de vos chaires…~»
  \end{verse}
  \begin{verse}
    (Les tables charrient de l’optique --\\
    divergente)
  \end{verse}
  \begin{verse}
    Frais comme des gardons, verts de shoa\\
    et aussi roses que le programme\\
    et aussi peu que le par-devant.
  \end{verse}
  \begin{verse}
    (Pas de pardessus en Septembre,\\
    les têtes restent flasques)
  \end{verse}
  \begin{verse}
    Allons nous escrimer -- d’un sens\\
    qu’on le prenne ou de l’autre,\\
    strabiens corrompus\\
    -- l’élastique se tord.
  \end{verse}
  \begin{verse}
    (L’heure passe comme une cerise dans ta bouche)
  \end{verse}
%-------------------------------------------------------------------
\PoemTitle{Allons en Orléans…puisque les autres routes sont coupées. (13 Septembre 2011)}
  \begin{verse}
    J’ai fait ma tournée chez les Pue-La-Mort,\\
    on dit orléanais,\\
    ~~~~quelquefois.
  \end{verse}
  \begin{verse}
    Ville dynamique, non, comique, assurément!\\
    Elle comble une folle d’une fête et d’un compliment --\\
    ~~~~Vermine
  \end{verse}
  \begin{verse}
    On marche presque sur les globes\\
    des rassis yeux racistes\\
    \& les décrets de mairie ont l’accent apostolique,\\
    ~~~~puanteur des tyranniques.
  \end{verse}
%-------------------------------------------------------------------
\PoemTitle{Aliments urbains (12 Septembre 2011)}
  \begin{verse}
    Il bruinait des gouttes comme des hommes\\
    -- une belle rentrée en somme;
  \end{verse}
  \begin{verse}
    La monstrueuse Orléans remuait ses osselets\\
    qui se tordaient -- pleins à craquer\\
    de moelle humaine -- prête à macérer.
  \end{verse}
  \begin{verse}
    Ô ouragan de murmures,\\
    Charybde désolée.
  \end{verse}
%-------------------------------------------------------------------
\newpage
\PoemTitle{Urbains aliments (12 Septembre 2011)}
  \begin{verse}
    Les termites ont achevé le chantier\\
    -- le bois dévale le mont quotidien;\\
    et moiS et moiS,\\
    toi et moiS, quand finirons nous?\\
    Ô concours d’entrailles usées…
  \end{verse}
  \begin{verse}
    De ce côté-ci l’auto’ pétarade\\
    sans lueur aucune\\
    -- ce qu’il en faut pour y voir clair\\
    dans la pieuvre nourricière!
  \end{verse}
  \begin{verse}
    Orléans et Plume, mes deux maniaques préférées,\\
    professions de glouglous d’encre et de vidanges
    \footnote{
	    Cf fin de \textsc{Beckett},\emph{Malone meurt}},\\
    étaleuses de sueur et de pisse!
  \end{verse}
  \begin{verse}
    La journée a fait tourner mal\\
    ces innombrables gouttes d’hommes en devenir…\\
    ~~~~~~~~~~~~~~~~~~~~~~~~~~~~~~~~~~~~~~~~~~~~~~~~g~e~s\\
    ~~~~~~~~~~~~~~~~~~~~~~~~~~~~~~~~~~~~~~~~~~~~~~~~~~a\\
    ~~~~~~~~~~~~~~~~~~~~~~~~~~~~~~~~~~~~~~~~~~~~~~~~~~u\\
    le soleil plein de honte en appelle aux n et à la p\\
    ~~~~~~~~~~~~~~~~~~~~~~~~~~~~~~~~~~~~~~~~~~~~~~~~~~~~~~~~~~~~~~~~~l\\
    ~~~~~~~~~~~~~~~~~~~~~~~~~~~~~~~~~~~~~~~~~~~~~~~~~~~~~~~~~~~~~~~~~~u\\
    ~~~~~~~~~~~~~~~~~~~~~~~~~~~~~~~~~~~~~~~~~~~~~~~~~~~~~~~~~~~~~~~~~~~i\\
    ~~~~~~~~~~~~~~~~~~~~~~~~~~~~~~~~~~~~~~~~~~~~~~~~~~~~~~~~~~~~~~~~~~~~e\\
    pour dénigrer nos têtes à cuvettes hygiéniques
    \footnote{
	    Les effets typographiques n’étaient pas nécessaires ni présents sur le carnet
	    d’origine mais ont été rajoutés pour détendre un peu l’atmosphère lourde du poème.}.
  \end{verse}
  \begin{verse}
    «~Une éclaircie peut-être,\\
    mon brave?\\
    -- Voyons donc demain~»
  \end{verse}
