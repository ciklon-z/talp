	%*** Licence ***

%Cette œuvre est diffusée sous les termes de la license Creative Commons
%«~CC BY-NC-SA 3.0~», ce qui signifie que :

%Vous êtes libres :

  %* de reproduire, distribuer et communiquer cette création au public ;
  %* de modifier cette création.

%Selon les conditions suivantes :
    	
  %* Paternité - vous devez citer le nom de l'auteur original de la manière indiquée par l'auteur de l'œuvre ou le
		%titulaire des droits qui vous confère cette autorisation (mais pas d'une manière qui suggérerait
		%qu'ils vous soutiennent ou approuvent votre utilisation de l'œuvre).
  %* Pas d’utilisation commerciale - Vous n'avez pas le droit d'utiliser cette œuvre à des fins commerciales. 
  %* Partage des conditions initiales à l'identique - si vous transformez ou modifiez cette œuvre pour en créer une nouvelle,
  						     %vous devez la distribuer selon les termes du même contrat ou avec une
						     %licence similaire ou compatible.

%{Comprenant bien que :

  %* Renoncement} - N'importe quelle condition ci-dessus peut être retirée si vous avez l'autorisation du détenteur des droits.
  %* Domaine public - Là où l'œuvre ou un quelconque de ses éléments est dans le domaine public selon le droit applicable, ce statut
		     %n'est en aucune façon affecté par le contrat.
  %* Autres droits - d'aucune façon ne sont affectés par le contrat les droits suivants :
    %- Vos droits de distribution honnête ou d’usage honnête ou autres exceptions et limitations au droit d’auteur applicables;
      %Les droits moraux de l'auteur;
    %- Droits qu'autrui peut avoir soit sur l'œuvre elle-même soit sur la façon dont elle est utilisée, comme la publicité
      %ou les droits à la préservation de la vie privée.

%=== Note ===

%Ceci est le résumé explicatif du Code Juridique; La version intégrale du contrat est consultable ici:
%<http://creativecommons.org/licenses/by-nc-sa/3.0/legalcode>.

%-------------------------------------------------------------------
\PoemTitle{La rampante (6 Juin 2011)}
  \begin{verse}
    L’araignée noire de jais glisse sur le mur
    comme une fille sur du poli acier;\\
    et la monstruosité cache mal ses profondes idées aux amères saveurs.
  \end{verse}
  \begin{verse}
    La rampante. La rampante, elle tourne autour de toi, qui l’aimante --
    joyeuse des deux tableaux, elle frappe ou patiente pour te dévorer,
    assène la matraque ou agit en corbac’.
  \end{verse}
  \begin{verse}
    Inévitablement, d’un recoin à l’autre de la tête, elle tisse ses fils couleur de lait…\\
    «~Je t’ai aperçu, Rampante! Crois-tu donc que j’ignore \emph{qui} tu es?~»
  \end{verse}
  \begin{verse}
    Araignée, tu avances et tu pointes tes pattes?\\
    Mais veux-je seulement de toi? Allons, va-t’en!
  \end{verse}
%-------------------------------------------------------------------
\PoemTitle{La solidarité des vers (Juin 2011)}
  \begin{verse}
    Qu’y-a-t-il de plus beau que les cimetières la nuit?\\
    Je dormais, je rêvais -- là est mon cimetière à moi;\\
    Comment concevrais-je la mort sans ses infinies variations?
  \end{verse}
  \begin{verse}
    Les vers mangent et puis meurent,\\
    sacrificateurs des dépouilles déposées:\\
    ils rendent la vie à qui en fut dépossédé;
  \end{verse}
  \begin{verse}
    Et  s’il faut,  pour  cela, éparpiller  son  Soi à  tous  les coins  de
    l’astral charnier, qu’est-ce que cela change, en vérité?
  \end{verse}
%-------------------------------------------------------------------
\PoemTitle{L’aura des choses mortes (14 Septembre 2011)}
  \begin{verse}
    Ce coin à l’ombre d’une cabane\\
    où grille le radiateur\\
    comporte plus de forêts à vélanes\\
    qu’ils n’en faut à l’aventurier amateur.
  \end{verse}
  \begin{verse}
    Les tiges s’étiolent et servent d’armes;\\
    au fond, le son de l’escalier comme une alarme;\\
    un terrible ogre s’époumonne;\\
    de toute la jungle, de petits pleurs raisonnent.
  \end{verse}
%-------------------------------------------------------------------
\newpage
\PoemTitle{L’avenir cellophane (20 Septembre 2011)}
  \begin{verse}
    Ô siècles en bière,\\
    ô siècles de cadavres\\
    -- abreuvez moi de femmes
  \end{verse}
  \begin{verse}
    Tant d’âges qui recommencent\\
    -- peut-on s’arrêter au bord de la route?\\
    Nous voyons ici la borne \textsc{«~C’est là que tu dois rester~»}:\\
    mais que les ages à venir nous tancent!

    Et ce qui chûte et puis revient;\\
    gorgées d’eau et de sang,\\
    sueurs d’effort et de peur,\\
    tout ce qui chûte et puis revient n’a pas de sens.
  \end{verse}
  \begin{verse}
    Ô longs effets de subsistance!\\
    Ce qu’il faut d’échardes aux doigts\\
    pour être correct -- \textit{au vu de la loi},\\
    tirer du gouffre sa pitance,\\
    écrire pour encore composer,\\
    se ressouvenir pour encore oublier…
  \end{verse}
  \begin{verse}
    Ô mythes modernes, sciences en cartons criards,\\
    ô Spectacle et puis vous journaux babillards;\\
    vous avez tout pour juger et -- encore -- rien pour nous poser…
  \end{verse}
