	%*** Licence ***

%Cette œuvre est diffusée sous les termes de la license Creative Commons
%«~CC BY-NC-SA 3.0~», ce qui signifie que :

%Vous êtes libres :

  %* de reproduire, distribuer et communiquer cette création au public ;
  %* de modifier cette création.

%Selon les conditions suivantes :
    	
  %* Paternité - vous devez citer le nom de l'auteur original de la manière indiquée par l'auteur de l'œuvre ou le
		%titulaire des droits qui vous confère cette autorisation (mais pas d'une manière qui suggérerait
		%qu'ils vous soutiennent ou approuvent votre utilisation de l'œuvre).
  %* Pas d’utilisation commerciale - Vous n'avez pas le droit d'utiliser cette œuvre à des fins commerciales. 
  %* Partage des conditions initiales à l'identique - si vous transformez ou modifiez cette œuvre pour en créer une nouvelle,
  						     %vous devez la distribuer selon les termes du même contrat ou avec une
						     %licence similaire ou compatible.

%{Comprenant bien que :

  %* Renoncement} - N'importe quelle condition ci-dessus peut être retirée si vous avez l'autorisation du détenteur des droits.
  %* Domaine public - Là où l'œuvre ou un quelconque de ses éléments est dans le domaine public selon le droit applicable, ce statut
		     %n'est en aucune façon affecté par le contrat.
  %* Autres droits - d'aucune façon ne sont affectés par le contrat les droits suivants :
    %- Vos droits de distribution honnête ou d’usage honnête ou autres exceptions et limitations au droit d’auteur applicables;
      %Les droits moraux de l'auteur;
    %- Droits qu'autrui peut avoir soit sur l'œuvre elle-même soit sur la façon dont elle est utilisée, comme la publicité
      %ou les droits à la préservation de la vie privée.

%=== Note ===

%Ceci est le résumé explicatif du Code Juridique; La version intégrale du contrat est consultable ici:
%<http://creativecommons.org/licenses/by-nc-sa/3.0/legalcode>.

%-------------------------------------------------------------------
\PoemTitle{Mort à la Culture! (14 Septembre 2011)}
  \textit{Suite à la campagne de propagande anti-partage de l’\textsc{HADOPI} et consorts, je bougonne ainsi:}
  \begin{verse}
    Pour ces pandores obsédées,\\
    ce n’est pas assez que je donne mes fesses\\
    pour leurs produits,\\
    il faut aussi que je me confesse\\
    d’avoir tout reproduit.
  \end{verse}
  \begin{verse}
      Panneaux à réclame, ils vous nomment savoir.
  \end{verse}
  \begin{verse}
    \textit{Il n’y a pas d’art gratuit} , \textsc{tout doit se vendre},\\
    \textsc{Lecteurs et auditeurs sont des criminels},\\
    voilà ce qu’ils veulent faire entendre,\\
    ces maisonnées de sucre et de miel!
  \end{verse}
  \begin{verse}
    Que vous arnaquiez «~vos~» auteurs,\\
    que vous les fichiez en laisse \&\& au labeur,\\
    pour ne rien leur laisser du tout…\\
    Quelle importance \emph{pour vous?}
  \end{verse}
  \begin{verse}
    Pas de créateur sans public lui offrant ses os…\\
    Dans l’oubli se perd le rire d’Hanshow,\\
    Pas de compositeur sans personne pour l’écouter…\\
    La guerre au partage suit le vieux Mickey.
  \end{verse}
%-------------------------------------------------------------------
\PoemTitle{Les jours fastes (15 Septembre 2011)}
  \begin{verse}
    Faut-il que comme ce dieu dont j’ai oublié le nom,\\
    ma face prenne encore la dimension des monts,\\
    faut-il encore que je m’égosille\\
    à jouer le joyeux drille\\
    ~~~~ou quoi, encore?
  \end{verse}
  \begin{verse}
    Que vous faut-il, vie,\\
    pour éviter que je m’y ennuie?\\
    Que vous faut-il encore\\
    pour me faire lâcher les oxymores?
  \end{verse}
  \begin{verse}
    Quoi encore, quoi \emph{pas assez}?\\
    Avec les mêmes ingrédients, la même potion,\\
    rimons donc n’importe quoi\\
    -- ce sera un affront.
  \end{verse}
%-------------------------------------------------------------------
\PoemTitle{Retrouvailles (9 Juin 2011)}
  \begin{verse}
    La folle à cheval a enfin une meilleure compagnie\\
    \& la place du Martroi s’est emplie de vie:\\
    vieux radio-prêcheurs moroses -- voyez!\\
    La place est prise, les criminels\footnote{Et lesquels, d’ailleurs?} sont indignés.
  \end{verse}
  \begin{verse}
  Ils parlent, rient, heureux \&\& et las comme en un bistrot \\
  -- \& restent là : «~Il le faut!~»\\
  Les cervelles fument autour d’eux\\
  et les plus indignes de la rue les foudroient des yeux…

  De débats en commissions, ils ne s’étouffent point,\\
  jeunes \&\&
  \footnote{
    Petite variation de l’esperluette par rapport aux autres pièces de la section.
    Hormis sur ce vers, l’esperluette est unique; il s’agit de bien montrer qu’il
    s’agit ici, à «~jeunes et vieux~», d’un AND logique.
    }
   vieux -- humains, sont alors démocrates\\
   tandis qu’en sourdine les cowboys passent au loin,\\
   frustrés de la matraque, aux ordres des ploutocrates. 
  \end{verse}
%-------------------------------------------------------------------
\PoemTitle{Un hacker grommella et dit (12 Septembre 2011)}
  \begin{verse}
    Je ne veux pas faire de livres comme des pièces à musées\\
    mais composer pour les muses et\\
    ceux qui pourront les bidouiller,
  \end{verse}
  \begin{verse}
    des pièces amusées\\
    pour que d’autres puissent -- les barbouiller.
  \end{verse}
%-------------------------------------------------------------------
\PoemTitle{Mode Insertion (5 Septembre 2011)}
  \begin{center}
    «~Vim. The Editor.~»    
  \end{center}
  \begin{verse}
    Entre deux modes, je trouve ceci:\\
    je goûte le \texttt{code} comme la poésie;\\
    ma tête vimiesque parle le français abscon\\
    autant que le méthodique python.
  \end{verse}
  \begin{verse}
    Le shell \&\& les livres sont mes plages préférées,\\
    elles charrient imperturbables une foule de données.
  \end{verse}
  \begin{verse}
    Un beau programme est tel un calligramme:\\
    bien pensé, écrit avec intérêt,\\
    on peut se plaire à le \#commenter.
  \end{verse}
  \begin{verse}
    De la plume || du curseur,\\
    quelles sont mes palettes préférées?\\
    L’encré me plait autant qu’un \texttt{vim} zenburnisé!
  \end{verse}
  \begin{verse}
    Une fenêtre, un buffer sur l’esprit en fête,\\
    voilà un beau duo pour donner de la tête!
  \end{verse}
  \begin{verse}
    C’est assez de cette {\Huge ’}~burlesque
    \footnote{
	    \begin{quotation}
	    C’est assez de cette \emph{apostrophe} burlesque,
	    \end{quotation}
	    Le choix de remplacer le mot par le signe approprié avait émergé
	    lors de la composition pour être rejeté. Finalement, j’adopte cette idée.
	    },\\
    tout est achevé, reste la touche\\
    \texttt{<Esc>}
  \end{verse}
%-------------------------------------------------------------------
