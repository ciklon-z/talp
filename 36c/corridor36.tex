	%*** Licence ***

%Cette œuvre est diffusée sous les termes de la license Creative Commons
%«~CC BY-NC-SA 3.0~», ce qui signifie que :

%Vous êtes libres :

  %* de reproduire, distribuer et communiquer cette création au public ;
  %* de modifier cette création.

%Selon les conditions suivantes :
    	
  %* Paternité - vous devez citer le nom de l'auteur original de la manière indiquée par l'auteur de l'œuvre ou le
		%titulaire des droits qui vous confère cette autorisation (mais pas d'une manière qui suggérerait
		%qu'ils vous soutiennent ou approuvent votre utilisation de l'œuvre).
  %* Pas d’utilisation commerciale - Vous n'avez pas le droit d'utiliser cette œuvre à des fins commerciales. 
  %* Partage des conditions initiales à l'identique - si vous transformez ou modifiez cette œuvre pour en créer une nouvelle,
  						     %vous devez la distribuer selon les termes du même contrat ou avec une
						     %licence similaire ou compatible.

%{Comprenant bien que :

  %* Renoncement} - N'importe quelle condition ci-dessus peut être retirée si vous avez l'autorisation du détenteur des droits.
  %* Domaine public - Là où l'œuvre ou un quelconque de ses éléments est dans le domaine public selon le droit applicable, ce statut
		     %n'est en aucune façon affecté par le contrat.
  %* Autres droits - d'aucune façon ne sont affectés par le contrat les droits suivants :
    %- Vos droits de distribution honnête ou d’usage honnête ou autres exceptions et limitations au droit d’auteur applicables;
      %Les droits moraux de l'auteur;
    %- Droits qu'autrui peut avoir soit sur l'œuvre elle-même soit sur la façon dont elle est utilisée, comme la publicité
      %ou les droits à la préservation de la vie privée.

%=== Note ===

%Ceci est le résumé explicatif du Code Juridique; La version intégrale du contrat est consultable ici:
%<http://creativecommons.org/licenses/by-nc-sa/3.0/legalcode>.

%-------------------------------------------------------------------
\PoemTitle{Anthropomorphe (1er Août 2011)}
  \begin{verse}
    C’est peut-être vieux jeu de convier à soi l’été,\\
    mais s’il me plait de vous barber, à vous de me supporter:
  \end{verse}
  \begin{verse}
    Avec l’âge des jeunes imbéciles, je dis\\
    ce soleil là est un caméléon infini\\
    et particulier, projetant ses couleurs au lieu de s’en combler,\\
    fait de la terre une tiède armoire à fruits\\
    et un beau ventre à minerais.
  \end{verse}
  \begin{verse}
    Bel été, vous êtes \emph{la} femme:\\
    une cohorte de vous -- n’arracherait de mot juste pour vous décrire\\
    et le temps passant ne vous fait pas vieillir;\\
    vous l’été, la plus belle saison de l’homme.
  \end{verse}
%-------------------------------------------------------------------
\PoemTitle{Zomig (23 Août 2011)}
  \begin{verse}
    «~Que me vaut ces bruits dans la nuit\\
    et ce poids lourd sur mon épaule?~»\\
    Délicatement, la petite chose a surgi,\\
    ronronné à mon oreille:
  \end{verse}
  \begin{verse}
    MIAOU\\
    «~Ce n’est que moi, un chat qui miaule\\
    longtemps, quand tu ne trouves le sommeil~»
  \end{verse}
  \begin{verse}
    «~Ais-je une femme? Point du tout\\
    -- me trouvent-elles trop fou?!\\
    Mais la nuit, la solitude culmine…~»

    Alors que mon chat\\
    ~~~~me piétine,\\
    je rie\\
    ~~~~tout doucement\\
    et le chat ~~~~~~~~~~ronronne\\
    ~~~~~~~~~~~~(vexé?)\\
    farouchement.~»
  \end{verse}
  \begin{verse}
    MIAOU\\
    «~Lèves-toi! Je veux sortir!\\
    Ça suffit, cesse de dormir!~»
  \end{verse}
  \begin{verse}
    «~Ais-je un maitre, un emploi?\\
    -- Suis-je assez malhonnête pour cela?\\
    Mais la nuit -- il fait si froid\\
    -- sauf pour ces satanés rats.
    
    Alors quands les chats,\\
    ~~~~~~~~~~~~~~~~~~~~~~~~~~~~~dans les rues\\
    ~~~~vadrouillent,
    
    je ferme les yeux,\\
    ~~~~~~~~~~~~~~~~~~~~~~je n’ai plus\\
    ~~~~la trouille…~»
  \end{verse}
%-------------------------------------------------------------------
\PoemTitle{Retour à l’origine (23 Août 2011)}
  \begin{verse}
    Le film était mauvais,\\
    les acteurs jouaint -- mal,\\
    et moi je faisais -- mon gros mâle\\
    et toi -- tu surjouais…
  \end{verse}
  \begin{verse}
    Mais maintenant,\\
    ~~~~je vais -- mal,\\
    tu m’as quitté\\
    ~~~~dans mes râles…
  \end{verse}
  \begin{verse}
    Dormir -- quelle éternité!\\
    Je gèle sous la couette\\
    et je m’entête\\
    -- à te retrouver…
  \end{verse}
  \begin{verse}
    Demain (dans huit heures)\\
    ~~~~c’est si long\\
    ~~~~~~~~~~~~~~~~~~~~~que j’en pleure…
  \end{verse}
  \begin{verse}
    Si demain mourant, j’en reviens singe enfant,\\
    j’irais la faire rire en dansant…\\
    Qu’elle me nourrisse -- de bananes\\
    ou me traite -- de vieil âne,\\
    que m’importe franchement\\
    si elle diffuse son rire-encens?
  \end{verse}
%-------------------------------------------------------------------
\PoemTitle{Lac de La Source (15 Septembre 2011)}
  \begin{verse}
    Pas encore initié que je rêve de suaires d’alcools,\\
    d’une prise électrique et d’une alcôve.\\
    A-t-elle du poison sur la langue ou bien\\
    faut-il que la langue saigne ketchup?
  \end{verse}
  \begin{verse}
    Cygnes sur le bord qui pincent\\
    ce que tu sens, ce que tu vois\\
    -- naturels censeurs pleins de grâce\\
    qui n’exige la noirceur.

    Silence singulier du lac sans grand plancher\\
    et qui s’accommode de tant de bruits de fond;\\
    Interrogations obsédées sur ces femmes attablées\\
    et ces fumets courant tout leur long.
  \end{verse}
%-------------------------------------------------------------------
\newpage
\PoemTitle{Naïf (27 Août 2011)}
  \begin{verse}
    Tu peux me grogner en cadence\\
    tous les récifs de l’état amoureux,\\
    m’horrifier par toutes ces souffrances,\\
    la solitude ne vaut pas mieux.

    Tu me gonfles avec ces vents coléreux\\
    avec lesquels elle te fouette…\\
    car pour être honnête\\
    -- cette violence, je la veux!
  \end{verse}
%-------------------------------------------------------------------
\PoemTitle{Asseyons nous sous l’interrogation (12 Septembre 2011)}
  \begin{verse}
    L’attente mal à l’aise a ses intérets:\\
    les autres sont véritables -- et sans attraits?\\
    Le genre humain s’essouffle (par ses Don Juan comiques).
  \end{verse}
  \begin{verse}
    Tout devient clair, froid -- et s’applique:\\
    le poulpe est bureaucrate et sans logique.
  \end{verse}
%-------------------------------------------------------------------
\PoemTitle{L’Helène seconde (19 Septembre 2011)}
  \begin{verse}
    Cette démone paralyse mon audace\\
    et me fait détourner la tête;\\
    ce qu’elle a de plaisant je ne sais\\
    -- mais ce n’est pas mince pour une naine.
  \end{verse}
  \begin{verse}
    Ô contes judaiques\footnote{La Bible, et foutaises ultérieures.}, que vous me donnez de migraines!
      \footnote{
	Cf \textsc{Isidore Ducasse} in \emph{Poésies I}:
	\begin{quote}
	  Ô Nuits d'Young ! vous m'avez causé beaucoup de migraines ! 
	\end{quote}
      }\\
    Vos personnages honteux et fourmillant de vices éternels\\
    ne sauraient égaler les services que je ferais pour Elle…

    Vos génies buvant le sang et froids aux fêtes,\\
    je vois leur antithèse dans son nom et sa silhouette;\\
    Les parfums de votre Cantique\footnote{Le Cantique des cantiques} que je ne désapprouve pas,\\
    j’en sens de meilleurs en restant dans ses pas.
  \end{verse}
%-------------------------------------------------------------------
\PoemTitle{Tendance aux tendons (20 Septembre 2011)}
  \begin{verse}
    La frénésie du vampire m’a envenimé\\
    et il n’y a rien en elle que je ne désire manger;\\
    rompre sa peau comme une écorce parfumée\\
    -- il n’y a pas que son sang que je désire sucer.
  \end{verse}
  \begin{verse}
    Du glauque au cramoisi,\\
    c’est la couleur de teint que je n’ai choisi,\\
    l’un et l’autre me font tergiverser;\\
    Il n’y a pas jusqu’à ces os que je n’aimerais dévorer.
  \end{verse}
  \begin{verse}
    Ô explosions de chaleur et de lumière,\\
    Ô instincts cannibales et primaires.
  \end{verse}
%-------------------------------------------------------------------
\PoemTitle{Les reptiles orléanais (13 Septembre 2011)}
  \begin{verse}
    \emph{Resignados} dans l’éther\\
    cognent le fond du verre\\
    clair ainsi que la charpente\\
    anguleuse ainsi qu’une mouche.
  \end{verse}
%-------------------------------------------------------------------
\PoemTitle{Salle 054 (15 Septembre 2011)}
  \begin{verse}
    Échos cavalants dans le couloir des \emph{absents}\\
    et cellules en papier posent un grand {\Huge ?}
  \end{verse}
  \begin{verse}
    CO\up{2} matinal et routinier,\\
    la FAC expire en gargantuesques bouffées,\\
    allez -- prêts pour une offense ?!
  \end{verse}
  \begin{verse}
    Parlons alors que le cours commence.
  \end{verse}
