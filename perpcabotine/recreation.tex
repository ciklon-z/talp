%	 *** Licence ***

% Cette œuvre est diffusée sous les termes de la license Creative Commons
% «~CC BY-NC-SA 3.0~», ce qui signifie que :

% Vous êtes libres :

%   * de reproduire, distribuer et communiquer cette création au public ;
%   * de modifier cette création.

% Selon les conditions suivantes :
    	
%  * Paternité - vous devez citer le nom de l'auteur original de la manière indiquée par l'auteur de l'œuvre ou le
%	         titulaire des droits qui vous confère cette autorisation (mais pas d'une manière qui suggérerait
%	         qu'ils vous soutiennent ou approuvent votre utilisation de l'œuvre).
%  * Pas d’utilisation commerciale - Vous n'avez pas le droit d'utiliser cette œuvre à des fins commerciales. 
%  * Partage des conditions initiales à l'identique - si vous transformez ou modifiez cette œuvre pour en créer une nouvelle,
% 						      vous devez la distribuer selon les termes du même contrat ou avec une
%					              licence similaire ou compatible.

% Comprenant bien que :

%  * Renoncement - N'importe quelle condition ci-dessus peut être retirée si vous avez l'autorisation du détenteur des droits.
%  * Domaine public - Là où l'œuvre ou un quelconque de ses éléments est dans le domaine public selon le droit applicable, ce statut
%		      n'est en aucune façon affecté par le contrat.
%  * Autres droits - d'aucune façon ne sont affectés par le contrat les droits suivants :
%    - Vos droits de distribution honnête ou d’usage honnête ou autres exceptions et limitations au droit d’auteur applicables;
%    - Les droits moraux de l'auteur;
%    - Droits qu'autrui peut avoir soit sur l'œuvre elle-même soit sur la façon dont elle est utilisée, comme la publicité
%      ou les droits à la préservation de la vie privée.

% === Note ===

% Ceci est le résumé explicatif du Code Juridique; La version intégrale du contrat est consultable ici:
% <http://creativecommons.org/licenses/by-nc-sa/3.0/legalcode>.

\PoemTitle{Coup de filet (27 Avril 2012)}
  \begin{verse}
    Nuit de désir?\\
    On a des éclats de bouteille dans le crâne,\\
    c’est ainsi qu’on justifie tout:\\
    la pluie fait le beau toutou.
  \end{verse}
  \begin{verse}
    Toutou tout écorché je t’aime\\
    je mordrais tout tout tout\\
    jusqu’à t’arracher la trachée\\
    mon joli toutou écorché.
  \end{verse}
  \begin{verse}
    C’est plus haut qu’est\\
    le néné-Néant\\
    -- miam! À croquer, un déliche!\\
    \hulu{À consommer avec modération}\\
    \hulu{Verbe non contractuel}
  \end{verse}
  \begin{verse}
    Pas de pot pour la réclame:\\
    le poème mutain mue\\
    (en mauvais lézard: que de glue!)\\
    et Lecteur sait ce qu’il brâme.\\
    \hulu{Votre attention s’il vous plait}
  \end{verse}
  \begin{verse}
    C’est encore le temps de couper l’Acte A\footnote{ACTA}\\
    \hulu{Pas de gorgones pour moi, merci.}
  \end{verse}
\PoemTitle{L’Élysée du pauvre (11 Mai 2012)}
  \begin{verse}
    Le pouvoir c’est de faire baver les chiens d’envie?\\
    Presque -- mais c’est une zone réservée\\
    à ceux qui l’ont confisquée;
  \end{verse}
  \begin{verse}
    il y a d’autres chiens et d’autres hommes\\
    en carton, et si la majorité est encore rose\\
    à force de se teinter Toçan,
  \end{verse}
  \begin{verse}
    bientôt les bleuâtres seront en plus grand nombre\\
    -- les signes ont cet avantage insigne\\
    de ne pas vomir d’envie du tout,
  \end{verse}
  \begin{verse}
    juste d’exister, de s’emmurer et de revivre,\\
    compositions amenant recompositions\\
    avant l’énième décomposition
  \end{verse}
  \begin{verse}
    -- avant que tu sois nettoyé\\
    mais déjà bel et bien mort,\\
    ils seront recomposés
  \end{verse}
  \begin{verse}
    les mots, morts-vivants du langage.
  \end{verse}
\PoemTitle{Baillon net (15 Mai 2012)}
  \begin{verse}
    Je voulais écrire un poème mais je crois que je vais te laisser faire;\\
    prends une couleur, des sons, des sentiments\\
    -- prends aussi des sous\\
    car la richesse d’un vers est au delà du meilleur eldorado;\\
    prends tes racines,\\
    qu’elles soient animales, vers de terre\\
    fouillant la langue et te donnant à boire,\\
    décapites -- ou honores\\
    le tout dans quelque chose de sonore,\\
    sois libre, prends-toi toi-même:\\
    compose des vers pour aller \emph{vers} autre chose\\
    (la révolution, le massacre,\\
    une baudruche divine, que sais-je!);\\
    ne rentres pas dans toi-même\\
    comme les coquillages que cent marées assassinent;
  \end{verse}
  \begin{verse}
    prends les armes et la plume\\
    sois méfiant de tout, étranger à rien,\\
    ô toi lecteur qui compose\\
    -- hérétique de la Morosité!
  \end{verse}
\PoemTitle{Parole d’estomac (3 Juin 2012)}
  \begin{verse}
    Dans la place\\
    calme c’est la nuit\\
    le jongleur\\
    joue et danse:\\
    crépitement doré,\\
    boules citronnées,\\
    le tout simplement\\
    dans le ciel vide.
  \end{verse}
  \begin{verse}
    Un poète\\
    du village en question\\
    arrive regarde\\
    jouer aussi bien\\
    quelle honte!\\
    Il l’asticote\\
    de tous côtés\\
    le jongleur se plie\\
    mais ne sait tomber.\\
    Le verbeux dépité\\
    propose aux autres\\
    gens bien éduqués\\
    la sublime solution\\
    de le lyncher.
  \end{verse}
  \begin{verse}
    Une fourche perfore\\
    la tenue de l’acrobate\\
    dansant jonglant au pied\\
    de dépit pas défait\\
    jette ses oranges\\
    de milliers de matin
  \end{verse}
  \begin{verse}
    Qui devinrent les étoiles\\
    dans les cieux apathiques et sages\\
    oh! le poète a saisi des étincelles au passage.
  \end{verse}
\PoemTitle{Nénupharose ( 2012)}
  \begin{verse}
    Mon cœur de dépressions\\
    charriant le rythme d’hiver\\
    le second monde plein d’impressions\\
    me tape sur les nerfs.
  \end{verse}
  \begin{verse}
    Il y a dix vers et l’uniforme\\
    rouge, noir, la distance restant,\\
    la nénuphare au plus haut de ses formes,\\
    noblesse des mots, misère du ventricule jusque dans le sang.
  \end{verse}
  \begin{verse}
    Cela n’est pas mien, qu’il dorme balourd,\\
    neutralisé, exterminé jusque dans les fours\\
    -- mais illusion, il ne reste qu’illusion\\
    des mots et des songes: mon cœeur de dépressions.
  \end{verse}
\PoemTitle{TAZ, plutôt ( 2012)}
  \begin{verse}
    Banderolles en bandes\\
    fumant le violet\\
    clapotis de pas clamé\\
    dans la machoire du monde.
  \end{verse}
  \begin{verse}
    Rien n’arrivera demain,\\
    pour autant dans le bled\\
    à folle du logis orléanais,
  \end{verse}
  \begin{verse}
    ou ce ne sera que des éclairs sans grand tonnerre\\
    ni humidité ni orage que lumière\\
    les marteaux tomberont dans la saumure\\
    dissous par l’absolu des murmures.
  \end{verse}
\PoemTitle{Quatre chemins ( 2012)}
  \begin{verse}
    Il est essentiellement tragique de spoyler\footnote{Forme francisée de \emph{spoiler}} les choses\\
    ce n’est pas tout que de se spolier d’une métamorphose\\
    faudrait jongler, improviser --
  \end{verse}
  \begin{center}
    \hulu{«~Toi le pro, ici!~»}
  \end{center}
  \begin{verse}
    Fiches le camp\\
    et avale ton formulaire 27-B,\\
    on n’est pas au Brésil,\\
    on a plutôt des jolis grummeaux dans l’existence\\
    tant nous sommes bonne pâte
  \end{verse}
  \begin{verse}
    Dix sons dix sons\\
    dis pour sonner la baudruche
  \end{verse}
  \begin{verse}
    -- Tout est poétisé: Amen!
  \end{verse}
