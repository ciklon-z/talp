%	 *** Licence ***

% Cette œuvre est diffusée sous les termes de la license Creative Commons
% «~CC BY-NC-SA 3.0~», ce qui signifie que :

% Vous êtes libres :

%   * de reproduire, distribuer et communiquer cette création au public ;
%   * de modifier cette création.

% Selon les conditions suivantes :
    	
%  * Paternité - vous devez citer le nom de l'auteur original de la manière indiquée par l'auteur de l'œuvre ou le
%	         titulaire des droits qui vous confère cette autorisation (mais pas d'une manière qui suggérerait
%	         qu'ils vous soutiennent ou approuvent votre utilisation de l'œuvre).
%  * Pas d’utilisation commerciale - Vous n'avez pas le droit d'utiliser cette œuvre à des fins commerciales. 
%  * Partage des conditions initiales à l'identique - si vous transformez ou modifiez cette œuvre pour en créer une nouvelle,
% 						      vous devez la distribuer selon les termes du même contrat ou avec une
%					              licence similaire ou compatible.

% Comprenant bien que :

%  * Renoncement - N'importe quelle condition ci-dessus peut être retirée si vous avez l'autorisation du détenteur des droits.
%  * Domaine public - Là où l'œuvre ou un quelconque de ses éléments est dans le domaine public selon le droit applicable, ce statut
%		      n'est en aucune façon affecté par le contrat.
%  * Autres droits - d'aucune façon ne sont affectés par le contrat les droits suivants :
%    - Vos droits de distribution honnête ou d’usage honnête ou autres exceptions et limitations au droit d’auteur applicables;
%    - Les droits moraux de l'auteur;
%    - Droits qu'autrui peut avoir soit sur l'œuvre elle-même soit sur la façon dont elle est utilisée, comme la publicité
%      ou les droits à la préservation de la vie privée.

% === Note ===

% Ceci est le résumé explicatif du Code Juridique; La version intégrale du contrat est consultable ici:
% <http://creativecommons.org/licenses/by-nc-sa/3.0/legalcode>.

%\PoemTitle{Le k\"archer des écoles}
%\pgraph{}

\PoemTitle{2101307-91-M12}
  \begin{verse}
    Cet anonymat de pacotille\\
    chaussé-vendu comme des espadrilles\\
    qui vous dit : «~Tu seras mieux noté~»,\\
    de l’école, il ne veut pas nous sauver.
  \end{verse}
  \begin{verse}
    Moins qu’une personne, plus que haine\\
    l’heure dernière sonne -- «~Je veux ma moyenne!~»\\
    Votre bouche bave blanche\\
    (elle recrache des craies, quelle offense!)
  \end{verse}
  \begin{verse}
    Dans la gueule? «~Quedalle~»,\\
    vous bouffez les livres d’une joie cannibale.
  \end{verse}
  \begin{verse}
    Marianne-Attila passe et arrache les années\\
    de l’enfance sans craintes que l’on rêve seulement\\
    -- le pesticide est breveté, elles ne repousseront jamais,\\
    votre corps se désagrège en baillements.
  \end{verse}
\PoemTitle{Légitime distance (20 Mai 2012)}
  \begin{verse}
    Rien de nouveau sous la terre\\
    elle ne s’engraisse\\
    pas encore du cerveau de mes pairs
  \end{verse}
  \begin{verse}
    pourtant l’ogresse\\
    est claire, elle baffre et gueule\\
    elle sue la graisse la p’tite bégueule!
  \end{verse}
  \begin{center}
    \hulu{Hop Hop Hop}
  \end{center}
  \begin{verse}
    Coup de balais si tu t’approches\\
    trop de la vérité\\
    -- en poète, on ne te le pardonnerait pas.
  \end{verse}
\PoemTitle{Écriture}
  \begin{verse}
    La Loire charriait de la bière\\
    gouffre de l’encre qui va venir\\
    ~~~~~~~~fuite en avant.
  \end{verse}
  \begin{verse}
    Rien qui ne passe sans hurler\\
    lenteur affreuse des mains\\
    ~~~~~~~~dérivant jusque dans la nuit;
  \end{verse}
  \begin{verse}
    les bouées furent jetées à la terre\\
    les bateaux éventrés\\
    une hémorragie d’or.
  \end{verse}
\PoemTitle{Pygmalion (19 Mai 2012)}
  \begin{verse}
    Le ciel est bleu\\
    parce que c’est moi qui l’ai voulu ainsi;\\
    nulle réalité autre que la mienne
  \end{verse}
  \begin{verse}
    ou bien dans d’autres univers\\
    où la page a cours aussi\\
    -- te précipite dans le fleuve
  \end{verse}
  \begin{verse}
    imaginaire adoré,\\
    sculpterais-je encore ta statue\\
    figurine de verre écaillée\\
    \& te baiserais-je ainsi jusqu’à t’aimer?
  \end{verse}
\PoemTitle{Misanthrope?}
  \begin{verse}
    Sac plastique à la dérive\\
    j’étoufferais bien un enfant avec.
  \end{verse}
  \begin{verse}
    Au milieu des hommes, il y a une mare\\
    -- on y croasse allègrement comme ailleurs\\
    «~Qu’ils se taisent!~» s’agace le badaud\\
    je lui creverais bien sa poche et ses os.
\end{verse}
\PoemTitle{Ragnarök joyeux (2 Juin 2012)}
  \begin{verse}
    Ils sont malpolis de vivre\\
    avec une vallée de mots angoissés\\
    de jaillir de l’Eden et d’être laids
  \end{verse}
  \begin{verse}
    oh lagons, îles incertaines,\\
    cercles concentriques et myopes\\
    qui ne savent contenter\\
    la faim de mordre et de frapper;
  \end{verse}
  \begin{verse}
    est-ce vrai alors que j’ai un trou dans la tête\\
    comme on dit de ces braves du Quebec qui s’entêtent,\\
    est-ce vrai alors que sans être exilé\\
    je ne puis rien comprendre à vos vies d’empaffés?
  \end{verse}
  \begin{verse}
    Mes bras parlent pourtant mais en sourdine\\
    à cueillir des mots on cueille la faim\\
    et sans doute la cervelle ne dira rien\\
    lorque je décortiquerais vos âmes-sardines,
  \end{verse}
  \begin{verse}
    comme cet autre que je n’ai pas, avec des amis\\
    ou des fils ou des quidams toutous\\
    -- quelle pierre pourtant à cracher, que c’est lourd\\
    à porter comme un éléphant par le vautour…
  \end{verse}
  \begin{verse}
    Ces chiens de rues avec lesquels je me balade\\
    ont dans la gueule du feu et des sabres\\
    je suis de retour mais en Surtr\\
    venu démolir le chateau du rustre.
  \end{verse}
  \begin{verse}
    Pas de barrages policiers pour l’indifférence,\\
    on paye les saignements comme une laisse\\
    plus rien sur la peau la mémoire reste.
  \end{verse}
\PoemTitle{D’ocre et d’ailleurs}
  \begin{verse}
    Au milieu du désert végétal\\
    serré en bandes martiales\\
    Chateaudun couve\\
    comme un roc tombé\\
    dans l’océan de terre et de blé;
  \end{verse}
  \begin{verse}
    Le Loir passe ici comme la royauté:\\
    épanoui dans les cavernes calcaires\\
    il meurt doucement dans le paysage vert;
  \end{verse}
  \begin{verse}
    petite ville de ténèbres que j’aimais\\
    comme on casse la croute au Champdé\\
    plus loin les sillons retournés\\
    abreuvent les vivants d’horreurs pesticidées
  \end{verse}
  \begin{verse}
    petite ville de campagne\\
    où par deux fois je suis né\\
    c’est fini, la route coule\\
    l’asphalte est écrabouillé\\
    je retourne chez la folle du Loiret.
  \end{verse}
\PoemTitle{Sourcil de rousse}
  \begin{verse}
    La Loire est là\\
    mais le Loiret, ça, je n’en sais rien\\
    le temps tourne\\
    comme une éolienne atteinte de bronchite\\
    le courant est là\\
    mais les eaux déjà ont pris mes acides
  \end{verse}
  \begin{verse}
    Orléans mon escapade\\
    ma chambre,\\
    j’y vois un pot d’échappement\\
    -- ou autre chose: faudrait tâter le monstre:\\
    Orléanstein la carpe difforme
  \end{verse}
  \begin{verse}
    mais la Loire est là qui se gondole\\
    et qui masquée rigole\\
    c’est la chasse aux têtes\\
    d’amour sous les combles,\\
    non la carpe n’a pas tout tué
  \end{verse}
  \begin{verse}
    sur la place quoiqu’à terre\\
    la vie bouge encore des siens,\\
    dans les rues, sur les bords du fleuve\\
    tout finira en belle bleue.
  \end{verse}
