%	 *** Licence ***

% Cette œuvre est diffusée sous les termes de la license Creative Commons
% «~CC BY-NC-SA 3.0~», ce qui signifie que :

% Vous êtes libres :

%   * de reproduire, distribuer et communiquer cette création au public ;
%   * de modifier cette création.

% Selon les conditions suivantes :
    	
%  * Paternité - vous devez citer le nom de l'auteur original de la manière indiquée par l'auteur de l'œuvre ou le
%	         titulaire des droits qui vous confère cette autorisation (mais pas d'une manière qui suggérerait
%	         qu'ils vous soutiennent ou approuvent votre utilisation de l'œuvre).
%  * Pas d’utilisation commerciale - Vous n'avez pas le droit d'utiliser cette œuvre à des fins commerciales. 
%  * Partage des conditions initiales à l'identique - si vous transformez ou modifiez cette œuvre pour en créer une nouvelle,
% 						      vous devez la distribuer selon les termes du même contrat ou avec une
%					              licence similaire ou compatible.

% Comprenant bien que :

%  * Renoncement - N'importe quelle condition ci-dessus peut être retirée si vous avez l'autorisation du détenteur des droits.
%  * Domaine public - Là où l'œuvre ou un quelconque de ses éléments est dans le domaine public selon le droit applicable, ce statut
%		      n'est en aucune façon affecté par le contrat.
%  * Autres droits - d'aucune façon ne sont affectés par le contrat les droits suivants :
%    - Vos droits de distribution honnête ou d’usage honnête ou autres exceptions et limitations au droit d’auteur applicables;
%    - Les droits moraux de l'auteur;
%    - Droits qu'autrui peut avoir soit sur l'œuvre elle-même soit sur la façon dont elle est utilisée, comme la publicité
%      ou les droits à la préservation de la vie privée.

% === Note ===

% Ceci est le résumé explicatif du Code Juridique; La version intégrale du contrat est consultable ici:
% <http://creativecommons.org/licenses/by-nc-sa/3.0/legalcode>.

\PoemTitle{L’union des périodes (28 Avril 2012)}
  \begin{verse}
    I am the dream devourer\\
    who eat himself and run away\\
    -- to long lost isles.
  \end{verse}
  \begin{verse}
    I destroyed cyclopean fathers\\
    with a long pencil-sword\\
    filled with rage and vanity.
  \end{verse}
  \begin{verse}
    So, here I am!\\
    Forever free in my mind,\\
    poet? I doubt of it.
  \end{verse}
\PoemTitle{Convergence des artistes}
  \begin{verse}
    ift prdr dla vtss\\
    \& ntndr cci : loctt\\
    -- a hu ddvid;
  \end{verse}
  \begin{verse}
    prtnt la littrtre\\
    qui auve \&dté, jladpt \\
    \& jlm cmm m\&progs    \\
    -- \textsc{Ms ps lrst, nn ps lrst}
  \end{verse}
  \begin{verse}
    car vnt dmonde jm\\
    ls chos qui vvnt \& trblnt\\
    instbl \& suprm.
  \end{verse}
\PoemTitle{Dévolu hors des voluptes (14 Avril 2012)}
  Notez ou ne notez pas:
  \begin{enumerate}
    \item[I] L’important, c’est l’esprit
    \begin{enumerate}
      \item[1.] Qui se balance hors des plans
      \item[2.] Qui tremble et sonne
      \item[3.] Et qui partout foisonne;
    \end{enumerate}
    \item[II] le savoir?
    \begin{enumerate}
      \item[1.] Que cela importe peu!
      \item[2.] Montrons-le
      \item[3.] Tout nu sans bras ni jambes;
    \end{enumerate}
    \item[III] Vivez sans carte
    \begin{enumerate}
      \item[1.] Pour savoir où aller,
      \item[2.] Pour lire -- pas prémâché,
      \item[3.] Être libre, non pas noté;
    \end{enumerate}
  \end{enumerate}
  Redisons-le encore une fois: que cela soit!
\PoemTitle{Wisigographie}
  \begin{verse}
    J’aime les lettres longues,\\
    bava{\Huge ss}er jusqu’à pas d’\rotatebox{180}{heure}\\
    me [bouffée?], me bouffer pour une tête à l’envers\\
    -- des Molires mes godasses\\
    ~~~~et le tromblon dans le reste de la syntaxe.
  \end{verse}
\PoemTitle{Un chat endormi (15 Avril 2012)}
  \begin{verse}
    C’est la fête des pétards\\
    tous les dragons sont libérés,\\
    \textsc{Quoi?} Quelle AOC?\\
    On ne sait pas,\\
    ~~~~ils ne veulent pas se souvenir de la liberté.
  \end{verse}
  \begin{verse}
    Marianne est plus belle\\
    ~~~~nue dans les fumées\\
    ~~~~des pacifistes hallucinés\\
    Marianne est si belle\\
    ~~~~qu’il faudrait la toucher\\
    ~~~~-- sinon elle disparait, \textsc{Pouf!}\\
    ~~~~comme dans les dessins animés.
  \end{verse}
  \begin{verse}
    Marianne n’a pas d’enfants\\
    ~~~~c’est une idée, la subite\\
    ~~~~qui fait à sa gloire remplir les marmites,\\
    ~~~~sauter les hâches de ville,\\
    ~~~~éventrer la flicaille vile.
  \end{verse}
\PoemTitle{Plus vite que la matière? (16 Avril 2012)}
  \begin{verse}
    Non plus le mot\\
    mais le gribouillis arrondi;
  \end{verse}
  \begin{verse}
    dans le \emph{\texttt{a}, il y a la bouche\\}
    ouverte pour vivre
  \end{verse}
  \begin{center}
    \hulu{Hahaha}
  \end{center}
  \begin{verse}
    le \emph{\texttt{d}} est un chien d-put{\Large \'{E}},\\
    da,da,da, il dit toujours «~OUI~»\\
    le cul nu à l’Assemblée.
  \end{verse}
  \begin{center}
    \hulu{Nom Nom Nom}

    \hulu{Il faut le manger}
  \end{center}
  \begin{verse}
    Tout cru, comme un {\Huge U},\\
    au grill, «~à poelle le barbecue!~»
  \end{verse}
  \begin{verse}
    D-U-A-lité Signifiant / Signifié\\
    ô le beau pied de nez.
  \end{verse}
\PoemTitle{-isme}
  Morphèmes, je vous aime, vous qui voulez signifier; vous n’êtes pas avares mais
  votre économie se contente d’elle-même et ressucite la joie; les plus braves
  charges et les plus braves douleurs ne vous entaillent pas, et ne vous ne serez
  jamais rouges à l’inverse de vos réunions solennelles -- là s’arrête la comparaison
  avec les mots.

  Mais moi je veux signifier et exister encore, mieux, signifier à d’autres que moi
  ce que n’est pas le renoncement; morphèmes, à prononcer la réunion qui vous fait,
  je crois voir un unijambiste bleu qui se casse la figure sur le pavé et aussi les
  infâmes et les admirables qui morflent la haine.

  Morphèmes je vous aime, vous qui voulez signifier : c’est ici bas une intention qui
  se perd, d’avoir de l’épaisseur et du ventre où refourguer les ripailles joyeuses et
  les fêtes amoureuses.
\PoemTitle{Fragments recomposés}
  \begin{verse}
    Je les aime bien, ces p’tites choses --\\
    morter, elles sont encore là\\
    mots de hasard, mots d’épouvante,
  \end{verse}
  \begin{verse}
    Le grand Ben éclate -- mauvaise\\
    traduction je pense de l’invisible\\
    qui dévore le calme jusqu’aux charentaises,\\
    innatendue inconnue, familière pourtant
  \end{verse}
  \begin{verse}
    des mots sans sens ni sang,\\
    et qui crèvent en existant
  \end{verse}
  \begin{center}
    \hulu{En étant rentables.}
  \end{center}
\PoemTitle{Déhaidée (6 Juin 2012)}
  \begin{verse}
    Je m’éloigne de moi\\
    comme n’importe qui des pylones au loin\\
    eux qui charrient la physique\\
    d’une roue à \emph{hamster society},
  \end{verse}
  \begin{verse}
    Ce court-jus d’enrôlé\\
    n’eut pour effet que le grandiose\\
    j’ai ingéré; cependant la dose\\
    revient, revient, je veux tolkieniser
  \end{verse}
  \begin{verse}
    bâtir des cathédrales\\
    comme à l’accoutumée,\\
    juste une extension de l’agréable et du virtuel\\
    je ne serais plus au coup de dé
  \end{verse}
  \begin{center}
    \hulu{Fumble pour le poème}
  \end{center}
