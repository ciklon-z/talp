%	 *** Licence ***

% Cette œuvre est diffusée sous les termes de la license Creative Commons
% «~CC BY-NC-SA 3.0~», ce qui signifie que :

% Vous êtes libres :

%   * de reproduire, distribuer et communiquer cette création au public ;
%   * de modifier cette création.

% Selon les conditions suivantes :
    	
%  * Paternité - vous devez citer le nom de l'auteur original de la manière indiquée par l'auteur de l'œuvre ou le
%	         titulaire des droits qui vous confère cette autorisation (mais pas d'une manière qui suggérerait
%	         qu'ils vous soutiennent ou approuvent votre utilisation de l'œuvre).
%  * Pas d’utilisation commerciale - Vous n'avez pas le droit d'utiliser cette œuvre à des fins commerciales. 
%  * Partage des conditions initiales à l'identique - si vous transformez ou modifiez cette œuvre pour en créer une nouvelle,
% 						      vous devez la distribuer selon les termes du même contrat ou avec une
%					              licence similaire ou compatible.

% Comprenant bien que :

%  * Renoncement - N'importe quelle condition ci-dessus peut être retirée si vous avez l'autorisation du détenteur des droits.
%  * Domaine public - Là où l'œuvre ou un quelconque de ses éléments est dans le domaine public selon le droit applicable, ce statut
%		      n'est en aucune façon affecté par le contrat.
%  * Autres droits - d'aucune façon ne sont affectés par le contrat les droits suivants :
%    - Vos droits de distribution honnête ou d’usage honnête ou autres exceptions et limitations au droit d’auteur applicables;
%    - Les droits moraux de l'auteur;
%    - Droits qu'autrui peut avoir soit sur l'œuvre elle-même soit sur la façon dont elle est utilisée, comme la publicité
%      ou les droits à la préservation de la vie privée.

% === Note ===

% Ceci est le résumé explicatif du Code Juridique; La version intégrale du contrat est consultable ici:
% <http://creativecommons.org/licenses/by-nc-sa/3.0/legalcode>.

%\PoemTitle{La perpendiculaire cabotine (14 Avril 2012)}
%\pgraph{}

\PoemTitle{Mal entendu (15 Avril 2012)}
  \begin{verse}
    Je suis une identité remarquable,\\
    en cela, humain comme les autres,\\
    Etienne réel, «~le poète~» commenté\\
    plutôt le compas que l’équerre plastifié.
  \end{verse}
  \begin{verse}
    Les mots sont mon algèbre\\
    -- eh quoi! Le reste m’ennuie,\\
    forction de pas grand chose\\
    je sais pourtant dériver
  \end{verse}
  \begin{verse}
    vers les sons océans d’étoiles,\\
    perpendiculaires à la musique,\\
    pleins d’êtres à en sonner\\
    la cahute qui se voulait sonnet.
  \end{verse}
\PoemTitle{Rien que ça (16 Avril 2012)}
  \begin{verse}
    Ce n’est pas moi qui hurle\\
    c’est la poésie qui tonne\\
    et ma cervelle grillée au monoxyde de carbone;
  \end{verse}
  \begin{verse}
    l’Ether de lumière\\
    entourant la pute Terre\\
    belle à m’éclater la tête contre les murs.
  \end{verse}
\PoemTitle{Sequelles de datalove (25 Mai 2012)}
  \begin{verse}
    Ô, encore une fois, dévoile moi\\
    l’horizon rouge de tes données\\
    le miel de mon relationnel,
  \end{verse}
  \begin{verse}
    ma beauté à moi\\
    ma jointure à toi\\
    ma raclure pour les autres au filet de bave;
  \end{verse}
  \begin{verse}
    Je t’adore et je te hais\\
    binaire tempérament\\
    -- connaissance ou jolie fille\\
    je te fais huit fois serment!
  \end{verse}
\PoemTitle{Antiduo (3 Juin 2012)}
  \begin{verse}
    La Conie creuse la page\\
    comme une dent funéraire\\
    fouillant les casiers judiciaires
  \end{verse}
  \begin{verse}
    Oh, crois-tu payer les fantômes;
  \end{verse}
  \begin{verse}
    Ses bras curieux s’étalent --\\
    un souvenir fauché dans la vase\\
    débauche les catins astrales
  \end{verse}
  \begin{verse}
    qui perdent leur frère en une phrase;
  \end{verse}
  \begin{verse}
    Du quantique trainant dans l’athmosphère --\\
    un coup de balais a suffi,\\
    une éclaircie dans le ciel et l’air\\
    s’est redoré de phosphates jazzy.
  \end{verse}
