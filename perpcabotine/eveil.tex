%	 *** Licence ***

% Cette œuvre est diffusée sous les termes de la license Creative Commons
% «~CC BY-NC-SA 3.0~», ce qui signifie que :

% Vous êtes libres :

%   * de reproduire, distribuer et communiquer cette création au public ;
%   * de modifier cette création.

% Selon les conditions suivantes :
    	
%  * Paternité - vous devez citer le nom de l'auteur original de la manière indiquée par l'auteur de l'œuvre ou le
%	         titulaire des droits qui vous confère cette autorisation (mais pas d'une manière qui suggérerait
%	         qu'ils vous soutiennent ou approuvent votre utilisation de l'œuvre).
%  * Pas d’utilisation commerciale - Vous n'avez pas le droit d'utiliser cette œuvre à des fins commerciales. 
%  * Partage des conditions initiales à l'identique - si vous transformez ou modifiez cette œuvre pour en créer une nouvelle,
% 						      vous devez la distribuer selon les termes du même contrat ou avec une
%					              licence similaire ou compatible.

% Comprenant bien que :

%  * Renoncement - N'importe quelle condition ci-dessus peut être retirée si vous avez l'autorisation du détenteur des droits.
%  * Domaine public - Là où l'œuvre ou un quelconque de ses éléments est dans le domaine public selon le droit applicable, ce statut
%		      n'est en aucune façon affecté par le contrat.
%  * Autres droits - d'aucune façon ne sont affectés par le contrat les droits suivants :
%    - Vos droits de distribution honnête ou d’usage honnête ou autres exceptions et limitations au droit d’auteur applicables;
%    - Les droits moraux de l'auteur;
%    - Droits qu'autrui peut avoir soit sur l'œuvre elle-même soit sur la façon dont elle est utilisée, comme la publicité
%      ou les droits à la préservation de la vie privée.

% === Note ===

% Ceci est le résumé explicatif du Code Juridique; La version intégrale du contrat est consultable ici:
% <http://creativecommons.org/licenses/by-nc-sa/3.0/legalcode>.

\PoemTitle{Mort à la croissance ( 2012)}
  \hulu{Leçon de Morale}:

  «~Si ma sœur est malade, dois-je lui arracher les yeux?
  
  -- Non, ce n’est pas moral : seul le médecin est qualifié pour le faire.
  
  -- Et si je ne veux plus consommer?
  
  -- Eh bien! Les gendarmes viendront t’interner.~»

\PoemTitle{Mille aiguilles valent un sabre (2 Mai 2012)}
  \begin{verse}
    Cannibalement votre,\\
    j’adore vous raconter\\
    les bras et les jambes:\\
    \emph{stabilité} assurée…
  \end{verse}
  \begin{verse}
    Intérieur ministère\\
    organe démentiel dissimulateur de la charogne:\\
    universelle à l’odeur d’urine.
  \end{verse}
  \begin{verse}
    Cannibalement votre,\\
    mes amis, je vous dévore
  \end{verse}
  \begin{center}
    \hulu{Logique du système}

    \hulu{Excusez du peu}
  \end{center}
  \begin{verse}
    Ça court pour la sueur,\\
    ça consomme sa graisse et ça meurt:\\
    l’orgueil et la déprime au même sourire.
  \end{verse}

\PoemTitle{Un bon début (30 Avril 2012)}
  \begin{verse}
    Je vous concède qu’à mon humble avis\\
    en bien tout honneur et fort cordialement\\
    je hais la politiquesse,
  \end{verse}
  \begin{verse}
    c’est une putain agraire, semeuse de bombes\\
    qui ravit les petits et les chiens\\
    hésitant entre l’os et le bâton\\
    la république-DASS et le maton.
  \end{verse}
  \begin{verse}
    Tant de misère en rétention\\
    qu’il faudra bien brûler les prisons,\\
    jouer à l’aiguille et baudruche éditocrate\\
    restaurer le sain empire de Picrate.
  \end{verse}

\PoemTitle{La chute du paradis (15 Avril 2012)}
  \begin{verse}
    Les petits poissons dans l’eau\\
    baisent, baisent, baisent baisent,\\
    les petits poisons dans l’eau\\
    f’ront crever l’prolo…
  \end{verse}
  \begin{verse}
    Des milliers d’étoiles dans la Cour,\\
    des milliers de rêves au jardin,\\
    des milliers de vies qui s’élancent\\
    -- et un seul, un seul Cerbère!
  \end{verse}
  \begin{verse}
    Une chauve-souris, vit à Paris,\\
    elle aime bien te sucer\\
    et habite l’Élysée.
  \end{verse}

\PoemTitle{Impression (2 Mai 2012)}
  Pleine de lumière, l’eau du lac de la Source est entourée de rose, de vert
  et de violet, en de petites touches hasardeuses, entrant en résonnance
  avec les ahannements des bourdons et les altercations des bougons. Un peu
  de vent qui pousse toutes les ordures et moi dans l’écume sale et sur des
  rives pourrissantes, un peu de calme -- bien relatif : dehors c’est la 
  gueule de la ville qui ricane et avale.

  Les mots restent pourtant inutiles devant tant de délabrement et de beauté
  involontaire.

\PoemTitle{Geek Faeries (9 Juin 2012)}
  \begin{verse}
    Je me suis branché à l’Ethernet de mes semblables\\
    -- certes pas tous moutonnant\\
    mais tous bêlant d’aise formidable:\\
    ici la solitude est un mendiant;
  \end{verse}
  \begin{verse}
    Pad, manette, plume, éloquence\\
    la dextérité se mélange\\
    il y a Trololo dans l’air et aussi de la danse\footnote{Du \emph{cosplay} en fait.}\\
    pour les nuages aux franges;
  \end{verse}
  \begin{verse}
    clairement, voici un \emph{home} sans drapeaux\\
    l’hymne Leek Spin ridiculise les oripeaux\\
    des plus sérieux otakus au taquet\\
    c’est chez moi que m’ont mené ces tiquets.
  \end{verse}

%\PoemTitle{La roue du bonheur ( 2012)}
  %\begin{verse}
  %\end{verse}
  %\begin{verse}
  %\end{verse}
  %\begin{verse}
  %\end{verse}
