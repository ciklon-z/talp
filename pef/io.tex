%	 *** Licence ***

% Cette œuvre est diffusée sous les termes de la license Creative Commons
% «~CC BY-NC-SA 3.0~», ce qui signifie que :

% Vous êtes libres :

%   * de reproduire, distribuer et communiquer cette création au public ;
%   * de modifier cette création.

% Selon les conditions suivantes :
    	
%  * Paternité - vous devez citer le nom de l'auteur original de la manière indiquée par l'auteur de l'œuvre ou le
%	         titulaire des droits qui vous confère cette autorisation (mais pas d'une manière qui suggérerait
%	         qu'ils vous soutiennent ou approuvent votre utilisation de l'œuvre).
%  * Pas d’utilisation commerciale - Vous n'avez pas le droit d'utiliser cette œuvre à des fins commerciales. 
%  * Partage des conditions initiales à l'identique - si vous transformez ou modifiez cette œuvre pour en créer une nouvelle,
% 						      vous devez la distribuer selon les termes du même contrat ou avec une
%					              licence similaire ou compatible.

% Comprenant bien que :

%  * Renoncement - N'importe quelle condition ci-dessus peut être retirée si vous avez l'autorisation du détenteur des droits.
%  * Domaine public - Là où l'œuvre ou un quelconque de ses éléments est dans le domaine public selon le droit applicable, ce statut
%		      n'est en aucune façon affecté par le contrat.
%  * Autres droits - d'aucune façon ne sont affectés par le contrat les droits suivants :
%    - Vos droits de distribution honnête ou d’usage honnête ou autres exceptions et limitations au droit d’auteur applicables;
%    - Les droits moraux de l'auteur;
%    - Droits qu'autrui peut avoir soit sur l'œuvre elle-même soit sur la façon dont elle est utilisée, comme la publicité
%      ou les droits à la préservation de la vie privée.

% === Note ===

% Ceci est le résumé explicatif du Code Juridique; La version intégrale du contrat est consultable ici:
% <http://creativecommons.org/licenses/by-nc-sa/3.0/legalcode>.

\PoemTitle{Eau bouillante (8 Janvier 2013)}
  \begin{verse}
    C’est parti pour déplumer\\
    Qui?\\
    Mais le grand nabab électronique!
  \end{verse}
  \begin{verse}
    Il est pur l’air des données\\
    éclatement de couleurs\\
    multi-interfaces indexées\\
    -- vieille rengaine livresque cependant\\
    code code codex\\
    interfère le coq de la locale grenouille
  \end{verse}
  \begin{verse}
    syntaxe claire ou éclair?\\
    Paradigmes, éclatement de couleurs\\
    Souveraineté de l’Un positif\\
    et conception des seuls patrons acceptables
  \end{verse}
  \begin{verse}
    \textsc{Miracle}\\
    Pas de cyber-morts, juste le réel\\
    s’il y avait eu autre chose\\
    -- nous l’aurions senti autrement que coloré\\
    mais les hertziens foldingues étaient lucides
  \end{verse}
  \begin{verse}
    Lucides, extralucides, pas capables de s’y faire\\
    appliquer soi plus plus, ça non\\
    surplus des ventes \textsc{Mais on vous dit que ça tue}
  \end{verse}
  \begin{verse}
    Il est pur l’air de la com’syntétique\\
    appuyez sur la touche Liberté et ENTER\\
    the real world -- nouvel avatar de la vie,\\
    extraction en cours!\\
    Achevée la tâche? Ça non,\\
    ’tant à bâtir encore oh mon anarchie\\
    plus belle de toutes si ce n’est Marie
  \end{verse}
  \begin{verse}
    marri, toi, je veux bien le croire\\
    le vivre, oh non, il faut déplumer\\
    les chancres de la tête au fraiseur numérique\\
    et pas que; ex-croissance qui pousse\\
    des ailes quant à l’autre sans ex’\\
    le concept callipyge on s’en balance\\
    des vracs, des phracks par dessus la falaise atome.
  \end{verse}
  \begin{verse}
    \textsc{Continent Libre}\\
    Ether-net illarant\\
    atone insecte de joie fébrile\\
    qu’est-ce au juste qu’un radiant\\
    un joli mot, un joli son ir-radiant
  \end{verse}
  \begin{verse}
    \textsc{Continent Libre}\\
    Abolissant le rêve, bravo\\
    mon Electre junkie adorée\\
    ah qu’il est pur, l’air-données.
  \end{verse}

\PoemTitle{\texttt{while 1} (16 Janvier 2013)}
  \begin{verse}
    Boucle cassée, flux dans le potage\\
    urbain, peut-être vil d’où surnagent\\
    les démons de la langue blanche
  \end{verse}
  \begin{verse}
    Boucle cassée, échanges protocolaires\\
    apparition des personnes, toutes débouchées\\
    cela discute, n’est plus un \emph{usager}
  \end{verse}
  \begin{verse}
    Point d’interrogation sur les têtes pleines\\
    de fric, de frasques comme autant de vertes fractures\\
    pourtant le vers est en courant continu
  \end{verse}
  \begin{verse}
    Sans fin, Uscule le mage joue, inutile\\
    sans fin, il court et n’entend point\\
    les rails soudain interrompus de l’inhumanité
  \end{verse}

\PoemTitle{Une fourchette à la lumière (27 et 29 Janvier 2013)}
  \begin{verse}
    Le ver est dans le fruit -- procédurier\\
    et rit d’être si lâchement à rimer\\
    aux bits aux zedges\\
    -- ce vers est clair (quoiqu’un peu lég’)
  \end{verse}
  \begin{center}
    |
  \end{center}
  \begin{center}
    (initialisation fugit)
  \end{center}
  \begin{center}
    |
  \end{center}
  \begin{verse}
    C’est la mer des idées à peine détestées\\
    arrimant la viande à son espace configuré\\
    -- Ceci n’est pas une complainte\\
    mais une requête à digérer
  \end{verse}
  \begin{center}
    |
  \end{center}
  \begin{center}
    (Vous temporisez l’inévitable)
  \end{center}
  \begin{center}
    |
  \end{center}
  \begin{verse}
    Contact de nouveau en ligne\\
    mire comme tes données sont jolie\\
    oveursugrées de praline,\\
    plaisantes aux apaches fleuris.
  \end{verse}

\newpage
\PoemTitle{Discours du pois-chiche (6 Février 2013)}
  \begin{verse}
    Œil trotteur globe\\
    n’est point amoureux, sire\\
    mais mi-vampire microbe\\
    n’est point gavé de désirs
  \end{verse}
  \begin{verse}
    Salutations, nourris-moi\\
    chattent l’ennui, le chat, l’aimant,\\
    le pataud domestique éléphant\\
    -- ça m’tâte, \hulu{j’barris, moi}!
  \end{verse}
  \begin{center}
    \textit{$>>$}
  \end{center}
  \begin{verse}
    Si c’est pour des bras qu’on s’onnède\\
    -- la beauté t’aura\\
    -- la beauté n’attend pas\\
    toute aura vaut la peine
  \end{verse}
  \begin{center}
    DAMNÈDE
  \end{center}

\PoemTitle{AFK (25 Janvier 2013)}
  \begin{verse}
    Ah ouais, forme en clé-bourde!\\
    Égaré, éparpillé en craie sur le tableau noir\\
    j’ai des vieilles trainées de meatspace.
  \end{verse}
  \begin{center}
    |
  \end{center}
  \begin{verse}
    Fixé dans la viande, incomplet insipide\\
    j’ai perdu le fil et les cables erji livides\\
    cloué dans la nasse IRL la mélasse
  \end{verse}
  \begin{center}
    \textit{$>>$}
  \end{center}
  \begin{verse}
    Repli repli dans la coquille que je sache et puis\\
    connexion! Addition des espaces vivables\\
    superbes et des données -- mais mortes aimables;
  \end{verse}

