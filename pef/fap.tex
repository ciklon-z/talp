%	 *** Licence ***

% Cette œuvre est diffusée sous les termes de la license Creative Commons
% «~CC BY-NC-SA 3.0~», ce qui signifie que :

% Vous êtes libres :

%   * de reproduire, distribuer et communiquer cette création au public ;
%   * de modifier cette création.

% Selon les conditions suivantes :
    	
%  * Paternité - vous devez citer le nom de l'auteur original de la manière indiquée par l'auteur de l'œuvre ou le
%	         titulaire des droits qui vous confère cette autorisation (mais pas d'une manière qui suggérerait
%	         qu'ils vous soutiennent ou approuvent votre utilisation de l'œuvre).
%  * Pas d’utilisation commerciale - Vous n'avez pas le droit d'utiliser cette œuvre à des fins commerciales. 
%  * Partage des conditions initiales à l'identique - si vous transformez ou modifiez cette œuvre pour en créer une nouvelle,
% 						      vous devez la distribuer selon les termes du même contrat ou avec une
%					              licence similaire ou compatible.

% Comprenant bien que :

%  * Renoncement - N'importe quelle condition ci-dessus peut être retirée si vous avez l'autorisation du détenteur des droits.
%  * Domaine public - Là où l'œuvre ou un quelconque de ses éléments est dans le domaine public selon le droit applicable, ce statut
%		      n'est en aucune façon affecté par le contrat.
%  * Autres droits - d'aucune façon ne sont affectés par le contrat les droits suivants :
%    - Vos droits de distribution honnête ou d’usage honnête ou autres exceptions et limitations au droit d’auteur applicables;
%    - Les droits moraux de l'auteur;
%    - Droits qu'autrui peut avoir soit sur l'œuvre elle-même soit sur la façon dont elle est utilisée, comme la publicité
%      ou les droits à la préservation de la vie privée.

% === Note ===

% Ceci est le résumé explicatif du Code Juridique; La version intégrale du contrat est consultable ici:
% <http://creativecommons.org/licenses/by-nc-sa/3.0/legalcode>.

%\newpage
\PoemTitle{Trou noir (6 Décembre 2012)}
  \begin{verse}
    ~~~~Esque.
  \end{verse}
\newpage

\PoemTitle{Sorti \textit{in nihilo} (23 Janvier 2013)}
  \begin{verse}
    Poèmes fleuves ma veine cave\\
    comme des larmes additionnelles\\
    -- c’est si je ne m’abuse\\
    une question de tuyaux félins
  \end{verse}
  \begin{center}
    |
  \end{center}
  \begin{verse}
    Poèmes d’hiver qui n’ont rien de ce moi\\
    du mois de Janvier ni d’hier\\
    prévenus accusés de toute chaleur\\
    qu’offusque un pet de travers
  \end{verse}
  \begin{center}
    |
  \end{center}
  \begin{verse}
    Poèmes de ce siècle oh ma sauvagerie\\
    coulent les vers parallèles\\
    aux trous des tigres et des frangins
  \end{verse}
  \begin{center}
    |
  \end{center}
  \begin{verse}
    Ni sens ni raison ni sang\\
    le cadavre du poème rit Ktus le blafard\\
    vieil ami des sincères épisodiques.
  \end{verse}

\PoemTitle{Derlate (24 Janvier 2013)}
  \begin{verse}
    Frise de l’absolue certitude\\
    tu défiles prompte comme un ixe:\\
    tu es la beauté; équation de l’inconnue.
  \end{verse}
  \begin{verse}
    Et mes doutes et mes craintes\\
    en si peu d’instants apaisés\\
    leur cruauté? Neutralisée.\\
    Leur chaleur? Absolument anémiée,\\
    la douleur se perd dans les neiges sales\\
    d’Orléans la bête et sa cathédrale
  \end{verse}
  \begin{verse}
    Horreur, je n’ai aucun sens nasal\\
    et pourtant l’allusion est impure\\
    à la situation mais pas à Cripure\\
    elle n’a pas l’exotisme du Natal.
  \end{verse}
  \begin{verse}
    Les pays au reste, OSEF\\
    je m’en fous, ou presque\\
    le langage de l’amour\\
    encore un temps sauvera le français;\\
    mais le regard s’en moque\\
    lui la révélation partagée.
  \end{verse}
  \begin{verse}
    ’Tomane, toman, ottoman,\\
    je ne sais rien de tout cela,\\
    tout cela considéré je ne suis rien, OSEF\\
    je m’en fous, ou presque\\
    je suis dissolu, absolu\\
    au sens judiciaire (et quoi d’autre?)\\
    ses paroles me remontent encore\\
    -- à d’autres!
  \end{verse}

\PoemTitle{Et par’s en break (30 Janvier 2013)}
  \begin{center}
    Un nouveau tableau sur le mur, je suis tout étonné --
  \end{center}
  \begin{table}[h]
    \centering
    \begin{tabular}{rcl}
      Vers, prose, quelle & \multirow{3}{*}{\Huge{X}} & fragmentation notoire \\
			  &                           & \\
      c’est une           &                           & différence
    \end{tabular}
  \end{table}
  \begin{figure}[h]
    \centering
      Derlate m’a embobinairé\\
      les vers en sont tout itérés\\
      à la chaine et peu s’en fifo
  \end{figure}
  \begin{center}
    \textsc{POP!}
  \end{center}
  Ta gallerie est ouverte, il faut le croire, aux indispositions
  \begin{verse}
    (\\
    ~~à cela on est toujours ouvert\\
    ~~ainsi l’ancien, le raseur, diable vauvert\\
    ~~stolquant des âmes à n’en savoir que faire\\
    ~~del(moi ça de cet air)\\
    )
  \end{verse}

\PoemTitle{Mauvaise transaction (7 Février 2013)}
  \begin{verse}
    Derlate mignonne\\
    entre ses mains l’heure qui sonne\\
    Dé numéro 1
  \end{verse}
  \begin{verse}
    Escroquerie du langage\\
    ++ au plus bel attelage\\
    de charité-naufrage\\
    so gentle boy pour son âge
  \end{verse}
  \begin{verse}
    Or, mort au mariages sonores\\
    non -- pas encore mort aux mariages contrôlés\\
    -- oui, certes, à tout jamais
  \end{verse}
  \begin{verse}
    Cherche SVP ma glotte dans la carafe\\
    car -- STOP -- j’ai la beauté au cœur\\
    cas -- STOP -- j’ai le beurre à l’amirauté\\
    une générale oubliant le vers au porte-épigraphes.
  \end{verse}

\PoemTitle{Plaistre pour palais (3 Février 2012)}
  \begin{verse}
    Reste-t-il vraiment plus de cette soirée\\
    qu’une confusion de pavés mouillés?\\
    L’air était tendre et abaubissant --\\
    le procédé précédent vraiment réél\\
    les préjugés me donnaient des ailes
  \end{verse}
  \begin{verse}
    \textsc{Plan large sur la vie}
  \end{verse}
  \begin{verse}
    Des myoperies violacées s’échangent\\
    zéro six -- protocole de la joie glacée\\
    j’ai perdu neuf livres au change\\
    à ce compte là, j’en peux encore brûler
  \end{verse}
  \begin{verse}
    Aux pieds des murs le langage\\
    de flique en fliche\\
    achète sa tête d’affiche\\
    à l’organe flage.
  \end{verse}
  \begin{verse}
    \textsc{Autre plan autre cœur}
  \end{verse}
  \begin{verse}
    Derlate la pieuvre\\
    aux bras de coryphée\\
    chante d’un flux fatigué\\
    de banals octo’ qui m’émeuvent --
  \end{verse}
  \begin{verse}
    Recule un peu avant le crache test\\
    (à pieuvre à multiples ventouses\\
    congénère ébahi qui ne lose\\
    rien d’elle n’a vingt-quatre têtes).
  \end{verse}

