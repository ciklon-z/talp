%	 *** Licence ***

% Cette œuvre est diffusée sous les termes de la license Creative Commons
% «~CC BY-NC-SA 3.0~», ce qui signifie que :

% Vous êtes libres :

%   * de reproduire, distribuer et communiquer cette création au public ;
%   * de modifier cette création.

% Selon les conditions suivantes :
    	
%  * Paternité - vous devez citer le nom de l'auteur original de la manière indiquée par l'auteur de l'œuvre ou le
%	         titulaire des droits qui vous confère cette autorisation (mais pas d'une manière qui suggérerait
%	         qu'ils vous soutiennent ou approuvent votre utilisation de l'œuvre).
%  * Pas d’utilisation commerciale - Vous n'avez pas le droit d'utiliser cette œuvre à des fins commerciales. 
%  * Partage des conditions initiales à l'identique - si vous transformez ou modifiez cette œuvre pour en créer une nouvelle,
% 						      vous devez la distribuer selon les termes du même contrat ou avec une
%					              licence similaire ou compatible.

% Comprenant bien que :

%  * Renoncement - N'importe quelle condition ci-dessus peut être retirée si vous avez l'autorisation du détenteur des droits.
%  * Domaine public - Là où l'œuvre ou un quelconque de ses éléments est dans le domaine public selon le droit applicable, ce statut
%		      n'est en aucune façon affecté par le contrat.
%  * Autres droits - d'aucune façon ne sont affectés par le contrat les droits suivants :
%    - Vos droits de distribution honnête ou d’usage honnête ou autres exceptions et limitations au droit d’auteur applicables;
%    - Les droits moraux de l'auteur;
%    - Droits qu'autrui peut avoir soit sur l'œuvre elle-même soit sur la façon dont elle est utilisée, comme la publicité
%      ou les droits à la préservation de la vie privée.

% === Note ===

% Ceci est le résumé explicatif du Code Juridique; La version intégrale du contrat est consultable ici:
% <http://creativecommons.org/licenses/by-nc-sa/3.0/legalcode>.

\PoemTitle{Les adieux (9 Janvier 2013)}
  \begin{verse}
    Un livre de fini\\
    et c’est un peu de mort en prime\\
    -- enclore l’horizon d’un coup\\
    quelle horreur quel coût
  \end{verse}
  \begin{verse}
    Que ça se termine en douce tristesse\\
    en mélancolie, en haine en liesse\\
    c’est un trou dessous la tête en prime\\
    la fin d’une surprise, que cela se termine --
  \end{verse}
  \begin{verse}
    infecte chose que cela\\
    qui se voudra elle même encore et toujours\\
    que les pages s’amoncellent\\
    que la beauté la surprise\\
    meurent un peu, et pour une vie --\\
    infecte chose que cela
  \end{verse}
  \begin{verse}
    Pourtant voit-on des rassasiés\\
    de ces rations de fictions démêlées\\
    alors que ce monde de tarés\\
    nous crie des plaintes embrouillées?
  \end{verse}
  \begin{verse}
    Pourtant cessera-t-on un jour\\
    de raconter ces histoires (et qu’importe le vrai)\\
    pour nous apaiser le rachitique espoir\\
    dans cette bouillie suffocante et noire?
  \end{verse}

\PoemTitle{Terrain conquis (24 Novembre 2012)}
  \paragraph{I}
  \begin{verse}
    Cette cité c’est la mienne\\
    et elle n’a pas la verrue au coffre\\
    qu’elle a d’ailleurs fort plein d’épouvante
  \end{verse}
  \begin{verse}
    d’étages étanches de sable et d’or\\
    vacuité superbe c’est là que la main\\
    se pose et là où vont les mongolfières\\
    s’ils ont dépassé la côte et les bois d’éthers euphorisants
  \end{verse}
  \begin{verse}
    C’est là que va l’oublié,\\
    les violés que je vois\\
    en rapt que je vois dans les airs
  \end{verse}
  \begin{verse}
    près des murailles d’ocres qui m’appellent
  \end{verse}
  \paragraph{II}
  \begin{verse}
    Les vallons sont pleins et riants en point\\
    de mots en idées pure\\
    la couleur insoutenable\\
    qu’on trouve désolé\\
    grévé contre les murs Elena la grisaille\\
    Elenabel vos visages affolants\\
    si près de moi que je me noye.
  \end{verse}
  \paragraph{III}
  \begin{verse}
    Et maintenant je vous vois venir
  \end{verse}

\PoemTitle{Catalyseur (24 Novembre 2012)}
  \paragraph{I}
  \begin{verse}
    Dans le brouillard il reste cette image diffuse,\\
    un vieux cadre d’une couleur sans couleur\\
    impensable et pourtant, en reflux\\
    de monstres que je devais à elles
  \end{verse}
  \paragraph{II}
  \begin{verse}
    Dans le brouillard il reste\\
    la panique, la lenteur de l’exécuté,\\
    pas de peur de cris jambes en tout sens\\
    saturent les lignes
  \end{verse}
  \begin{verse}
    Dans le brouillard il reste\\
    le clone de moi, le téléphérique\\
    le bousier qui tourne boule et ne cédera pas.
  \end{verse}
  \paragraph{III}
  \begin{verse}
    C’était demain partout\\
    Et encore nulle part à l’heure d’hier
  \end{verse}
  \begin{center}
    {\Huge ..}
  \end{center}

\PoemTitle{Oniros IV (18 Décembre 2012)}
  \begin{verse}
    Au cœur de l’hiver\\
    il y avait ce soleil législatif\\
    emplâtre certain mais en colique\\
    il frappait des totems iambiques\\
    sur le front des journaliers hâtifs
  \end{verse}
  \begin{verse}
    Et puis ils vinrent
  \end{verse}
  \begin{verse}
    Faire de la politique\\
    lucides avec leurs singes décapants\\
    armant l’armoise des suicidés sanglants\\
    des ouests acidulés au rênes des déterminants
  \end{verse}
  \begin{verse}
    Et puis ils vinrent
  \end{verse}
  \begin{verse}
    À la batte, accouraient pour se faire pendre\\
    colchiques collant dans la confiture amoureuse\\
    des rêves soyeux et blancs\\
    que n’arrêtent pas les cartons de la poudreuse.
  \end{verse}

\newpage
\PoemTitle{Oniros V (3 Janvier 2013)}
  \begin{verse}
    Qu’ont-ils à fourrailler dans la cave\\
    leurs ténèbres, bric à brac de gingembre\\
    aussitôt qu’ils se lèvent en duo\\
    leurs pas sont claquants et de métal
  \end{verse}
  \begin{verse}
    Qu’ont-ils à fourrailler dans la veine\\
    de mes délires ces sangsues ces ténèbres\\
    aussitôt qu’ils déguerpissent rouges et repus\\
    ils ont encore leur odeur de cadavre
  \end{verse}
  \begin{verse}
    Qu’ont-ils à fourrailler là?\\
    Quel monstre est élevé là?\\
    Qui va frapper et reveiller la bête?
  \end{verse}
  \begin{verse}
    Car il ne faut pas croire que son cortège nous soit inconnu,\\
    non, il éblouit l’ocre à la tire\\
    au sommet des cités fleuves\\
    où se déversent les vers,\\
    là où loin des ronchonnades atomiques\\
    sourit forcée la dame de pique.
  \end{verse}

\PoemTitle{La nuit des morts-amants (25 Janvier 2013)}
  \begin{verse}
    Je n’irai pas à reculons\\
    ni en chef équestre à bascule\\
    conquerir l’amour avec mes trois galons\\
    ce n’est pas de haine que je recule
  \end{verse}
  \begin{verse}
    pas en follet que j’avance:\\
    en moi point de preux à lance\\
    juste un amoureux que tance\\
    le désir de pencher la balance.
  \end{verse}
