%	 *** Licence ***

% Cette œuvre est diffusée sous les termes de la license Creative Commons
% «~CC BY-NC-SA 3.0~», ce qui signifie que :

% Vous êtes libres :

%   * de reproduire, distribuer et communiquer cette création au public ;
%   * de modifier cette création.

% Selon les conditions suivantes :
    	
%  * Paternité - vous devez citer le nom de l'auteur original de la manière indiquée par l'auteur de l'œuvre ou le
%	         titulaire des droits qui vous confère cette autorisation (mais pas d'une manière qui suggérerait
%	         qu'ils vous soutiennent ou approuvent votre utilisation de l'œuvre).
%  * Pas d’utilisation commerciale - Vous n'avez pas le droit d'utiliser cette œuvre à des fins commerciales. 
%  * Partage des conditions initiales à l'identique - si vous transformez ou modifiez cette œuvre pour en créer une nouvelle,
% 						      vous devez la distribuer selon les termes du même contrat ou avec une
%					              licence similaire ou compatible.

% Comprenant bien que :

%  * Renoncement - N'importe quelle condition ci-dessus peut être retirée si vous avez l'autorisation du détenteur des droits.
%  * Domaine public - Là où l'œuvre ou un quelconque de ses éléments est dans le domaine public selon le droit applicable, ce statut
%		      n'est en aucune façon affecté par le contrat.
%  * Autres droits - d'aucune façon ne sont affectés par le contrat les droits suivants :
%    - Vos droits de distribution honnête ou d’usage honnête ou autres exceptions et limitations au droit d’auteur applicables;
%    - Les droits moraux de l'auteur;
%    - Droits qu'autrui peut avoir soit sur l'œuvre elle-même soit sur la façon dont elle est utilisée, comme la publicité
%      ou les droits à la préservation de la vie privée.

% === Note ===

% Ceci est le résumé explicatif du Code Juridique; La version intégrale du contrat est consultable ici:
% <http://creativecommons.org/licenses/by-nc-sa/3.0/legalcode>.

\PoemTitle{\emph{La} vice infinie}
  \begin{verse}
    Et malgré moi bouffi d’indécence\\
    toujours je repousserai le fléau\\
    débile, tant adoré, géniteur de tombeaux\\
    l’immonde artefact qu’on nomme innocence.
  \end{verse}
  \begin{verse}
    Eh! Tournera-t-il encore ce chaudron de malheurs\\
    qui invariablement, de son ton assuré\\
    la joie en peine, les rires en pleurs\\
    ironique et criard fait ainsi basculer?
  \end{verse}
  \begin{verse}
    Je ne la pleurerai pas, moi, plutôt crever!
  \end{verse}
  \begin{verse}
    À rien ni personne il ne faut se laisser réduire;\\
    quant à cette catin il faut se l’humilier --\\
    la souffrance ne dit qu’elle même;\\
    Que l’innocence vous deteste et ne dise rien.
  \end{verse}
  \begin{verse}
    Je ne la pleurerai pas, moi, plutôt lutter!
  \end{verse}

\PoemTitle{Lai réchauffé (27 Janvier 2013)}
  \begin{verse}
    À sa dame le sot chante laisse\\
    il ment comme tous, abusé de liesse\\
    il boit et beaucoup voit de fesses --\\
    qu’on s’étonne qu’en fin elle le délaisse,
  \end{verse}
  \begin{verse}
    À sa dame le sot n’offre qu’une laisse\\
    fait tant jurer serment avec adresse\\
    que son ressentiment, il dresse --\\
    qu’on s’étonne qu’en fin elle le délaisse,
  \end{verse}
  \begin{verse}
    À sa dame, le sincère n’offre que lui\\
    «~Pour demain, peut-être, absolument -- aujourd’hui!~»\\
    et elle n’exige pas fidélité de lui --\\
    réciproque, le libre amour luit,
  \end{verse}
  \begin{verse}
    À sa dame, n’offre aucun idéal\\
    et si elle est homme, c’est normal,\\
    et s’il est femme, c’est normal --\\
    il n’y a point de sexe pour l’amour banal.
  \end{verse}

\newpage

\PoemTitle{Ouroboros (20 Novembre 2012)}
  \begin{verse}
    Les têtes mongolfières\\
    ~~~~ont de grands projets
  \end{verse}
  \begin{verse}
    -- \textsc{Utopie}
  \end{verse}
  \begin{verse}
    ~~~~crient les geignards\\
    ~~~~peureux de la liberté
  \end{verse}
  \begin{verse}
    Les têtes mongolfières\\
    ~~~~rient et déclarent\\
    ~~~~«~Regardez le
  \end{verse}
  \begin{verse}
    -- \textsc{Passé}
  \end{verse}
  \begin{verse}
    S’il n’y est pas déjà\\
    ~~~~quelques utopies réalisées~»
  \end{verse}
  \begin{verse}
    «~Mensonge et
  \end{verse}
  \begin{verse}
    -- \textsc{Infamie}
  \end{verse}
  \begin{verse}
    ~~~~il ne faut rien changer\\
    ~~~~et toujours conspuer tout~»
  \end{verse}
  \begin{verse}
    -- Et aller voter?
  \end{verse}
  \begin{verse}
    -- \textsc{Utopie}!
  \end{verse}

\PoemTitle{SNeuveSmaliceS (29 Janvier 2013)}
  \begin{verse}
    Édbanguera, édbanguera pas?\\
    -- \hulu{Point de salaces ici bien sûr}\\
    juste la musique d’extase pur\\
    et des riffs en fracas
  \end{verse}
  \begin{verse}
    Nous nous soulèverons\\
    pour casser ces lois infectes\\
    cette pornographie violente\\
    consensuellement parlementaire\\
    -- il faut exploser tout net.
  \end{verse}
  \begin{verse}
    Nous ne voulons pas changer le monde
  \end{verse}
  \begin{center}
    \texttt{$>>$}
  \end{center}
  \begin{verse}
    Alors ce sera le rien commun\\
    où plus rien d’autre ne comptera\\
    -- le temps pour nous de mettre bas\\
    le réseau antimâtins
  \end{verse}
  \begin{verse}
    \hulu{Qui hurlent à la Lune?}
  \end{verse}
