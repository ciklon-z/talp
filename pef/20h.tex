%	 *** Licence ***

% Cette œuvre est diffusée sous les termes de la license Creative Commons
% «~CC BY-NC-SA 3.0~», ce qui signifie que :

% Vous êtes libres :

%   * de reproduire, distribuer et communiquer cette création au public ;
%   * de modifier cette création.

% Selon les conditions suivantes :
    	
%  * Paternité - vous devez citer le nom de l'auteur original de la manière indiquée par l'auteur de l'œuvre ou le
%	         titulaire des droits qui vous confère cette autorisation (mais pas d'une manière qui suggérerait
%	         qu'ils vous soutiennent ou approuvent votre utilisation de l'œuvre).
%  * Pas d’utilisation commerciale - Vous n'avez pas le droit d'utiliser cette œuvre à des fins commerciales. 
%  * Partage des conditions initiales à l'identique - si vous transformez ou modifiez cette œuvre pour en créer une nouvelle,
% 						      vous devez la distribuer selon les termes du même contrat ou avec une
%					              licence similaire ou compatible.

% Comprenant bien que :

%  * Renoncement - N'importe quelle condition ci-dessus peut être retirée si vous avez l'autorisation du détenteur des droits.
%  * Domaine public - Là où l'œuvre ou un quelconque de ses éléments est dans le domaine public selon le droit applicable, ce statut
%		      n'est en aucune façon affecté par le contrat.
%  * Autres droits - d'aucune façon ne sont affectés par le contrat les droits suivants :
%    - Vos droits de distribution honnête ou d’usage honnête ou autres exceptions et limitations au droit d’auteur applicables;
%    - Les droits moraux de l'auteur;
%    - Droits qu'autrui peut avoir soit sur l'œuvre elle-même soit sur la façon dont elle est utilisée, comme la publicité
%      ou les droits à la préservation de la vie privée.

% === Note ===

% Ceci est le résumé explicatif du Code Juridique; La version intégrale du contrat est consultable ici:
% <http://creativecommons.org/licenses/by-nc-sa/3.0/legalcode>.

\PoemTitle{Un studieux mâtin de Novembre (29 Novembre 2012)}
  \begin{center}
    \textit{Écrit un après-midi.}
  \end{center}
  \begin{verse}
    Vous n’avez pas le droit de ne pas savoir\\
    Et pourtant et pourtant\\
    nul part il y a l’abondance
  \end{verse}
  \begin{verse}
    Français un synonyme d’inepte\\
    et pourtant et pourtant\\
    c’est vrai
  \end{verse}
  \begin{verse}
    ailleurs que dans les têtes\\
    et pourtant et pourtant\\
    il y a l’abondance\\
    des mots et des têtes cornues
  \end{verse}
  \begin{verse}
    des mots insuffisants\\
    translatés de L1 à D2\\
    c’est le prix des œillères enrimées\\
    rhume des sonorités et des yeux\\
    et pourtant et pourtant…
  \end{verse}

\PoemTitle{Chat plein de volumes à nœuds (29 Novembre 2012)}
  \begin{verse}
    Un appendice à mon néant\\
    ces gens là tous autant qu’ils sont\\
    sont morts -- et, contradiction!
  \end{verse}
  \begin{verse}
    On les avait déjà volé\\
    pompé le sang et la pine\\
    crevé le ventre et la pensée\\
    tout ça rien qu’en épines.
  \end{verse}
  \begin{verse}
    Tous déjà un peu consommés\\
    CM en 201 pour les L2 de 2012\\
    + 98\% de matière zombie\\
    ce n’est plus un jeu\\
    il n’y a plus d’envies
  \end{verse}
  \begin{verse}
    Et les susurreurs de la veille\\
    et les voleurs du pollen immense\\
    appelaient leurs susucreurs éternels\\
    qui se perdent dans la machinerie dense.
  \end{verse}

\newpage
\PoemTitle{Formule (16 Janvier 2013)}
  \begin{verse}
    Tombe, étoffe des démons!\\
    Et regarde au delà de l’éternel hiver français\\
    ce mal à l’âme, cette horreur sur patte\\
    -- quelle morosité.
  \end{verse}
  \begin{verse}
    Tombe, étendard de la peur!\\
    Et déchante par delà les longues résistances\\
    ces braves du mot, ces souvenirs dans l’octet\\
    -- que de mots pour neutraliser…
  \end{verse}
