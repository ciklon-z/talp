%	 *** Licence ***

% Cette œuvre est diffusée sous les termes de la license Creative Commons
% «~CC BY-NC-SA 3.0~», ce qui signifie que :

% Vous êtes libres :

%   * de reproduire, distribuer et communiquer cette création au public ;
%   * de modifier cette création.

% Selon les conditions suivantes :
    	
%  * Paternité - vous devez citer le nom de l'auteur original de la manière indiquée par l'auteur de l'œuvre ou le
%	         titulaire des droits qui vous confère cette autorisation (mais pas d'une manière qui suggérerait
%	         qu'ils vous soutiennent ou approuvent votre utilisation de l'œuvre).
%  * Pas d’utilisation commerciale - Vous n'avez pas le droit d'utiliser cette œuvre à des fins commerciales. 
%  * Partage des conditions initiales à l'identique - si vous transformez ou modifiez cette œuvre pour en créer une nouvelle,
% 						      vous devez la distribuer selon les termes du même contrat ou avec une
%					              licence similaire ou compatible.

% Comprenant bien que :

%  * Renoncement - N'importe quelle condition ci-dessus peut être retirée si vous avez l'autorisation du détenteur des droits.
%  * Domaine public - Là où l'œuvre ou un quelconque de ses éléments est dans le domaine public selon le droit applicable, ce statut
%		      n'est en aucune façon affecté par le contrat.
%  * Autres droits - d'aucune façon ne sont affectés par le contrat les droits suivants :
%    - Vos droits de distribution honnête ou d’usage honnête ou autres exceptions et limitations au droit d’auteur applicables;
%    - Les droits moraux de l'auteur;
%    - Droits qu'autrui peut avoir soit sur l'œuvre elle-même soit sur la façon dont elle est utilisée, comme la publicité
%      ou les droits à la préservation de la vie privée.

% === Note ===

% Ceci est le résumé explicatif du Code Juridique; La version intégrale du contrat est consultable ici:
% <http://creativecommons.org/licenses/by-nc-sa/3.0/legalcode>.

\PoemTitle{Poème de con (6 Décembre 2012)}
  \begin{verse}
    {\Huge .}
  \end{verse}
  \begin{verse}
    Concon Con Con\\
    ConconCon Con,\\
    Con Con Con,\\
    \emph{Cococon} Con.
  \end{verse}
  \begin{verse}
    Con Concococon\\
    Coco Con Cocon:\\
    «~Concon Con Con\\
    Concococon Con?
  \end{verse}
  \begin{verse}
    -- Conco Conco Nono\\
    Concon Co NoNocon\\
    Concon \^{Con} Con
  \end{verse}
  \begin{verse}
    Nononono Conconconono\\
    Con Con Con Nocon Con\\
    Cocon? No, NoNoCon.
  \end{verse}
  \begin{verse}
    {\Huge \rotatebox{180}{M}}
  \end{verse}

\PoemTitle{Fragment de réclame (6 Décembre 2012)}
  \begin{center}
    {\Large \textsc{ISEULT 2}}

    \textit{Crâme anti-âge!}

    Avec elle, baisez enfin les hommes lépreux

    Et vivez avec eux de grandes tranches de vie!
  \end{center}
  Nos expaires vous conceillent sur les sots oints dermes à taux logiques! Nautre la bourratoire,
  fondé en KollaboRation avec \textsc{Paceteur} étant pointe sur les sujets sciantifiques que les sots
  haut rôtés ont validé comme l’hein des plus cérieux du pays, vualatélé.

\PoemTitle{Chomsky abatardi}
  \begin{center}
    \begin{tabular}{lclclcl}
      Poesie         & = & Silence Nul   & + & Son d’un Verbe &   & \\
      Silence Nul    & = & inDeterminées & + & déNominations  &   & \\
      Son d’un Verbe & = & Parfait       & + & SouPir         & + & Sous Poète \\
      inDeterminées  & = & Monde         & + & Cafards        &   & \\
      déNominations  & = & Trahison      & + & Vie            &   & \\
      Parfait        & = & Chose         & + & Inexistence    &   & \\
      SouPir         & = & MONEY!        & + & Pire!          &   & \\
      Sous Poète     & = & Poète         & - & Troll          &   & \\
    \end{tabular}
  \end{center}

