%	 *** Licence ***

% Cette œuvre est diffusée sous les termes de la license Creative Commons
% «~CC BY-NC-SA 3.0~», ce qui signifie que :

% Vous êtes libres :

%   * de reproduire, distribuer et communiquer cette création au public ;
%   * de modifier cette création.

% Selon les conditions suivantes :
    	
%  * Paternité - Vous devez citer le nom de l'auteur original de la manière indiquée par l'auteur de l'œuvre ou le
%	         titulaire des droits qui vous confère cette autorisation (mais pas d'une manière qui suggérerait
%	         qu'ils vous soutiennent ou approuvent votre utilisation de l'œuvre).
%  * Pas d’utilisation commerciale - Vous n'avez pas le droit d'utiliser cette œuvre à des fins commerciales. 
%  * Partage des conditions initiales à l'identique - Si vous transformez ou modifiez cette œuvre pour en créer une nouvelle,
% 						      vous devez la distribuer selon les termes du même contrat ou avec une
%					              licence similaire ou compatible.

% Comprenant bien que :

%  * Renoncement - N'importe quelle condition ci-dessus peut être retirée si vous avez l'autorisation du détenteur des droits.
%  * Domaine public - Là où l'œuvre ou un quelconque de ses éléments est dans le domaine public selon le droit applicable, ce statut
%		      n'est en aucune façon affecté par le contrat.
%  * Autres droits - d'aucune façon ne sont affectés par le contrat les droits suivants :
%    - Vos droits de distribution honnête ou d’usage honnête ou autres exceptions et limitations au droit d’auteur applicables;
%    - Les droits moraux de l'auteur;
%    - Les droits qu'autrui peut avoir soit sur l'œuvre elle-même soit sur la façon dont elle est utilisée, comme la publicité
%      ou les droits à la préservation de la vie privée.

% === Note ===

% Ceci est le résumé explicatif du Code Juridique; La version intégrale du contrat est consultable ici:
% <http://creativecommons.org/licenses/by-nc-sa/3.0/legalcode>.

\PoemTitle{Poème sans support (17 Janvier 2012)}
  \begin{verse}
    Nous venions de la Terre et du ciel aux étoiles sans nombre\footnote{Hésiode}\\
    et qu’arriva-t-il?\\
    Nous nous perdîmes en chemin.
  \end{verse}
  \begin{verse}
    Christ Merde -- La joie des décapiteurs -- Tchak Tchak Tchak -- Le beau 
    présentoir et Rosalie qui vous transperce les cœurs.
  \end{verse}
  \begin{verse}
    Deux ou trois veines noires\\
    le long du Rhein, le train qui roule\\
    sur les corps et les salives qui hurlaient\\
    «~La Der’ des Der’ des Der’ des Der’~».
  \end{verse}
  \begin{center}
    \textsc{Et Washington fut atomisée}
  \end{center}
  \begin{verse}
    (tout est question de détails)
  \end{verse}
  \begin{verse}
    Pour tous les plans\\
    le diktat des biens du Bien,\\
    le mal mâle\\
    et Lucie flottant dans les airs;
  \end{verse}
  \begin{verse}
    Les années rouges et belles,\\
    dorées atomiques\\
    -- Écrases ton semblable au DDT,\\
    quelques guerres et la Croissance est assurée.
  \end{verse}
  \begin{verse}
    Et puis l’oxygène qui --
  \end{verse}
  \begin{center}
    \textsc{Ça n’imprime plus}
  \end{center}
  \begin{verse}
    Plus  de  français,  de  fracas,  d’allemands,  d’enigmas  iroquoises,
    l’alpha et l’oméga du transistor;
  \end{verse}
  \begin{verse}
    L’ENIAC s’immisce sans silicium cervelle  deuzéro -- c’est pour plus
    tard, ça vous prend l’étage, mais vous  en aurez un bientôt et puis ne
    riez plus.
  \end{verse}
  \begin{center}
    \textsc{De Gaulle hâché menu Place de Grèves}
  \end{center}
  \begin{verse}
    (c’est pour bientôt --\\
    -- si vous le voulez).
  \end{verse}
\PoemTitle{Retour de mythologies (28 Décembre 2011)}
  \begin{verse}
    C’est tellement long d’attendre de se réveiller\\
    murs grisâtres aux pieds desquels\\
    -- rien du tout.
  \end{verse}
  \begin{verse}
    Exister est le luxe des bipèdes sans imaginations
    \footnote{Le pluriel est intentionnel.}\\
    pour le reste il faut jongler\\
    des pieds des mains des ventres\\
    -- et avaler la pillule.
  \end{verse}
  \begin{verse}
    Armada du désir,\\
    piétinante-péteuse-de-cables-en-soie\\
    \textsc{Je vais te man--}
  \end{verse}
  \begin{verse}
    Oh le beau conte!
  \end{verse}
  \begin{verse}
    «~Grand poème, pourquoi tant de dents?\\
    -- Pour mieux te gaver, mon enfant.~»
  \end{verse}
\PoemTitle{Point--Barre--Merde (24 Janvier 2012)}
  \begin{verse}
    C’est un tertre mort et bossu\\
    assoifé de terre et de béton\\
    picoré avec la clameur de la hache;
  \end{verse}
  \begin{verse}
    (~ranges-ci, ranges-ça,\\
    ~~plus à droite, HAIL!~~)
  \end{verse}
  \begin{verse}
    N’étant propice à rien\\
    il s’engage à tous les masques,
  \end{verse}
  \begin{verse}
    «~Dis-moi, c’est quoi quoi?~»
  \end{verse}
  \begin{verse}
    Si peu de femmes et que de vasques\\
    -- percées pour abreuver la colère\\
    et si peu de place à un prix de misère\\
    et la détresse des corps…
  \end{verse}
\PoemTitle{L’homme vert (23 Décembre 2011)}
  \begin{verse}
    J’ai divorcé avec le  sens cet après-midi; il avait mis  des bas qui ne
    lui allaient pas,
  \end{verse}
  \begin{verse}
    tableau  bizarre   que  sa  tête   à  venin  dénuée  de   plaisir  avec
    l’expression de la catin oratrice;
  \end{verse}
  \begin{verse}
    Ça suffit, ais-je dit -- je ne  veux plus jouer de toi, avec mes nouvelles
    amantes.
  \end{verse}
  \begin{verse}
    Argues donc qu’elles ont le bras long  et pas de cervelle -- mais c’est
    bien assez de membres pour me caresser,
  \end{verse}
  \begin{verse}
    Tant que je  vivrais, je saurais jouer  avec des ô et des  -a, plonger mes
    mains dans des barils de pois et de -é
  \end{verse}
  \begin{verse}
    laisses moi  en bonne compagnie,  toi et  tes os et  ton bâ triste,  tu le
    sais, j’ai divorcé, n’ai plus de  foi mais planté tant de haies entre
    toi et moi.
  \end{verse}
\PoemTitle{Décloisonnements (8 Janvier 2012)}
  \begin{verse}
    Trop de couteaux qui volent\\
    et me dépiautent en long en large\\
    Trop de bocaux d’yeux\\
    et de poignées à cervelles.
  \end{verse}
  \begin{verse}
    -- ô fascinations.
  \end{verse}
  \begin{verse}
    Je l’aime comme un fou\\
    et le suis et le sais aussi sûrement\\
    que des brumes de plâtre frigorifiées.
  \end{verse}
  \begin{verse}
    (on dit ville -- mais c’est un autre ogre)
  \end{verse}
  \begin{verse}
    bourrées de céquatre en barres\\
    ainsi que de ligaments électroniques --
  \end{verse}
  \begin{verse}
    \textsc{Booum!}
  \end{verse}
  \begin{verse}
    (la marche du progrès\\
    et le sens de l’histoire\\
    -- il y manque la mitraille)
  \end{verse}
  \begin{verse}
    se croient fermées au monde\\
    par la grisaille ordonnée de la civilisation.
  \end{verse}
\PoemTitle{La brêche (23 Janvier 2012)}
  \begin{verse}
    Tant de murs gris dans ma tête\\
    et des bois et des montagnes\\
    où l’on mange les doigts,
  \end{verse}
  \begin{verse}
    de pénombre bleue et orangeâtre\\
    -- les teintes des heureux mouroirs\\
    et les cris, et les cris…
  \end{verse}
  \begin{verse}
    Je suis dans l’océan et me noie\\
    -- la Terre a des éclats -- de mers\\
    et la figure se tord de rire;
  \end{verse}
  \begin{verse}
    \textsc{Réintégration des ombres}
  \end{verse}
  \begin{verse}
    Ô Ocre, Ô Ténèbres -- qui dégoulinez,\\
    c’est clairement la fin de vallons entiers\\
    d’Avalon, par dessus les remparts.
  \end{verse}
