%	 *** Licence ***

% Cette œuvre est diffusée sous les termes de la license Creative Commons
% «~CC BY-NC-SA 3.0~», ce qui signifie que :

% Vous êtes libres :

%   * de reproduire, distribuer et communiquer cette création au public ;
%   * de modifier cette création.

% Selon les conditions suivantes :
    	
%  * Paternité - Vous devez citer le nom de l'auteur original de la manière indiquée par l'auteur de l'œuvre ou le
%	         titulaire des droits qui vous confère cette autorisation (mais pas d'une manière qui suggérerait
%	         qu'ils vous soutiennent ou approuvent votre utilisation de l'œuvre).
%  * Pas d’utilisation commerciale - Vous n'avez pas le droit d'utiliser cette œuvre à des fins commerciales. 
%  * Partage des conditions initiales à l'identique - Si vous transformez ou modifiez cette œuvre pour en créer une nouvelle,
% 						      vous devez la distribuer selon les termes du même contrat ou avec une
%					              licence similaire ou compatible.

% Comprenant bien que :

%  * Renoncement - N'importe quelle condition ci-dessus peut être retirée si vous avez l'autorisation du détenteur des droits.
%  * Domaine public - Là où l'œuvre ou un quelconque de ses éléments est dans le domaine public selon le droit applicable, ce statut
%		      n'est en aucune façon affecté par le contrat.
%  * Autres droits - d'aucune façon ne sont affectés par le contrat les droits suivants :
%    - Vos droits de distribution honnête ou d’usage honnête ou autres exceptions et limitations au droit d’auteur applicables;
%    - Les droits moraux de l'auteur;
%    - Les droits qu'autrui peut avoir soit sur l'œuvre elle-même soit sur la façon dont elle est utilisée, comme la publicité
%      ou les droits à la préservation de la vie privée.

% === Note ===

% Ceci est le résumé explicatif du Code Juridique; La version intégrale du contrat est consultable ici:
% <http://creativecommons.org/licenses/by-nc-sa/3.0/legalcode>.

\PoemTitle{Place Quiproco (30 Novembre 2011)}
  \paragraph*{I}
    \begin{verse}
      J’ai joui du rêve de brûler mon école,\\
      comme c’était beau.\\
      Plus de professeurs de riens,\\
      de devoirs sans droits, d’humiliations…
    \end{verse}
  \paragraph*{II}
    \begin{verse}
      Pas de murs gris et à barreaux\\
      -- centrale chimique de savoir en boite,\\
      productrice de nitro’ auto-contradictoire.
    \end{verse}
    \begin{center}
      -- \textsc{On dit Dialec-Tique!}
    \end{center}
    \begin{verse}
      Ni même respect immérité\\
      aux apparatchiks du bureau d’en face,\\
      publieurs de querelles de clochers.
    \end{verse}
    \begin{center}
      -- \textsc{On dit Recherche!}
    \end{center}
    \begin{verse}
      Ô la belle blague des apostrophes lyriques,\\
      de comparer Sophocle le décrépi à Anouilh le nucléaire,\\
      prendre des pincettes en parlant de bois d’arbre\\
      et dire de faire pourtant
    \end{verse}
    \begin{center}
      -- \textsc{Sujet, Verbe, Complément!}
    \end{center}
  \paragraph*{III}
    \begin{verse}
      J’ai joui du rêve de brûler mon école\\
      pour qu’elle revienne au peuple -- cad, à la rue,\\
      une éducation de mendiants qui ne mendient pas.
    \end{verse}
    \begin{verse}
      Cramer jusqu’aux dernières plumes\\
      les ailes des grandes écoles de toutes les nations,\\
      nourries de l’arbre hiérarchique,
    \end{verse}
    \begin{verse}
      rêver, peut-être, mais rendre libre et donner envie,\\
      envie de ces mers chaudes de lettres et de mots\\
      et des choses à sentir, à vous gonfler…
    \end{verse}
  \begin{center}
    -- \textsc{Utopiste!}
  \end{center}
\PoemTitle{Kim Jong ’zy (2 Janvier 2012)}
  \begin{verse}
    Populace au grand choeur d’éxécutions\\
    tout le reste en toi n’est que télévision\\
    ne détournes plus les yeux du présentateur\\
    -- grâce à lui, le monde va mieux;
  \end{verse}
  \begin{verse}
    De leurs chaires en plastoc’ lubrifié\\
    à grands coups de reins ils t’informent\\
    -- tel tyran est tombé, nous sommes la Liberté,\\
    oh oui, encore, encore plus d’audimat résigné!
  \end{verse}
  \begin{verse}
    Le plâtre qui les forme coule de leurs oreilles\\
    car ils sont amorphes ainsi qu’une bouche\\
    -- ces membres, ils te l’ont volé à la présidencielle;\\
    ces messieurs, jettes leur sur le chef une babouche.
  \end{verse}
  \begin{verse}
    Je n’irais pas voter en Mai\\
    mon cul est vierge de ces inepties\\
    la mâchoire lourde tu somnoles:\\
    j’aimerais te voir en démocratie.
  \end{verse}
\PoemTitle{Comme un goût de cadavre (4 Janvier 2012)}
  \begin{verse}
    Mais l’âge, que cela veut-il dire?
  \end{verse}
  \begin{verse}
    J’ai connu un résigné -- il  vomissait des algues, un peu comme lorsque
    vous me parliez, le fiel ironique dans la bouche.
  \end{verse}
  \begin{verse}
    Et il parlait ainsi que le font\\
    les générateurs programmés et réguliers,\\
    avec forme ampoulée et pas de fond;
  \end{verse}
  \begin{verse}
    il avait perdu à l’âge qui était le mien\\
    toutes les saveurs des gloires des défaites comme l’écho,\\
    mais l’âge, que cela veut-il dire?
  \end{verse}
  \begin{verse}
    Peu importe la révolution\\
    de la Terre par le Soleil\\
    -- il ne faut que des révolutions\\
    des hommes par les hommes.
  \end{verse}
\PoemTitle{Made in Connards (24 Janvier 2012)}
  \begin{verse}
    Outrevent des mers oblongues,\\
    mon nouveau mot de cire -- héraclitéen;\\
    l’humain sôt fait des trous\\
    ou fait du flou de mots\\
    -- bélier du blockhaus verbal.
  \end{verse}
  \begin{verse}
    Ô Joie,\\
    ~~~~\textsc{C’est la Guerre}
  \end{verse}
  \begin{verse}
    (~~Depuis qu’atomes animés nous sommes\\
    ~~~entre nous la vie de partager somme\\
    ~~~et fidèles à nos molécules réfléchies\\
    ~~~-- tous ces sales moines, nous les avons infléchis~~~)
  \end{verse}
  \begin{center}
    «~Mais moi, ne puis-je rien faire?~»
  \end{center}
  \begin{verse}
    Si tu es un oiseau du vertige possédé,\\
    pourquoi ne pas tenter du gouffre dégringoler?\\
    -- C’est en tombant que l’on sait voler;\\
    Rien de plus facile que la mort sans volonté.
  \end{verse}
\PoemTitle{Bras tendus (4 Janvier 2012)}
  \begin{verse}
    Un jour tu seras libre\\
    mais ce jour est déjà tien\\
    -- tu peux toujours ouvrir une voie,
  \end{verse}
  \begin{verse}
    Et combler les vides, vider les décharges;
  \end{verse}
  \begin{verse}
    Non plus prêcher aux déserts\\
    et signer les décharges\\
    mais goûter au dessert\\
    de la libre charge.
  \end{verse}
  \begin{verse}
    Car il faut bien vivre, parait-il;
  \end{verse}
  \begin{verse}
    Vis donc comme cela t’arrange\\
    ainsi que comme cela dérange\\
    les troupeaux amers et vers d’âtres.
  \end{verse}
  \begin{verse}
    Si la liberté a le goût d’amande,\\
    Si la liberté a le goût du sang,\\
    juge donc de la saveur des pleurs,\\
    subis les rythmes tyranniques et stressants;
  \end{verse}
  \begin{verse}
    Et tu verras alors
  \end{verse}
  \begin{verse}
    que tu n’auras plus peur;
  \end{verse}
  \begin{verse}
    En ce jour tu es libre, déjà\\
    et ce jour est tien -- encore.
  \end{verse}
%\PoemTitle{Conspis hors de nos vies (14 Décembre 2011)}
% Finalement, le poème  anti-conspirationnistes, je ne le mets  pas. Ils savent
% que je suis un israelien  réptilien. Afin d’éviter leurs critiques, j’ai
% pris les devants.  J’ai donc monté une conspiration contre  eux. S’ils ne
% peuvent me critiquer, c’est parce que je suis très fort.
