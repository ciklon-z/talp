%	 *** Licence ***

% Cette œuvre est diffusée sous les termes de la license Creative Commons
% «~CC BY-NC-SA 3.0~», ce qui signifie que :

% Vous êtes libres :

%   * de reproduire, distribuer et communiquer cette création au public ;
%   * de modifier cette création.

% Selon les conditions suivantes :
    	
%  * Paternité - Vous devez citer le nom de l'auteur original de la manière indiquée par l'auteur de l'œuvre ou le
%	         titulaire des droits qui vous confère cette autorisation (mais pas d'une manière qui suggérerait
%	         qu'ils vous soutiennent ou approuvent votre utilisation de l'œuvre).
%  * Pas d’utilisation commerciale - Vous n'avez pas le droit d'utiliser cette œuvre à des fins commerciales. 
%  * Partage des conditions initiales à l'identique - Si vous transformez ou modifiez cette œuvre pour en créer une nouvelle,
% 						      vous devez la distribuer selon les termes du même contrat ou avec une
%					              licence similaire ou compatible.

% Comprenant bien que :

%  * Renoncement - N'importe quelle condition ci-dessus peut être retirée si vous avez l'autorisation du détenteur des droits.
%  * Domaine public - Là où l'œuvre ou un quelconque de ses éléments est dans le domaine public selon le droit applicable, ce statut
%		      n'est en aucune façon affecté par le contrat.
%  * Autres droits - d'aucune façon ne sont affectés par le contrat les droits suivants :
%    - Vos droits de distribution honnête ou d’usage honnête ou autres exceptions et limitations au droit d’auteur applicables;
%    - Les droits moraux de l'auteur;
%    - Les droits qu'autrui peut avoir soit sur l'œuvre elle-même soit sur la façon dont elle est utilisée, comme la publicité
%      ou les droits à la préservation de la vie privée.

% === Note ===

% Ceci est le résumé explicatif du Code Juridique; La version intégrale du contrat est consultable ici:
% <http://creativecommons.org/licenses/by-nc-sa/3.0/legalcode>.

\PoemTitle{À tous les modes, tous les temps (3 Janvier 2012)}
  \begin{verse}
    Treize heures -- 21 ans toujours ça de pris,\\
    de moins de couteaux dans le cibouleau;\\
    la belle légende des dates retenues!
  \end{verse}
  \begin{verse}
    Avant toujours plus et le \texttt{reset} impossible\\
    «~On n’a vingt ans qu’une fois~»\\
    mais être, nous le verbalisons jusqu’à la fin\\
    -- c’est un dicton de l’irascible.
  \end{verse}
\PoemTitle{Amor ou presque (13 Janvier 2012)}
  \begin{verse}
    Remueurs d’éclipse et bras de fumées\\
    prières au grand Oris\\
    jamais exhumées.
  \end{verse}
  \begin{verse}
    Donnézanmoi dix\\
    pour gerber les étoiles\\
    et les délices des mots doux.
  \end{verse}
  \begin{verse}
    Cette nuit était un nouveau matin\\
    synapses toutes ensoleillées\\
    un seul bond pour toutes les enlever.
  \end{verse}
  \begin{verse}
    -- coriasses harassés\\
    un seul fleuve, tant de gouttes et de courant\\
    -- les femmes de toute cité…
  \end{verse}
\PoemTitle{L’air désséché de la grammaire empaillée (9 Janvier 2012)}
  \begin{verse}
    Ah, serais-je si sôt\\
    pour ne point comme vous connaitre un mot\\
    (et aimer ainsi qu’un poulpe des mers)\\
    de ce que vous appellez «~grammaire~»?
  \end{verse}
  \begin{verse}
    «~Haro sur le baudet~», dis-je,\\
    il n’y a pas de science mais un abattoir\\
    et du haut de mes nains vers j’exige\\
    -- que vos syntagmes aillent se faire voir.
  \end{verse}
  \begin{verse}
    chez céméssieux les romains zélés,\\
    grecs désargentés,\\
    montrer leur profil en dandinant du cul\\
    (ainsi que font les ours de rue).
  \end{verse}
  \begin{verse}
    Apparaissent cétacés et céhodés\\
    que voila les vers les plus beaux sous vide,\\
    empaquetés et llyophilisés…
  \end{verse}
\PoemTitle{Prométhée nihiliste (30 Décembre 2011)}
  \begin{verse}
    Puis les voilà créés\\
    et je suis bactérie à leurs côtés;
  \end{verse}
  \begin{verse}
    microbes sans texture ni odeur\\
    encenseurs pourtant de la myrte de jadis\\
    plus beaux que tous corps et pubis.
  \end{verse}
  \begin{verse}
    La flamme sitôt peinte\\
    me dévore bien plus la peau\\
    et le récit de la joie défunte\\
    me boute jusqu’au caniveau.
  \end{verse}
  \begin{verse}
    Créer -- \textsc{Oui!}\\
    Et pourtant que de honte
      \footnote{Cf. le «~complexe de Prométhée~» de Sanders}
      ressassée\\
    face à ces mots expectorés\\
    -- les veinards n’ont pas de vie.
  \end{verse}
\PoemTitle{Ultimatictions (13 Janvier 2012)}
  \begin{verse}
    Vos idées sont aussi noires\\
    que sales vos habits\\
    -- crevez pères de tout acabit!
  \end{verse}
  \begin{verse}
    Il fallait à l’enfant un grand monde moins \emph{cinglé}\\
    (il n’en fallait pas autant pour ne pas être cuisinier;\\
    celui qui à son goût assaisonne ses fils\\
    n’a que le langage du fumier).
  \end{verse}
  \begin{verse}
    La tête en berne sans Alexandrie\\
    (et la haine de tout ce qui fait joie et comédie)\\
    n’emergea du continuel brouillard\\
    que par les yeux de la NénuPhare.
  \end{verse}
  \begin{verse}
    À qui tout était interdit\\
    viendra prendre les bras et l’envol\\
    -- la jeunesse, l’avantage des fols\\
    et l’excitation des castrés instruits!
  \end{verse}
  \begin{verse}
    Tu m’en donneras beaucoup d’\emph{Innocent Fools}
      \footnote{C’est une chanson sous licence libre de \emph{The Rest}.}\\
    -- le reste est de mon choix,\\
    ouvrir la plume et faire pleurer le carnet\\
    jusqu’à la nouvelle louse.\footnote{\emph{loose} anglais francisé}
  \end{verse}
\PoemTitle{Jamais comme lui (14 Janvier 2012)}
  \begin{verse}
    Enfants de ma tête et de mes sueurs\\
    vous n’aurez pas ce que je sais des pères,\\
    pour vous, je veux et voudrais le meilleur,\\
    vous n’aurez pas peur de moi j’espère.
  \end{verse}
  \begin{verse}
    Mais le temps est un salaud,\\
    il nous change et nous fait plus mauvais encor;
  \end{verse}
  \begin{verse}
    Les primptemps sont beaux et féminins à la rime,
      \footnote{Vu en recopiant: Proche des \emph{Colchiques} d’\textsc{Appollinaire}}\\
    le vieux tyran frappe, frappe aussi fort;
    \footnote{
      La répétition de frappe était à l’origine une erreur de recopie.
      Mais en fait, en ajoutant une virgule, c’est pas mal.
    }
  \end{verse}
  \begin{verse}
    Les fils réparent les torts par l’escrime\\
    et le déni bat ses ailes, jappe en corniaud.
  \end{verse}
