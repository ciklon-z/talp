%	 *** Licence ***

% Cette œuvre est diffusée sous les termes de la license Creative Commons
% «~CC BY-NC-SA 3.0~», ce qui signifie que :

% Vous êtes libres :

%   * de reproduire, distribuer et communiquer cette création au public ;
%   * de modifier cette création.

% Selon les conditions suivantes :
    	
%  * Paternité - Vous devez citer le nom de l'auteur original de la manière indiquée par l'auteur de l'œuvre ou le
%	         titulaire des droits qui vous confère cette autorisation (mais pas d'une manière qui suggérerait
%	         qu'ils vous soutiennent ou approuvent votre utilisation de l'œuvre).
%  * Pas d’utilisation commerciale - Vous n'avez pas le droit d'utiliser cette œuvre à des fins commerciales. 
%  * Partage des conditions initiales à l'identique - Si vous transformez ou modifiez cette œuvre pour en créer une nouvelle,
% 						      vous devez la distribuer selon les termes du même contrat ou avec une
%					              licence similaire ou compatible.

% Comprenant bien que :

%  * Renoncement - N'importe quelle condition ci-dessus peut être retirée si vous avez l'autorisation du détenteur des droits.
%  * Domaine public - Là où l'œuvre ou un quelconque de ses éléments est dans le domaine public selon le droit applicable, ce statut
%		      n'est en aucune façon affecté par le contrat.
%  * Autres droits - d'aucune façon ne sont affectés par le contrat les droits suivants :
%    - Vos droits de distribution honnête ou d’usage honnête ou autres exceptions et limitations au droit d’auteur applicables;
%    - Les droits moraux de l'auteur;
%    - Les droits qu'autrui peut avoir soit sur l'œuvre elle-même soit sur la façon dont elle est utilisée, comme la publicité
%      ou les droits à la préservation de la vie privée.

% === Note ===

% Ceci est le résumé explicatif du Code Juridique; La version intégrale du contrat est consultable ici:
% <http://creativecommons.org/licenses/by-nc-sa/3.0/legalcode>.

\PoemTitle{Allegro (23 Décembre 2011)}
  \begin{verse}
    Pas en fuite, négation de ce qui nous reste -- soyons ainsi!
  \end{verse}
  \begin{verse}
    Courir dans le désordre ou s’entasser dans les fifo
    \footnote{
      Jargon de programmeur. Structure de données qui traite les données
      selon un modèle «~First In First Out~».
      Se référer à \href{https://fr.wikipedia.org/wiki/File_(structure_de_données)}{Wikipedia}
    }
    des émetteurs de cuivre -- peu pour nous,\\
    il faut s’emporter, déborder.
  \end{verse}
  \begin{verse}
    Le temps de regarder tes yeux\\
    la nuit déjà tombée\\
    a brisé les gonds,
  \end{verse}
  \begin{center}
    \textsc{Éclatement}
  \end{center}
  \begin{verse}
    Ô le théâtre de l’Éclipse aux clichés d’enfer, de châtié langage
    et d’adjectifs antéposés  si loin qu’on ne voit plus  rien à travers
    que de la brume de poète;
  \end{verse}
  \begin{verse}
    C’est fini, il faut faire le deuil, c’était si beau la vie, \textsc{Il
    faut bien mourir} et payer pour cela
  \end{verse}
  \begin{verse}
    Les orpailleurs de bas-étage et autres empaffés d’arrières mondes
    à aspirer hebdomadairement.
  \end{verse}
  \begin{verse}
    et-pas-plus, avec-modération.
  \end{verse}
\PoemTitle{Steampunk FTW (14 Décembre 2011)}
  \begin{verse}
    Oh le nerd à vapeur\\
    se boute hors la Terre\\
    en astronef de ferraille à rouille!
  \end{verse}
  \begin{verse}
    «~Est-ce là un nouvel aérolithe?\\
    -- Non, ce n’est que du steampunk\\
    ~~~~-- à la sauce satellite!~»
  \end{verse}
  \begin{verse}
    Donnez moi tout l’oxygène des temps vaporisés\\
    de l’Art Nouveau réifié.
  \end{verse}
\PoemTitle{Étimakoyton Berkwaga (24 Janvier 2012)}
  \begin{verse}
    J’ai mes Amériques à moi\\
    et le Vieux Monde secondaire de peu de nostalgie
  \end{verse}
  \begin{center}
    -- CRÈVES CRÈVES\\
    ~~~CRÈVES CRÈVES !
  \end{center}
  \begin{verse}
    Des murs baignés, dans le Soleil\\
    et d’autres si gris au fond de la caboche\\
    -- à croire les mâtinées de l’un fantoche,\\
    les verdâtres refont leurs entrées;
  \end{verse}
  \begin{verse}
    Des pavés à détailler longuement\\
    et d’autres à jeter -- traitons \textit{Universal} justement,\\
    le rire aux bouches et musique aux oreilles;
  \end{verse}
  \begin{flushright}
    Latences-tensions-fonds de pensions -- voleurs, vous êtes des voleurs,
    vous, ayants-droits -- Babels clairsemées et les mélodies du cosmos
    ouvrier.
  \end{flushright}
\PoemTitle{Global Crotteur (23 Décembre 2011)}
  \begin{center}
     \textsc{Esthétique du Sabotage!}
  \end{center}
  \begin{verse}
    Je veux des milliers de vers, certes\\
    ne disant pas autre chose que vie\\
    (et débordement alcoolique).
  \end{verse}
  \begin{verse}
    allumons\\
    ~~~~~~~~~~~~la\\
    ~~~~~~~~~~~~~~~mèche\\
    ~~~~~~~~~~~~~~~~~~~~~~~~pendant\\
    ~~~~~~~~~~~~~~~~~~~~~~~~~~~~~~~~~~~que le grammairien
  \end{verse}
  \begin{verse}
    \textsc{Dissèque} l’écorce des mots pour\\
    la soupe qu’il sait seulement avaler.\\
    Le texte contre l’auteur, pure poésie\\
    -- l’utopie pour vous a l’odeur du cadavre.
  \end{verse}
  \begin{verse}
    Sourions avec l’Apocalypse\\
    et la tranquillité des caméras\\
    positivons au carrefour\\
    de la pub et d’l’ennui,
  \end{verse}
  \begin{verse}
    Oui, dansons sur les beaux charniers\\
    Comme c’est beau le néant décoré,\\
    comme un crâne des fleurs dans le nez.
  \end{verse}
  \begin{center}
    \textsc{Hahaha}
  \end{center}
  \begin{verse}
    «~En vérité, je vous le dis,\\
    (car ces phrases annoncent toute bave)\\
    l’homme n’est rien sans l’argent\\
    (ou dieu minuscule, ou Viagra, peu importe)~»
  \end{verse}
  \begin{verse}
    Du néant je veux reparaitre,\\
    de ce vide, je veux manufacturer des montagnes\\
    et bouter les marchands \emph{de} temples
  \end{verse}
  \begin{verse}
    hors de toutes les têtes\\
    et n’y rien placer du tout en bon lobbyiste poète,\\
    il faut bien violer les consciences pour crier au loup.
  \end{verse}
  \begin{center}
    \textsc{Et la campagne présidencielle fit un Flop.}
  \end{center}
\PoemTitle{Tequila (1er Janvier 2012)}
  \begin{verse}
    C’est la vie pourtant et il n’y aura pas de cadavres\\
    veines aux bras du couchant\\
    le rouge étoilé qui trône triomphant.
  \end{verse}
  \begin{verse}
    Une grande fosse pour moi\\
    que de chaleur\\
    vertige des verres verts\\
    l’art c’est la vie\\
    la vie c’est vicelard.
  \end{verse}
  \begin{verse}
    Les abélards ont de vieux habits\\
    c’est eux l’autre pôle des enfers\\
    les glaces du cœur\\
    brisures de la joie coque de noix.
  \end{verse}
  \begin{verse}
    Jolis chats-tapis aux petites pattes\\
    joyeux de leurs offrandes de pacotille\\
    le sang et les pleurs les émoustille
  \end{verse}
  \begin{verse}
    Maigres félins aux cirques dévolus\\
    aux paraboles bêtes et circonvolues;
  \end{verse}
  \begin{verse}
    Tu peux rire, je me suis mis à table.
  \end{verse}
\PoemTitle{Freins requis (24 Janvier 2012)}
  \begin{verse}
    Crevure du fond des limbes
  \end{verse}
  \begin{verse}
    (~Pas de poème dessus!\\
    ~~-- je ne sais pas ce que c’est,\\
    ~~je ne sais pas non plus ce que je dis~~)
  \end{verse}
  \begin{verse}
    Lovecraft, ou l’angoissé primordial\\
    j’aime tes dieux sans les adorer\\
    -- donner les maux à l’homme,\\
    quoi de plus banal! Allons, qu’il souffre un peu!
  \end{verse}
  \begin{verse}
    Lui qui pourtant ne vaut rien --
  \end{verse}
  \begin{verse}
    (~Je n’y crois pas\\
    ~~et ne connais point la suite…~~)
  \end{verse}
  \begin{verse}
    Crevure du fond des limbes
  \end{verse}
  \begin{verse}
    Contes hébraiques vous m’ennuyez beaucoup\\
    -- rien de nouveau là-dedans\\
    et même pas des livres, pourtant…
  \end{verse}
  \begin{verse}
    Crevure du fond des limbes,\\
    -- Ce que je suis et que j’ignore.
  \end{verse}
