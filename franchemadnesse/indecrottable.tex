%	 *** Licence ***

% Cette œuvre est diffusée sous les termes de la license Creative Commons
% «~CC BY-NC-SA 3.0~», ce qui signifie que :

% Vous êtes libres :

%   * de reproduire, distribuer et communiquer cette création au public ;
%   * de modifier cette création.

% Selon les conditions suivantes :
    	
%  * Paternité - Vous devez citer le nom de l'auteur original de la manière indiquée par l'auteur de l'œuvre ou le
%	         titulaire des droits qui vous confère cette autorisation (mais pas d'une manière qui suggérerait
%	         qu'ils vous soutiennent ou approuvent votre utilisation de l'œuvre).
%  * Pas d’utilisation commerciale - Vous n'avez pas le droit d'utiliser cette œuvre à des fins commerciales. 
%  * Partage des conditions initiales à l'identique - Si vous transformez ou modifiez cette œuvre pour en créer une nouvelle,
% 						      vous devez la distribuer selon les termes du même contrat ou avec une
%					              licence similaire ou compatible.

% Comprenant bien que :

%  * Renoncement - N'importe quelle condition ci-dessus peut être retirée si vous avez l'autorisation du détenteur des droits.
%  * Domaine public - Là où l'œuvre ou un quelconque de ses éléments est dans le domaine public selon le droit applicable, ce statut
%		      n'est en aucune façon affecté par le contrat.
%  * Autres droits - d'aucune façon ne sont affectés par le contrat les droits suivants :
%    - Vos droits de distribution honnête ou d’usage honnête ou autres exceptions et limitations au droit d’auteur applicables;
%    - Les droits moraux de l'auteur;
%    - Les droits qu'autrui peut avoir soit sur l'œuvre elle-même soit sur la façon dont elle est utilisée, comme la publicité
%      ou les droits à la préservation de la vie privée.

% === Note ===

% Ceci est le résumé explicatif du Code Juridique; La version intégrale du contrat est consultable ici:
% <http://creativecommons.org/licenses/by-nc-sa/3.0/legalcode>.

\PoemTitle{Doubtcomic (28 Décembre 2011)}
  \begin{verse}
    À ne pas me répondre\\
    Baladez\\
    Baladez à fond monsieur,
  \end{verse}
  \begin{verse}
    Décembre se meurt\\
    Bientôt comètes et catas’\\
    déferleront\\
    -- en fait, tout reste si plat\\
    pour l’ovation.
  \end{verse}
  \begin{verse}
    Nouveau tour de manège,\\
    je suis une \textsc{toupie en bourrique},\\
    à ne pas répondre\\
    au grapin sur la corniche\\
    reviendra\\
    l’heurt fait aux bonniches.
  \end{verse}
  \begin{verse}
    Pluies d’étoiles qui sanglotent\\
    et le reste qui me crache dessus
  \end{verse}
  \begin{verse}
    -- Ô comme c’est long d’attendre d’être déçu.
  \end{verse}
\PoemTitle{Un sacamot troué (8 Janvier 2012)}
  \begin{verse}
    Il faut voler les mots à l’Académie\\
    mais à toi, je ne t’ai rien demandé,\\
    pourquoi voler les miens?
  \end{verse}
  \begin{verse}
    Il faut les amalgamer, les faire respirer,\\
    en créer, répandre de vieux idiomes\\
    -- un travail d’artiste-peintre.
  \end{verse}
  \begin{verse}
    (de façades)
  \end{verse}
  \begin{verse}
    Et à celles-ci, nous avons les Interfaces\\
    -- ces beautés trahies sont à refactorer\\
    (ainsi qu’on fait d’un livre corné,\\
    en le caressant tendrement pour le remettre droit).
  \end{verse}
  \begin{verse}
    Et pourtant,
  \end{verse}
  \begin{center}
    \textsc{Foutaises}
  \end{center}
  \begin{verse}
    Il n’y a plus de mots là où tu es,\\
    de Poésie nulle part où tu n’es pas,\\
    et seulement la poésie où tu fus.
  \end{verse}
\PoemTitle{Estrange Delaite (13 Janvier 2012)}
  \begin{verse}
    C’était curieux,\\
    la matinée toute mâtinée de blanc\\
    les spectres affreux des enfants;
  \end{verse}
  \begin{verse}
    ocre de peau que j’attendais\\
    déjà baisée cent fois,\\
    ogre de peu -- je la tendais\\
    mais on n’en voulais pas.
  \end{verse}
  \begin{verse}
    Les poumons urbains gonflaient mal,\\
    les badauds arboraient des dents de porcelaine,\\
    crever dix-mille fois que je l’aime\\
    (en bon lourdaud de malin mâle).
  \end{verse}
  \begin{center}
    (Relire le tout avec)
  \end{center}
  \begin{verse}
    le ventricule qui galope,\\
    les conventions de sa gouverne;\\
    l’écriture est une salope\\
    pis que l’hydre de Lerne.
  \end{verse}
\PoemTitle{Holmes (6 Janvier 2012)}
  \begin{verse}
    Ârt contempôrain,\\
    plein de ô et de bas\\
    affriolants, cannettes humaines empilées.
  \end{verse}
  \begin{verse}
    Lady NénuPhare que déserte le silence\\
    lectures et bras de fumée tendus\\
    -- le doux pléonasme d’un cliché.
  \end{verse}
\PoemTitle{Une page et demi (6 Janvier 2012)}
  \begin{verse}
    Les hommes sont des papillons migrateurs\\
    et les femmes des gobeuses encheffées,\\
    l’amour passe le temps de l’aviateur\\
    et les passagers n’en sont pas moins inquiets.
  \end{verse}
  \begin{verse}
    Nous vivons dans de grandes carcasses en fer,\\
    et nous y sommes si serrés\\
    qu’on y croit sentir la chaleur;
  \end{verse}
  \begin{verse}
    \textsc{Donnez votre place}, importun\\
    -- je veux étouffer par ici\\
    -- car je suis un papillon à gober.
  \end{verse}
\PoemTitle{Cycle (14 Janvier 2012)}
  \begin{verse}
    Je regarde en arrière\\
    et ne vois que poèmes amassés\\
    -- plus de mémoire à redresser,\\
    plus de statues à honorer;
  \end{verse}
  \begin{verse}
    J’ai des souvenirs de toi comme des réalités myopes:\\
    plus de lumières que de clarté,\\
    réveilles-toi,
  \end{verse}
  \begin{verse}
    est-ce qu’il ne faut pas aplanir\\
    les détritus en vagues magnétiques décolorés,\\
    réveilles-toi,
  \end{verse}
  \begin{verse}
    J’ai posé les yeux sur l’Ailleurs\\
    et n’ai vu que tes flouteries jolies\\
    -- plus que le présent à honorer\\
    le bonheur déjà dressé.
  \end{verse}
\PoemTitle{Locked :) (6 Janvier 2012)}
  \begin{quotation}
    Tu les a eu, tes béquilles, mais il te manquait les jambes.
  \end{quotation}
  \begin{verse}
    Retour au cœur primaire des amygdales empapaouatées,\\
    elle est sitôt ici bas qu’il n’y a plus rien sur terre,\\
    hormi le oui qui \textsc{gueule} et le non matraquant.\\
  \end{verse}
  \begin{verse}
    Territoires insolites aux couleurs bleues du ciel,\\
    frasques de tête de pipe à fiel\\
    et la Lune qui monte verte et la colline.
  \end{verse}
  \begin{center}
    \textsc{Menteur, va.}
  \end{center}
  \begin{verse}
    Il t’a fallu vingt secondes pour construire,\\
    il ne t’en faudra pas autant pour abandonner.
  \end{verse}
\PoemTitle{La folie guide (24 Janvier 2012)}
  \begin{verse}
    Mais si je tombe encoR\\
    Avant cela je veux m’éparpilleR\\
    Rire encore et encor pleurer --\\
    Impuni comme jamaiS\\
    Et le corps gros de larmeS
  \end{verse}
  \begin{verse}
    Pas dans le silence\\
    ricochant je ne veux pas durer ainsi --\\
    envie d’un eternel bref\\
    Ô Présent, tu as une belle selle…
  \end{verse}
  \begin{verse}
    C’est mort, et pourtant ça vit encore\\
    -- avec toutes ces épines dans la chair\\
    ça résiste, ne défaille pas -- allons,
  \end{verse}
  \begin{center}
    \textsc{Un peu de tenue}\\
    \textsc{pour les gens bien}
  \end{center}
  \begin{verse}
    (ce qui m’exclue d’office)
  \end{verse}
  \begin{verse}
    mourrez bien au chaud,\\
    lits de terre recouverts un peu plus tôt\\
    -- j’arracherais tes dents et le reste\\
    -- jusqu’à ta mâchoire jusqu’au caveau.
  \end{verse}
