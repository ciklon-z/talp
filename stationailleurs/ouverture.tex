%	 *** Licence ***

% Cette œuvre est diffusée sous les termes de la license Creative Commons
% «~CC BY-NC-SA 3.0~», ce qui signifie que :

% Vous êtes libres :

%   * de reproduire, distribuer et communiquer cette création au public ;
%   * de modifier cette création.

% Selon les conditions suivantes :
    	
%  * Paternité - Vous devez citer le nom de l'auteur original de la manière indiquée par l'auteur de l'œuvre ou le
%	         titulaire des droits qui vous confère cette autorisation (mais pas d'une manière qui suggérerait
%	         qu'ils vous soutiennent ou approuvent votre utilisation de l'œuvre).
%  * Pas d’utilisation commerciale - Vous n'avez pas le droit d'utiliser cette œuvre à des fins commerciales. 
%  * Partage des conditions initiales à l'identique - Si vous transformez ou modifiez cette œuvre pour en créer une nouvelle,
% 						      vous devez la distribuer selon les termes du même contrat ou avec une
%					              licence similaire ou compatible.

% Comprenant bien que :

%  * Renoncement - N'importe quelle condition ci-dessus peut être retirée si vous avez l'autorisation du détenteur des droits.
%  * Domaine public - Là où l'œuvre ou un quelconque de ses éléments est dans le domaine public selon le droit applicable, ce statut
%		      n'est en aucune façon affecté par le contrat.
%  * Autres droits - d'aucune façon ne sont affectés par le contrat les droits suivants :
%    - Vos droits de distribution honnête ou d’usage honnête ou autres exceptions et limitations au droit d’auteur applicables;
%    - Les droits moraux de l'auteur;
%    - Les droits qu'autrui peut avoir soit sur l'œuvre elle-même soit sur la façon dont elle est utilisée, comme la publicité
%      ou les droits à la préservation de la vie privée.

% === Note ===

% Ceci est le résumé explicatif du Code Juridique; La version intégrale du contrat est consultable ici:
% <http://creativecommons.org/licenses/by-nc-sa/3.0/legalcode>.

\PoemTitle{La lyre contre la corne (13 Mars 2012)}
  \begin{verse}
    Je te donne ma parole;\\
    redonnes la moi;\\
    es-tu plus pauvre?\\
    Certainement pas.
  \end{verse}
  \begin{verse}
    Pour vivre, il faut s’encorner d’abondance.\\
    N’est pas inutile ce qui plane dans l’air\\
    bramant ses syntagmes le cul par terre\\
    et pas plus l’élancement -- {\Large soudain}\\
    de la pensée, mieux, des corps vers l’asemblable.
  \end{verse}
  \begin{verse}
    D’ailleurs, nous sommes d’Ailleurs\\
    -- c’est le plus semblable et le plus proche arrêt.
  \end{verse}
  \begin{verse}
    Mes amours sont ma posture\\
    -- ma station d’Ailleurs:\\
    un de plus dans les parages?
  \end{verse}
  \begin{verse}
    et voici l’arrêt du \textsc{verbe}.
  \end{verse}
