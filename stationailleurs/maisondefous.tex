%	 *** Licence ***

% Cette œuvre est diffusée sous les termes de la license Creative Commons
% «~CC BY-NC-SA 3.0~», ce qui signifie que :

% Vous êtes libres :

%   * de reproduire, distribuer et communiquer cette création au public ;
%   * de modifier cette création.

% Selon les conditions suivantes :
    	
%  * Paternité - Vous devez citer le nom de l'auteur original de la manière indiquée par l'auteur de l'œuvre ou le
%	         titulaire des droits qui vous confère cette autorisation (mais pas d'une manière qui suggérerait
%	         qu'ils vous soutiennent ou approuvent votre utilisation de l'œuvre).
%  * Pas d’utilisation commerciale - Vous n'avez pas le droit d'utiliser cette œuvre à des fins commerciales. 
%  * Partage des conditions initiales à l'identique - Si vous transformez ou modifiez cette œuvre pour en créer une nouvelle,
% 						      vous devez la distribuer selon les termes du même contrat ou avec une
%					              licence similaire ou compatible.

% Comprenant bien que :

%  * Renoncement - N'importe quelle condition ci-dessus peut être retirée si vous avez l'autorisation du détenteur des droits.
%  * Domaine public - Là où l'œuvre ou un quelconque de ses éléments est dans le domaine public selon le droit applicable, ce statut
%		      n'est en aucune façon affecté par le contrat.
%  * Autres droits - d'aucune façon ne sont affectés par le contrat les droits suivants :
%    - Vos droits de distribution honnête ou d’usage honnête ou autres exceptions et limitations au droit d’auteur applicables;
%    - Les droits moraux de l'auteur;
%    - Les droits qu'autrui peut avoir soit sur l'œuvre elle-même soit sur la façon dont elle est utilisée, comme la publicité
%      ou les droits à la préservation de la vie privée.

% === Note ===

% Ceci est le résumé explicatif du Code Juridique; La version intégrale du contrat est consultable ici:
% <http://creativecommons.org/licenses/by-nc-sa/3.0/legalcode>.

\PoemTitle{Des moustiques sous la Lune (27 Mars 2012)}
  La tête  disjointe et  les béquilles pour  le reste je  pète --  les mots,
  nauséabonde et créatrice -- ex-plosion. \hulu{La césure c’est pour les
  chiens.}

  Donnez  moi  votre  corps,  le  reste   je  m’en  fous,  c’est  mon  plat
  de  résistance  avant  l’ordure;  que j’examine  un  peu  cette  colonne
  vertébrale;  \hulu{il  faut être  un  peu  dentiste} dans  l’âme  pour
  écrire  : j’ai  envie, ici,  là, maintenant  c’est à  dire jamais  (le
  sublime  du récit!)  de vous  arracher  la colonne  vertébrale d’un  coup
  d’un seul  : colonne, colonne, colonne  vertébrale, rien que des  mots que
  tout cela. Vos muscles,  votre chair et votre cervelle et  vos dents, je puis
  en faire le monde et vous gaver  comme des oies de mes sottises; qu’importe
  après tout, si les moustacharbres ne sont pas des miennes?

  Badaboum crac, fiiiyou -- le vampire est paré à décoller, le taon prêt à
  voler  --  \hulu{un peu  d’élecrochocs  pour  la peine?}  L’essaim  va
  bourdonner  ailleurs de  mots et  de maux  à lire,  trouver des  lecteurs à
  agacer --  \hulu{On comprend rien!}  -- mais quoi?  Ne sommes nous  pas des
  moustiques  sous  la  Lune  qui  bourdonnent de  joie  sous  les  tuiles?

  Et pourquoi devrions-nous  continuer -- de faire sens? Au  diable, au diable!
  L’allitération, le cyclone, ou le néant!

\PoemTitle{Bourdonnent (13 Mars 2012)}
  Je marchais le long  du Champdé quand des tuiles de la  maison la plus moche
  et proche,  une tête me  tomba dessus;  lourde et poussiéreuse,  sèche par
  endroits,  gluante et  purulante en  d’autres, une  horrible expression  la
  scarifiait.

  Un homme arriva  et proposa de me  la troquer pour dix sous;  «~Avec ça, tu
  pourras chanter au  troquet!~» dit-il, emportant le ballon  et poussant plus
  loin son hernie héraclétique.

  C’était le  soir, les angelots  étaient de rire  morts sur la  façade du
  tertre; un peu de vent ramena à moi la pestilence, puis plus rien.

\PoemTitle{Deux joies (13 Mars 2012)}
  J’ai  tranché la  tête à  Hérédia, c’est  mon trophée  à moi.  Non
  qu’ouvrir ses oreilles aux ambages à étages de l’art sculptural et sans
  doute à la baiserie grecque au  profil ravagé ne m’escagasse les poils de
  narine.

  Mais je veux engloutir la fonction dramatique dans le verbe inutile et passer
  pour un baveur de mots -- c’est avec mon auteur, un point commun.

\PoemTitle{Sous les tuiles (16 Mars 2012)}
  Folie  -- c’est  le faux  lit où  l’on se  love entier  pour dormir  une
  éternité? Faux!  C’est la lie  du faussé  qui s’enfuit; ainsi  fait sa
  première syllabe, couleur de lys -- dénaturées vous êtes la musique qui a
  fa-illi.

  Folie,  amie d’Amour  qu’elle guide  mais on  ne sait  guère de  quoi la
  dernière  est le  contraire --  Mour,  sans doute  est-ce la  haine ou  quoi
  encore, la folie de la folie, je veux dire la raison? Faux!

  Folie qui me  nourrit et dont on me  taxe -- voici par là  les honoraires du
  poète pourriture merde chiure, insérer ici un autre terme de salissure pour
  me couper le souffle ou non, cesser cette fois sept fois de plus la politesse
  -- cette folie!

\PoemTitle{Perzonale Branline (3 Avril 2012)}
  \begin{verse}
    «~Tu seras beau -- je l’ordonne de ma bouche de fer\\
    le diktat dans la gueule qui gueule\\
    en vomissant du plastique.\\
  \end{verse}
  \begin{verse}
    - Ta voix, j’y foutrais du plastic\\
    et b{\large ou}{\Large m}, plus de vaches à meugler\\
    dans les près goudronnés.
  \end{verse}
  \begin{verse}
    - Tu seras fort, si l’argent est là\\
    tu me donneras ton sang\\
    et des emprûnts si tu n’en a pas:\\
    le cœur endetté pardonne mieux.
  \end{verse}
  \begin{verse}
    - De l’air, où est ma respiration\\
    encre de poumons et encre de la baise?
  \end{verse}
  \begin{verse}
    Point de tout cela sans déboucheur d’évier\\
    prends l’Alka Setzer plus soluble que l’air\\
    -- au mieux, vends-toi bien\\
    jusqu’aux os\\
    ~~~~à exposer aux balcons\\
    boulimie de mots, moi, mais quoi? Rien? Non, prends femme, prends chair,
    dépenses -- toi, seulement alors -- tu seras.
  \end{verse}

\PoemTitle{Feuille double, autre Moi (3 Avril 2012)}
  Feuille Seyiès, je ne sais si tu m’oblige à dire oui à la vie ou bien si
  c’est ma  cervelle qui me  dispense cet  éléctrochoc quotidien; et  si la
  poésie n’est  qu’électricité dans les  neurones en folie de  mes amis,
  est-ce à dire que la beauté n’est que -- radioactivité ?

  Lignes  bleues, verdâtres,  rouge, \litcit{lignes  de conduite,  étiquette,
  netiquette  de vivrensemble  au sein  d’un  projet pour  restorer le  tissu
  social} dit-on aussi, mais c’est de l’ordure verbale aussi droite qu’un
  ballon pris  dans la  figure chez  les fouteux ou  ailleurs, courbe  comme un
  mensonge qui charrie ses déchets vers la ligne verte qui convient.

  Mots à ne pas dire dans cette dissertation de cancres semi-morts : Moi, toi,
  révolution.  Sauf parfois,  quand  l’aérolithe mortifère  revient --~ ce
  boomerang! --~:~ c’est à moi, c’est  à toi, tu es à  moi, j’ai ceci,
  j’ai  cela et  c’est une  révolution; rien  qu’à l’écrire,  j’ai
  déjà en tête la  débile voix dressée aigu\"e, montée sur  le I comme au
  solutré -- pour ne rien dire.

  Il  aurait fallu  empoisonner  deux  fois Socrate  :  lui  et son  chienchien
  propagandiste de la résignation fébrile.

\PoemTitle{Itération d’un triste (3 Avril 2012)}
  Calendrier;  le journées  sont longues  mais on  les raye  proprement d’un
  trait noir comme  si elles étaient mortes  avec nous, et elles  le sont sans
  doute dans nos intestins et nos rides, mais guère ailleurs.

  Avec  lui, le  décompte des  mois, des  jours avant  tel petit  événement,
  l’épiphanie  ou  l’épiphénomène  se  fait tout  seul,  avec  paresse,
  tranquillité.  \hulu{L’angoisse  c’est  pour plus  tard},  quand  demain
  voudra toujours dire \litcit{Vas-tu mourir ce jour-ci?}. On n’y pense guère
  plus jusqu’à la  nouvelle itération des saisons, des jours  de la nuit --
  cycle lunaire, cycle  scolaire, cycle solaire, jours de vacances  \& jours de
  déprime.

  Parfois, il y  a 29 jours et déjà  4 ans de passés, 4 ans,  que cela passe
  vite quatre ans, comme  devant une vitrine de fripes si  l’on est un homme,
  il n’y  a même pas  de souvenirs accrocheurs  chez l’écolier :  fin des
  cours -- Juillet, fête des pétards --  le 14, ennui -- 11 Novembre, joie --
  un dix-huit en mathématiques extraterrestre, peine -- une fille qui part?
