%	 *** Licence ***

% Cette œuvre est diffusée sous les termes de la license Creative Commons
% «~CC BY-NC-SA 3.0~», ce qui signifie que :

% Vous êtes libres :

%   * de reproduire, distribuer et communiquer cette création au public ;
%   * de modifier cette création.

% Selon les conditions suivantes :
    	
%  * Paternité - Vous devez citer le nom de l'auteur original de la manière indiquée par l'auteur de l'œuvre ou le
%	         titulaire des droits qui vous confère cette autorisation (mais pas d'une manière qui suggérerait
%	         qu'ils vous soutiennent ou approuvent votre utilisation de l'œuvre).
%  * Pas d’utilisation commerciale - Vous n'avez pas le droit d'utiliser cette œuvre à des fins commerciales. 
%  * Partage des conditions initiales à l'identique - Si vous transformez ou modifiez cette œuvre pour en créer une nouvelle,
% 						      vous devez la distribuer selon les termes du même contrat ou avec une
%					              licence similaire ou compatible.

% Comprenant bien que :

%  * Renoncement - N'importe quelle condition ci-dessus peut être retirée si vous avez l'autorisation du détenteur des droits.
%  * Domaine public - Là où l'œuvre ou un quelconque de ses éléments est dans le domaine public selon le droit applicable, ce statut
%		      n'est en aucune façon affecté par le contrat.
%  * Autres droits - d'aucune façon ne sont affectés par le contrat les droits suivants :
%    - Vos droits de distribution honnête ou d’usage honnête ou autres exceptions et limitations au droit d’auteur applicables;
%    - Les droits moraux de l'auteur;
%    - Les droits qu'autrui peut avoir soit sur l'œuvre elle-même soit sur la façon dont elle est utilisée, comme la publicité
%      ou les droits à la préservation de la vie privée.

% === Note ===

% Ceci est le résumé explicatif du Code Juridique; La version intégrale du contrat est consultable ici:
% <http://creativecommons.org/licenses/by-nc-sa/3.0/legalcode>.

\PoemTitle{Parole de connaisseur (13 Mars 2012)}
  \begin{verse}
    Mentir est le propre de l’homme\\
    c’est pourquoi nous rions beaucoup\\
    de notre propre sérieux,\\
    notre cohérence dans l’absurde.
  \end{verse}
\PoemTitle{Marie-Persephone (19 Mars 2012)}
  \begin{verse}
    Qui aime les arbres a tout compris de l’amour\\
    -- que l’arbre tombe? Un autre viendra\\
    -- mais ce ne sera jamais le même;\\
    plutôt le meme, plutôt l’asemblable;\\
    on ne jouit jamais -- des mêmes racines.
  \end{verse}
\PoemTitle{Jacob sonne (13 Mars 2012)}
  \begin{verse}
    Signifiant -- je me donne à toi\\
    comme un gosse et ses jouets,\\
    en casseur de cochons tir--
  \end{verse}
  \begin{center}
    \hulu{Ma cassette!}
  \end{center}

\PoemTitle{Quatrevingt-treize (13 Mars 2012)}
  \begin{verse}
    Hugo, classique \emph{imitateur} de l’accent des chèvres,\\
    le verbe ample et double dans l’aberration\\
    -- fait jaillir des beautés catatoniques.
  \end{verse}

\PoemTitle{Vox Magistri (13 Mars 2012)}
  \begin{verse}
    Ces pandores si intéressées de nous enseigner à vivre\\
    oublient qu’elles nous font mourir d’ennui\\
    -- le vivre est un avant-goût de la décrépitude des corps.
  \end{verse}
\PoemTitle{Uvée (13 Mars 2012)}
  \begin{verse}
    Ces gens qui nous tiennent en prison --\\
    la belle responsabilité que voilà!\\
    Ils ne font pas tant de manières pour nous y laisser étouffer.
  \end{verse}
