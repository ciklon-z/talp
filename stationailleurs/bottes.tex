%	 *** Licence ***

% Cette œuvre est diffusée sous les termes de la license Creative Commons
% «~CC BY-NC-SA 3.0~», ce qui signifie que :

% Vous êtes libres :

%   * de reproduire, distribuer et communiquer cette création au public ;
%   * de modifier cette création.

% Selon les conditions suivantes :
    	
%  * Paternité - Vous devez citer le nom de l'auteur original de la manière indiquée par l'auteur de l'œuvre ou le
%	         titulaire des droits qui vous confère cette autorisation (mais pas d'une manière qui suggérerait
%	         qu'ils vous soutiennent ou approuvent votre utilisation de l'œuvre).
%  * Pas d’utilisation commerciale - Vous n'avez pas le droit d'utiliser cette œuvre à des fins commerciales. 
%  * Partage des conditions initiales à l'identique - Si vous transformez ou modifiez cette œuvre pour en créer une nouvelle,
% 						      vous devez la distribuer selon les termes du même contrat ou avec une
%					              licence similaire ou compatible.

% Comprenant bien que :

%  * Renoncement - N'importe quelle condition ci-dessus peut être retirée si vous avez l'autorisation du détenteur des droits.
%  * Domaine public - Là où l'œuvre ou un quelconque de ses éléments est dans le domaine public selon le droit applicable, ce statut
%		      n'est en aucune façon affecté par le contrat.
%  * Autres droits - d'aucune façon ne sont affectés par le contrat les droits suivants :
%    - Vos droits de distribution honnête ou d’usage honnête ou autres exceptions et limitations au droit d’auteur applicables;
%    - Les droits moraux de l'auteur;
%    - Les droits qu'autrui peut avoir soit sur l'œuvre elle-même soit sur la façon dont elle est utilisée, comme la publicité
%      ou les droits à la préservation de la vie privée.

% === Note ===

% Ceci est le résumé explicatif du Code Juridique; La version intégrale du contrat est consultable ici:
% <http://creativecommons.org/licenses/by-nc-sa/3.0/legalcode>.

\PoemTitle{Cycle du misanthrope (28 Février 2012)}
  \begin{verse}
    Pour se connaitre soi même\\
    il faut d’abord cesser de se nier\\
    pour qu’un jour\\
    nous nous aimions avec envergure.
  \end{verse}
  \begin{verse}
    J’idolâtre les cathédrales\\
    d’un goût nécrophage élevé,\\
    entre la joie de les voir\\
    et l’insatiable plaisir de les saper.
  \end{verse}
  \begin{verse}
    Ô toi mon amour,\\
    limbes étern-ailes,\\
    Ô toi Septembre aux feuilles qui coulent,\\
    pluie qui rassure et déchire;
  \end{verse}
  \begin{verse}
    ne suis-je là que par causalité?\\
    Ou bien seulement suis-je de ce monde\\
    ou de l’autre\\
    simplement pour t’avoir aimé?
  \end{verse}
  \begin{verse}
    Et c’est cela qui perpétué\\
    donne sans vergogne le tout oxygéné\\
    soldé comme cette bête à abattre -- je veux dire, l’humanité\\
    comme toi, comme moi, comme elle,\\
    nous, les monstres de papier.
  \end{verse}
  \begin{verse}
    Silence, oui, silence\\
    le partage est le drapeau le plus beau\\
    -- le tricolore est un faux\\
    et demande-t-on leurs papiers aux mots?
  \end{verse}
  \begin{verse}
    Silence pour les cloportes\\
    -- ils veulent dire, silence,\\
    pour nous qui assassinons le peuple.
  \end{verse}

\PoemTitle{Merde à la flexibilité (3 Mars 2012)}
  \begin{verse}
    La ville est belle\\
    quant y brûlent des millions de cadavres;\\
    dit-il non aux lumières mourantes et si jeunes\\
    du matin au soir geignant l’ennui?
  \end{verse}
  \begin{verse}
    Ou bien n’a-t-il plus rien dans le crâne\\
    pour dire oui ainsi à l’urbaine toxine\\
    confortée des vers-amphétamines?
  \end{verse}

\PoemTitle{MORE DEUT (3 Avril 2012)}
  \begin{verse}
    La ville a des cils\\
    et des immeubles -- murailles cadratines;\\
    \litcit{Graor Broum Broum} fait le monstre\\
    point d’exclamation géant\\
    flottant, insultant\\
    ~~~~\hulu{Quoi?}
  \end{verse}
  \begin{verse}
    L’homme, la vie la joie,\\
    en pollen-gazoil\\
    -- rendu de froid à poelle.
  \end{verse}

\PoemTitle{Il faut le mériter (3 Mars 2012)}
  \begin{verse}
    La civilisation se porte bien\\
    elle ne fait que donner des coups très durs,\\
    le langage de la politique reste l’ordure;
  \end{verse}
  \begin{center}
    \hulu{Un peu de respect pour la Réépublique}

    \hulu{(Pour ma prostituée)}

    (dixit le banquier)
  \end{center}
\PoemTitle{Affamons les végétariens (28 Mars 2012)}
  \begin{verse}
    Demain je vociférerai\\
    toute ma soif de jours\\
    à m’en éclater la gueule\\
    sur les tableaux noirs\\
    qui m'’ont arraché toute la peau.
  \end{verse}
  \begin{verse}
    Leur voix sèche comme des coffrages\\
    pour contenir l’'atomisation\\
    -- j’ai la matière grise sous echaufaudages;
  \end{verse}
  \begin{verse}
    Non, l'homme n'est qu’'un cheval -- pas beau,\\
    ce canasson -- ceux qui le moquent sont des ânes\\
    -- voici mon adresse à toute l’éternité
  \end{verse}
  \begin{center}
    \hulu{HiHan}
  \end{center}
  \begin{verse}
    Hihan à l’hémistiche je ne sais pas\\
    les mots sont la colique des gorges:\\
    avant que cela -- finisse sur l’écarrelage\\
    empuantissez un peu! C'est ma tournée.
  \end{verse}
  \begin{verse}
    De joie je me foutrais\\
    la Loire à dos,\\
    elle et sa peau\\
    de vasemingue managère.
  \end{verse}
\PoemTitle{API (20 Mars 2012)}
  \begin{verse}
    Repas du corps de guerre de littérature:\\
    dévorer le silence -- décapiter Saussure\\
    souffler le cor de guerre dans les landes futures\\
    jusqu’à plus de souffle (et la pourriture).
  \end{verse}
  \begin{verse}
    Le soleil qui revient et c’est Sparte qui crève\\
    la gueule d’ahuri des bouchers de rêves\\
    donner l’eau les mots et le rythme ternaire\\
    pour divaguer la poésie sans en avoir l’air.
  \end{verse}
