%	 *** Licence ***

% Cette œuvre est diffusée sous les termes de la license Creative Commons
% «~CC BY-NC-SA 3.0~», ce qui signifie que :

% Vous êtes libres :

%   * de reproduire, distribuer et communiquer cette création au public ;
%   * de modifier cette création.

% Selon les conditions suivantes :
    	
%  * Paternité - Vous devez citer le nom de l'auteur original de la manière indiquée par l'auteur de l'œuvre ou le
%	         titulaire des droits qui vous confère cette autorisation (mais pas d'une manière qui suggérerait
%	         qu'ils vous soutiennent ou approuvent votre utilisation de l'œuvre).
%  * Pas d’utilisation commerciale - Vous n'avez pas le droit d'utiliser cette œuvre à des fins commerciales. 
%  * Partage des conditions initiales à l'identique - Si vous transformez ou modifiez cette œuvre pour en créer une nouvelle,
% 						      vous devez la distribuer selon les termes du même contrat ou avec une
%					              licence similaire ou compatible.

% Comprenant bien que :

%  * Renoncement - N'importe quelle condition ci-dessus peut être retirée si vous avez l'autorisation du détenteur des droits.
%  * Domaine public - Là où l'œuvre ou un quelconque de ses éléments est dans le domaine public selon le droit applicable, ce statut
%		      n'est en aucune façon affecté par le contrat.
%  * Autres droits - d'aucune façon ne sont affectés par le contrat les droits suivants :
%    - Vos droits de distribution honnête ou d’usage honnête ou autres exceptions et limitations au droit d’auteur applicables;
%    - Les droits moraux de l'auteur;
%    - Les droits qu'autrui peut avoir soit sur l'œuvre elle-même soit sur la façon dont elle est utilisée, comme la publicité
%      ou les droits à la préservation de la vie privée.

% === Note ===

% Ceci est le résumé explicatif du Code Juridique; La version intégrale du contrat est consultable ici:
% <http://creativecommons.org/licenses/by-nc-sa/3.0/legalcode>.

\PoemTitle{404 (2 Mars 2012)}
  \begin{verse}
    Je ne fais de mal à personne, vraiment --\\
    je ne suis que Tout -- étonnement;\\
    mes détracteurs sont mes amis\\
    -- ils agitent les chaines, c’est permis.
  \end{verse}
  \begin{verse}
    Plus de lieu où aller, pourtant!\\
    Stratosphères au loyer trop cher\\
    \textsc{La Voie Lactée}\\
    ça peut être tentant.
  \end{verse}
  \begin{verse}
    Mes mondes sont bondes\\
    d’encre pour ma création\\
    bondés de vestiges immondes\\
    de mon imagination.
  \end{verse}
  \begin{verse}
    Cependant je dors sans être quelque chose,\\
    ne juge plus rien que le songe;\\
    \hulu{ta vie en est un},\\
    démiurge amateur.
  \end{verse}
\PoemTitle{Probité (19 Février 2012)}
  \begin{verse}
    Le jaloux a-t-il fait un monde en sept jours?\\
    Qu’importe, j’en compose bien douze tour à tour;\\
    quoi de moins ahurissant\\
    que de vivre dans ces cathédrales humaines,
  \end{verse}
  \begin{verse}
    ennuyeuses ou non c’est selon?
  \end{verse}
  \begin{verse}
    Mais de toutes les nations\\
    je refuse l’étendard -- ensanglanté toujours\\
    du mensonge et de la répression;
  \end{verse}
  \begin{verse}
    Mon chez-moi, c’est le grand Internet,\\
    à l’instar de mes amours verticaux\\
    et non de mes vers, il chante l’Octo’~!
  \end{verse}

\PoemTitle{La rhapsodie du nombril (13 Mars 2011)}
  \begin{verse}
    J’aime et connais mon prochain\\
    -- que ma poésie ne soit donc pas\\
    -- vraisemblable!
  \end{verse}
  \begin{verse}
    Ici et là, de bons fournisseurs\\
    en limites et \emph{taiseries} -- que ma poésie\\
    -- éclate!
  \end{verse}
  \begin{verse}
    Pas de corps là dedans? Mon œil!\\
    La culture se présenta\\
    ainsi: «~Je suis la rhapsodie du nombril~»…
  \end{verse}
\PoemTitle{Retour à l’envoyeur (5 Mars 2012)}
  \begin{verse}
    Les souvenirs sont de curieuses bâtisses\\
    fêlées teintes des ocres que l’on veut bien\\
    qui sans qu’on le comprend tout à fait\\
    tour à tour hennissent;
  \end{verse}
  \begin{verse}
    \emph{Hu!} Était-ce vrai tout cela?\\
    \emph{Hu!} Ais-je aimé \emph{ainsi}?\\
    \emph{Hu!} Étais-je bête à ce point là?
  \end{verse}
  \begin{verse}
    Lejeune
    \footnote{
      Théoricien de l’autobiographie. On notera l’ironie de son nom de
      famille.
    } est l’ennemi du fossilisé\\
    tout est clair, net et chenu\\
    percées de lumières dans nos têtes exig\"ues;
  \end{verse}
  \begin{verse}
    images qu’on ne saurait attraper,\\
    erreurs qu’on ne saurait abolir,\\
    fleuves où l’on ne saurait deux fois se tremper\\
    -- ce vieux sac d’os sans mots inversés\\
    souvenirs les.
  \end{verse}
\PoemTitle{Effrayants Espaces Infinis (23 \& 27 Mars 2012)}
  \begin{verse}
    Je suis plein des gens que je connais\\
    et pis, ’me prennent à la gorge\\
    «~vomis, vomis petit.\\
    Tu vivras bien sans remettre à plus tard.~»
  \end{verse}
  \begin{verse}
    «~D’outre-vers et pourtant si -- imitateur,\\
    outré d’être l’outre des crève-mots?\\
    Les rues sans conscience se moquent d’être pleines\\
    -- veux-tu être notre mégalomacropole?~»
  \end{verse}
  \begin{verse}
    Je suis plein des gens que je connais\\
    sans doute pour crier à la face des nuées\\
    la vermine solitude, les manques singuliers\\
    qui me font -- composer!
  \end{verse}

\PoemTitle{Le vertige n’est pas… (13 Mars 2012)}
  \begin{verse}
    Il me faudra encore beaucoup d’absences\\
    d’abstrincte\footnote{Cf. \noun{Jost Vincent}, «~Abstraction~»}\\
    afin que ma pensée ait la transparence\\
    de l’absinthe;
  \end{verse}
  \begin{verse}
    et que tout ce qui peut rugir, courir, toutes mes sensations d’éclatement,\\
    et que tout ce qui défend la raison comme une mère son fils -- abusivement,\\
    et que tout ce que me fait verdâtre, vide d’être plein de part et d’autre et de ne point vomir puisse
  \end{verse}
  \begin{center}
    \begin{tabular}{ccccccc}
      T &   & R &   & L &   & É\\
        & E &   & A &   & C &
    \end{tabular}{}
  \end{center}
  \begin{verse}
    Spectacle sans nom de la main qui donne
    de l’exubérance de l’autre et puis
    courir, courir toujours -- ailleurs;
  \end{verse}
  \begin{verse}
    demain c’était la veille où je suis ainsi\\
    et pis que du pied-verre-pilé je vis\\
    de veines bleues et vives
  \end{verse}
  \begin{center}
    \hulu{Il faudra bien que Ça te passe}
  \end{center}
  \begin{verse}
    -- dessus comme les autres?\\
    Je vous laisse à la corvée.
  \end{verse}
\PoemTitle{Pleurs d’une fille (27/28 Mars 2012)}
  \begin{verse}
    Gloire au sourcil dressé\\
    comme une arme et je sais
  \end{verse}
  \begin{verse}
    que c’est pire qu’un marteau\\
    pour m’aplatititir;
  \end{verse}
  \begin{verse}
    cette tristesse là qui gratuitement\\
    me renvoie sous terre\\
    creuser les mots qui ne disent ne pensent rien\\
    saleté foutraque de virus Empathite A.
  \end{verse}
\PoemTitle{moar awesomeness (8 Mars 2012)}
  \begin{verse}
    La toile tait Niflheim grise\\
    envahie de noir lierre\\
    \& le cadre pas tellement
  \end{verse}
  \begin{center}
    \textsc{Spatio-Temporel}
  \end{center}
  \begin{verse}
    mais déliriomaniacodéprimonirologique.
  \end{verse}
  \begin{verse}
    Des cliquetis, desserres l’étau\\
    la présence est cendrée au bec de miel\\
    -- sous les plumes courent des serres et tout et tout,
  \end{verse}
  \begin{verse}
    à quoi ressemblait Niflheim dans leur caboche?\\
    À vos bruits de béton, de vocable fantoche\\
    d’égorgeurs de pauvres classieux.
  \end{verse}
  \begin{verse}
    Renversement de bocaux\\
    voici que les vases parlent\footnote{Ils communiquent bien, non?}\\
    et profèrent le très saint nom
  \end{verse}
  \begin{center}
    \hulu{Aglouglou}
  \end{center}
  \begin{verse}
    Brulez le, brulez le -- l’infâme\\
    se gausse du monde\\
    il n’en chargera pas la courbure.
  \end{verse}
\PoemTitle{Discussion privée (13 Mars 2012)}
  \begin{verse}
    Champdé -- champ des morts\\
    éclatant de vie à t’en pestiférer la langue\\
    -- que d’arbres, que d’arbres en façade tombés
  \end{verse}
  \begin{verse}
    -- est la nuit couleur de chocolat,\\
    des pépites de sens\\
    amertume de la pourriture.
  \end{verse}
