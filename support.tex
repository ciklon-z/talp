\addcontentsline{toc}{part}{Soutien à l’auteur}

\section*{Soutien à l’auteur}

Le présent  ouvrage est  le fruit  d’un travail  et une  œuvre de
l’esprit.  En  tant qu’œuvre  de  l’esprit,  j’ai jugé  bon
pour  les droits  du  public  de renoncer à tous  mes  droits  sur cette  œuvre
en  l’élevant  au  domaine  public\footnote{Voir  la  licence  CC0
ci-dessus}.

Cependant -- et  cette invitation au soutien ne le  montre que trop bien! --  je ne suis
pas mort, contrairement à la quasi-totalité des auteurs des œuvres
sises dans le domaine public.

Aussi,  si vous  avez apprécié  cette œuvre,  si vous  désirez en
découvrir d’autres, un don serait appréciable.

\addcontentsline{toc}{section}{Plateformes de don}
\subsection*{Plateformes de don}

  \subsubsection*{Flattr}
    \begin{itemize}
      \item \href{http://flattr.com/profile/enadji}{Compte \emph{Flattr}}\footnote{Note à disposition des éditions papier : \texttt{http://flattr.com/profile/enadji}}.
    \end{itemize}
  \subsubsection*{~~In Libro Veritas}
    Mes œuvres, pour des raisons propres à leur rendu\footnote{Les lignes vides se transformaient en passage à la ligne, ce qui est fâcheux pour de la poésie…}
    ne sont plus disponible sur  In Libro Veritas. Cependant, ma page
    d’auteur est encore là et permet de procéder à des dons.
    \begin{itemize}
      \item \href{http://www.inlibroveritas.net/auteur11390.html}{Profil \emph{In Libro Veritas}}\footnote{Note à disposition des éditions papier : \texttt{http://www.inlibroveritas.net/auteur11390.html}}
    \end{itemize}

\addcontentsline{toc}{section}{Contact}
\subsection*{Contact}

  \begin{description}
    \item[~~Mail] \texttt{etnadji@eml.cc}
    \item[~~Blog] \href{http://tviblindi.legtux.org/blog/}{\texttt{http://tviblindi.legtux.org/blog/}}
    \item[~~Identi.ca] \texttt{@panoptes}
    \item[~~Github] \texttt{@tviblindi}
    \item[~~Twitter] \texttt{@tuxmetal}
  \end{description}
