	%*** Licence ***

%Cette œuvre est diffusée sous les termes de la license Creative Commons
%«~CC BY-NC-SA 3.0~», ce qui signifie que :

%Vous êtes libres :

  %* de reproduire, distribuer et communiquer cette création au public ;
  %* de modifier cette création.

%Selon les conditions suivantes :
    	
  %* Paternité - vous devez citer le nom de l'auteur original de la manière indiquée par l'auteur de l'œuvre ou le
		%titulaire des droits qui vous confère cette autorisation (mais pas d'une manière qui suggérerait
		%qu'ils vous soutiennent ou approuvent votre utilisation de l'œuvre).
  %* Pas d’utilisation commerciale - Vous n'avez pas le droit d'utiliser cette œuvre à des fins commerciales. 
  %* Partage des conditions initiales à l'identique - si vous transformez ou modifiez cette œuvre pour en créer une nouvelle,
  						     %vous devez la distribuer selon les termes du même contrat ou avec une
						     %licence similaire ou compatible.

%{Comprenant bien que :

  %* Renoncement} - N'importe quelle condition ci-dessus peut être retirée si vous avez l'autorisation du détenteur des droits.
  %* Domaine public - Là où l'œuvre ou un quelconque de ses éléments est dans le domaine public selon le droit applicable, ce statut
		     %n'est en aucune façon affecté par le contrat.
  %* Autres droits - d'aucune façon ne sont affectés par le contrat les droits suivants :
    %- Vos droits de distribution honnête ou d’usage honnête ou autres exceptions et limitations au droit d’auteur applicables;
      %Les droits moraux de l'auteur;
    %- Droits qu'autrui peut avoir soit sur l'œuvre elle-même soit sur la façon dont elle est utilisée, comme la publicité
      %ou les droits à la préservation de la vie privée.

%=== Note ===

%Ceci est le résumé explicatif du Code Juridique; La version intégrale du contrat est consultable ici:
%<http://creativecommons.org/licenses/by-nc-sa/3.0/legalcode>.

\PoemTitle{Power off (21 Septembre 2011)}
  \begin{flushright}
    \textit{
    Oh tell me where your freedom lies

    The streets are fields that never die
    }

    \textsc{The Doors}, \emph{The Crystal Ship}
  \end{flushright}
  \end{quotation}
  \begin{verse}
    Villes vous avez de grises nuits et de grises mines!\\
    Taudis qui se cachent dans les passants,\\
    amours contrariés, pigeons doublement intoxiqués,\\
    rats lourdauds aux dépends des chômeurs braves!
  \end{verse}
  \begin{verse}
    Qu’on les éteigne, ces lampadaires!\\
    Qu’on les étreigne, ces manque-pas-d’air!\\
    Une vraie nuit, plus une parodie de jour,\\
    le silence, le vent, et puis…tout le reste en découle.
    \footnote{Remarqué lors de la recopie: à rapprocher de la chanson de Renaud, 
	      \emph{Fatigué}}
  \end{verse}
\PoemTitle{Le Privilège des ignorants (26 Septembre 2011)}
  \begin{verse}
    Château aux salles fondues\\
    et limites repoussées à l’Imaginaire,\\
    livre curieux quoi que pécuniaire,\\
    combien grâce à toi ont lu?
  \end{verse}
  \begin{verse}
    Peut-être n’es-tu point parfait\\
    mais tu n’es pas comme ces nouveaux romans\\
    aux charmes si délicatement contrefaits\\
    qu’exècrent avec raison les enfants!
  \end{verse}
  \begin{verse}
    Survivras-tu au temps et à tes détracteurs,\\
    jolie abîme à paraboles,\\
    ou seras-tu ignorée des âmes des lecteurs\\
    et jetée aux littéraires nécropoles?
  \end{verse}
  \begin{verse}
    Ou bien renonçant à te grandir\\
    ainsi que les fabuleux contes,\\
    demeureras-tu dans les langues\\
    de celles que tes sorciers ont enflammé?
  \end{verse}
\PoemTitle{Et caetera… (26 Septembre 2011)}
  \begin{verse}
    Mais qu’ais-je à faire des nations?	\\
    Qu’elles meurent toutes -- et la mienne la première\\
    -- que les hommes fraternisent\\
    et ruinent marchands et bouchers.
  \end{verse}
  \begin{verse}
    Oh le beau langage des années scandales:\\
    «~Autonome~» -- un patron et un loyer;\\
    «~Devenir soi même~» -- aller se faire étriper;\\
    et vous les porcs, vous jugez encore de notre morale?
  \end{verse}
  \begin{verse}
    Vivre est le plus grand, le plus beau héroisme\\
    -- la desertion un devoir\\
    et la guerre une bêtise. 
  \end{verse}
  \begin{verse}
    Le sang -- hélas -- coule encore de votre pressoir\\
    au piètre nom de capitalisme\\
    et la paix vêtue de civilisation\\
    dans la course au pétrole s’enlise.
  \end{verse}
  \begin{verse}
    Mais voici! Nous ne regardons plus la télévision;\\
    par delà le ’net un souffle se propage\\
    et pour peu que nous le suivions\\
    il mettra à bas nos inutiles aéropages.
  \end{verse}
  \begin{verse}
    Je mange français,\\
    mon stylo est taiwanais,\\
    mon éditeur de texte américain\\
    -- toute frontière est morte, je ne suis qu’humain.
  \end{verse}
  \begin{verse}
    Amateurs de sang, formez des clubs\\
    et étripez-vous en groupe;\\
    quand à moi, je célébrerais la vie (et les croupes)\\
    et les rythmiques \emph{invasions} de ma t…
  \end{verse}
\PoemTitle{Par delà Aristote (26 Septembre 2011)}
  \begin{verse}
    Myopes, vos âmes dans le brouillard,\\
    écoliers et idiots, les avez-vous perdus de votre fait?
  \end{verse}
  \begin{verse}
    Admirez la puissance de la Broyeuse.
  \end{verse}
  \begin{verse}
    Ton amitié comme une tache\\
    -- sur mon CV,\\
    ton sourire me détache\\
    -- de mes administrés.
  \end{verse}
  \begin{verse}
    Admirez la puissance de la Broyeuse.
  \end{verse}
  \begin{verse}
    Animal qui produit la Nation,\\
    ainsi qu’une mer, ta gueule est admirable
    \footnote{\textsc{Lautréamont},\emph{Les Chants de Maldoror},«~Vieil océan~,
     etc…»};\\
    mère des mutilations et de tous les charniers,\\
    l’École a modelé nos crânes de résignés.
  \end{verse}
  \begin{verse}
    Admirez la puissance de la Broyeuse.
  \end{verse}
  \begin{verse}
    Nous la génération dont les diplômes jamais ne nous aiderons\\
    et qui poussons chaotiquement ainsi que le liseron\\
    -- retrouvons la terre, soyons peuple et agricoles,\\
    destituons les patrons de l’École.
  \end{verse}
\PoemTitle{Charon de pacotille (10 Octobre 2011)}
  \begin{verse}
    Allons entre -- voilà\\
    Je te donnes un avenir! Ne souffle pas.\\
    Ces chiffres me disent tout de toi,\\
    tes paroles, je n’en crois rien -- tais-toi.
  \end{verse}
  \begin{verse}
    Qu’as-tu à paraître confus,\\
    il n’est pas clair\\
    cet organigramme? Tout pourtant s’éclaire\\
    en lisant le programme.
  \end{verse}
  \begin{verse}
    Aimes-tu l’histoire?\\
    Tu enseigneras de la soupe éduquée!\\
    Comment ça «~Faut voir…~»?\\
    Je suis là pour t’orienter…
  \end{verse}
\PoemTitle{Humanus Ex Machina (28 Septembre 2011)}
  \begin{verse}
    Défilé de visages fermés;\\
    on vous reprocherait presque d’avoir des yeux:\\
    oh la vieille apostrophe abâtardie,\\
    Ô les joies de la consommation!
  \end{verse}
  \begin{verse}
    Soupe musicale et potage à dissoudre,\\
    un président mercenaire à absoudre,\\
    l’art comme célébrateur de la résignation;\\
    mais où se perdent nos affonts?
  \end{verse}
  \begin{verse}
    Véhicules de plastique\\
    -- vous serez notre avenir tragique;\\
    Conbien de polymères avalons-nous\\
    -- reniera-t-on le plaisir?
  \end{verse}
  \begin{verse}
    Mais toute joie est partie dans la fosse septique -- «~Trop peu cher!~»\\
    Regardez tout ce qu’ils faussent\\
    ces perses émissaires!
  \end{verse}
  \begin{verse}
    Et quand commenceront-ils leurs autodafés?\\
    Et quand cesseront-ils de nous transformer?\\
    Quand nous serons là à protester\\
    -- et à les jeter dans leurs propres bûchers.
  \end{verse}
