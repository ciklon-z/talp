	%*** Licence ***

%Cette œuvre est diffusée sous les termes de la license Creative Commons
%«~CC BY-NC-SA 3.0~», ce qui signifie que :

%Vous êtes libres :

  %* de reproduire, distribuer et communiquer cette création au public ;
  %* de modifier cette création.

%Selon les conditions suivantes :
    	
  %* Paternité - vous devez citer le nom de l'auteur original de la manière indiquée par l'auteur de l'œuvre ou le
		%titulaire des droits qui vous confère cette autorisation (mais pas d'une manière qui suggérerait
		%qu'ils vous soutiennent ou approuvent votre utilisation de l'œuvre).
  %* Pas d’utilisation commerciale - Vous n'avez pas le droit d'utiliser cette œuvre à des fins commerciales. 
  %* Partage des conditions initiales à l'identique - si vous transformez ou modifiez cette œuvre pour en créer une nouvelle,
  						     %vous devez la distribuer selon les termes du même contrat ou avec une
						     %licence similaire ou compatible.

%{Comprenant bien que :

  %* Renoncement} - N'importe quelle condition ci-dessus peut être retirée si vous avez l'autorisation du détenteur des droits.
  %* Domaine public - Là où l'œuvre ou un quelconque de ses éléments est dans le domaine public selon le droit applicable, ce statut
		     %n'est en aucune façon affecté par le contrat.
  %* Autres droits - d'aucune façon ne sont affectés par le contrat les droits suivants :
    %- Vos droits de distribution honnête ou d’usage honnête ou autres exceptions et limitations au droit d’auteur applicables;
      %Les droits moraux de l'auteur;
    %- Droits qu'autrui peut avoir soit sur l'œuvre elle-même soit sur la façon dont elle est utilisée, comme la publicité
      %ou les droits à la préservation de la vie privée.

%=== Note ===

%Ceci est le résumé explicatif du Code Juridique; La version intégrale du contrat est consultable ici:
%<http://creativecommons.org/licenses/by-nc-sa/3.0/legalcode>.

\PoemTitle{Longues langueurs ou Anti-gravité localisée  (23 Septembre 2011)}
  \begin{verse}
    Ô siècles sans courant,\\
    que vous fûtes ampoulés!\\
    Le silence n’effraie plus,\\
    on lui court après\\
    -- comme un chien!
  \end{verse}
  \begin{verse}
    Échelles intouchables\\
    de lumières alignées,\\
    Ô le long couloir,\\
    ô la belle couchée.
  \end{verse}
  \begin{verse}
    Au delà, il n’y a plus d’hommes\\
    mais pour le myope seulement des taches qui se confondent,\\
    ici et là, les minutes s’effondrent.\\
    À ma droite la sortie\\
    ~~~~dans l’air frais\\
    loin d’elle et de \emph{la} maladie.
  \end{verse}
  \begin{verse}
    Sourires d’acteurs sur les charitables panneaux,\\
    de vieux échos se perdent dans mes oreilles.
  \end{verse}
  \begin{verse}
    Parler seul ou rompre la glace,\\
    un toussotement et tout s’efface.
  \end{verse}
\PoemTitle{Les vents de passage chez nous Samedi soir (07 Octobre 2011)}
  \begin{verse}
    \textsc{Retour des choses agréables}\\
    -- un escalier de néons\\
    porte nos voix
  \end{verse}
  \begin{verse}
    Vampirisons encore si nous le pouvons,\\
    ces afflux sanguins sont bien ~~~~ solitaires\\
    -- la pompe fait un boucan d’enfer.
  \end{verse}
  \begin{verse}
    Libertad aux couloirs a ouvert les bondes,\\
    le vacarme populeux a fui comme une onde\\
    -- le silence nous paie des délices.
  \end{verse}
  \begin{verse}
    \textsc{Retour des démons pour le morphéophage:}\\
    plus rien n’est clair ni noir,\\
    l’avenir tout cuistre étalé et brumeux\\
    ouvrira ses bras à qui le veut.
  \end{verse}
\PoemTitle{La barrière des langues (26 Septembre 2011)}
  \begin{verse}
    Aériens et délicats\\
    ainsi sont les doigts d’albâtre\\
    de la Lady sacrée comme un reliquat.
  \end{verse}
  \begin{verse}
    Haut et bas, géométrie,\\
    vous m’ennuyez! Vive l’irrégularité\\
    fondatrice de ces lignes jolies!
  \end{verse}
  \begin{verse}
    Asseyons-nous sous l’interrogation
    \footnote{À titre d’éclairage, le premier titre de 
    ce poème était \emph{Intertextualité}}\\
    de la Lady puisqu’aliments\\
    et atomes nous sommes à la ville;
  \end{verse}
  \begin{verse}
    La rêverie a atteint son emphase la plus vile\\
    -- flottons subtils alors que la belle est là,\\
    et surtout n’ayons l’air de rien\\
    ~~~~-- fuyons de ce pas…
  \end{verse}
\PoemTitle{Engagement (11Octobre 2011)}
  \begin{verse}
    Rassie comme une riche\\
    petite ainsi qu’une nonne,\\
    elle organise la politique triche\\
    et les ouvertures mignonnes.
  \end{verse}
  \begin{verse}
    Le loup rompu et le météore\\
    dans le même bateau\\
    voguent sur les limbes de mon cerveau.
  \end{verse}
  \begin{verse}
    J’ai même un mauvais scaphandre\\
    -- mais une bonne place à l’ombre --,\\
    un bois fortement perdu,\\
    les doutes d’un cénobite…
  \end{verse}
\PoemTitle{Antienne (27 Septembre 2011)}
  \begin{verse}
    Ô la puanteur du genre humain,\\
    égouts fastueux (et pestiférés),\\
    bouche vomissant les cadavres et l’indignité;
  \end{verse}
  \begin{verse}
    Ô le gâchis du genre humain,\\
    qui se fait encore et encore si mal\\
    et le nihilisme exhalé des cathédrales!
  \end{verse}
  \begin{verse}
    Ô la cruauté de la belle nature,\\
    falaises et vagues grondant,\\
    collines effacées suintant le sang;
  \end{verse}
  \begin{verse}
    Ô la cruauté de la belle nature,\\
    dolmens usés, abandonnés\\
    -- et en dessous, la chair du mouton déchirée;
  \end{verse}
  \begin{center}
     {\huge …}
  \end{center}
  \begin{verse}
    Ô le prévisible génie de l’être humain!
  \end{verse}
  \begin{verse}
    Sens exsudé du papier,\\
    vitraux déformateurs de lumière,\\
    villes grandioses au fond du Finistère\\
    et le cri que Nietzsche a poussé!
  \end{verse}
  \begin{verse}
    Ô le prévisible génie de l’être humain!
  \end{verse}
  \begin{verse}
    La masse chantante et en grève,\\
    les corps des tyrans à jamais foulés,\\
    la glorieuse fin d’une honteuse trêve,\\
    la populace qui s’est soulevée!
  \end{verse}
\PoemTitle{Noir aux idées bleues (25 Octobre 2011)}
  \begin{verse}
    Simulons le rire puisque nous sommes englués,\\
    ouvrons nos portes et notre cynisme\\
    puisque les valves se ferment.
  \end{verse}
  \begin{verse}
    Noyés dans le bleu tacheté\\
    se perdent nos ambitions \& délires,\\
    plus rien encore.
  \end{verse}
\PoemTitle{Flap Flap Flap (Octobre 2011)}
  \begin{verse}
    La lionne avait des doigts en colliers\\
    aux extrémités pleines de sang,\\
    la grâce de l’araignée.
  \end{verse}
  \begin{verse}
    Tisse, tisse, tisse entre les choses qui n’ont plus de sens,\\
    ces pantins me déplaisent à outrance\\
    et la ville n’y a rien changé;
  \end{verse}
  \begin{verse}
    Plus de gazoil que d’oxygène,\\
    vaisseaux d’encre et de bitume\\
    à peine irrigués abdiquant pour la plume\\
    -- ou peut-être que je t’…
  \end{verse}
  \begin{verse}
    Partie dans un bruissement d’ailes\\
    la dealeuse experte de l’éther?\\
    Allons! Redevenons ce que nous fûmes\\
    et appliquons ce que nous voulions en faire.
  \end{verse}
\PoemTitle{Cheveux (9 Octobre 2011)}
  \begin{verse}
    Je ne trouve plus ma tête, où l’ais-je donc rangée?\\
    Ou serait-ce des kamikazes\\
    qui me l’ont écrabouillée?
  \end{verse}
  \begin{verse}
    J’en reprendrais une d’occase\\
    pour venir te chercher;\\
    regardes -- je viens t’honorer.
  \end{verse}
  \begin{verse}
    La situation est sérieuse, essaye l’aspirine,\\
    ma tête est lourde comme de la baramine,\\
    allons, approches-toi du drogué\\
    pour qu’il puisse t’aimer.
  \end{verse}
\PoemTitle{À l’université du Loch Ness (27 Septembre 2011)}
  \begin{verse}
    Un losange de nuages\\
    comme un diadème.
  \end{verse}
  \begin{verse}
    Ô îles vallonnées, eaux profondes et polluées!
  \end{verse}
  \begin{verse}
    Même les oiseaux volent en groupe,\\
    il n’y a que l’homme qui est idiot.
  \end{verse}
  \begin{verse}
    Eh quoi, ne sont-ils que des vagues\\
    dont se disperse le rideau,\\
    le temps finissant leur blague\\
    et recouvrant leurs os?
  \end{verse}
  \begin{verse}
    Mais moi je cours après le bonheur\\
    ainsi qu’une vague chassant sa prédécesseuse;\\
    j’ai encore trop froid sous la canicule élongatrice de maux\\
    et je disperse encore trop de mes membres et de mes mots.
  \end{verse}
  \begin{verse}
    Le losange aérien s’est aplati,\\
    les eaux se sont troublées -- et la Lady a jailli,\\
    d’un bond je me suis levé\\
    -- et, dis-moi, me suis-je enfui?
  \end{verse}
\PoemTitle{Bris de tête (21 Octobre 2011)}
  \begin{verse}
    Refuge blanc à carreaux,\\
    froid ainsi qu’un rire accusateur\\
    qui donne pourtant des rougeurs en plaques,
  \end{verse}
  \begin{verse}
    que tes rivets sautent et éclatent\\
    -- limités aux songes que nous sommes,\\
    nos horizons fermés aux yeux inconnus… 
  \end{verse}
  \begin{verse}
    de même démoliront!
  \end{verse}
  \begin{verse}
    Et la géométrie tyrannique\\
    peste incompréhensible et bubonique\\
    de même dépérira\\
    et sous le béton écroulé la vie
  \end{verse}
  \begin{verse}
    de même refleurira!
  \end{verse}
\PoemTitle{Mi soleil, mi … (27 Septembre 2011)}
  \begin{verse}
    (Je marchais au soleil\\
    le bord du lac de La Source\\
    quand de ses fleurs vermeilles\\
    il inspira ma bouche)
  \end{verse}
  \begin{verse}
    «~Un petit insecte vert s’est agrippé à ma chaussure\\
    et une petite femme s’est enchainée à mon cœur;\\
    j’ai retiré le verdâtre pour son malheur\\
    -- ce qu’elle résiste à cette usure!~»
  \end{verse}
  \begin{verse}
    Elle ne brille pas autant que le Soleil qui se couche\\
    mais bien plus que lui, elle réchauffe\\
    et craquèle par ses vocales escarmouches\\
    les masques qui me faussent.
  \end{verse}
\PoemTitle{Diversion (11 Octobre 2011)}
  \begin{verse}
    Deux posées\\
    valent un paradis,\\
    que coure le tram\\
    -- c’est un lieu infini.
  \end{verse}
  \begin{verse}
    Ce que valent quelques vents\\
    rien sinon ce qu’on y trouve,\\
    la pensée sa douve\\
    et la fraternité s’empêche.
  \end{verse}
  \begin{verse}
    Prenons les lieux, introduisons le verbe,\\
    tout sera dément -- et à nous:\\
    le chaudron sera porté sur elles\\
    et donnera d’allemands coquillages.
  \end{verse}
\PoemTitle{Et le temps ordonna la suspension d’audience (03 Octobre 2011)}
  \begin{verse}
    Retour arrière
    -- supprimons nos pas,\\
    franchissons la barrière\\
    -- ne nous éloignons pas.
  \end{verse}
  \begin{verse}
    Le temps se fige ainsi que les poumons\\
    -- les murènes et requins sommeillent,\\
    la vie revient aux mômes bougons,\\
    l’attention à la beauté s’éveille.
  \end{verse}
  \begin{verse}
    Détails {\small \emph{minuscules}} -- livre lu,\\
    pensées que l’on veut:\\
    au pire pleurera-t-on un peu\\
    si rien n’y \textbf{lu-it}.
  \end{verse}
  \begin{verse}
    Gongs numériques, taisez-vous!\\
    À vos binaires cerveaux policés\\
    -- j’ordonne que vous ralentissiez.
  \end{verse}
  \begin{verse}
    Merci, merci, merci;\\
    tant de mots pour elle,\\
    ailes oblongues aux éclats de dilemme et de rougeurs.
    \footnote{Je ne sais pas ce que ça veut dire. C’est juste beau.}
  \end{verse}
