	%*** Licence ***

%Cette œuvre est diffusée sous les termes de la license Creative Commons
%«~CC BY-NC-SA 3.0~», ce qui signifie que :

%Vous êtes libres :

  %* de reproduire, distribuer et communiquer cette création au public ;
  %* de modifier cette création.

%Selon les conditions suivantes :
    	
  %* Paternité - vous devez citer le nom de l'auteur original de la manière indiquée par l'auteur de l'œuvre ou le
		%titulaire des droits qui vous confère cette autorisation (mais pas d'une manière qui suggérerait
		%qu'ils vous soutiennent ou approuvent votre utilisation de l'œuvre).
  %* Pas d’utilisation commerciale - Vous n'avez pas le droit d'utiliser cette œuvre à des fins commerciales. 
  %* Partage des conditions initiales à l'identique - si vous transformez ou modifiez cette œuvre pour en créer une nouvelle,
  						     %vous devez la distribuer selon les termes du même contrat ou avec une
						     %licence similaire ou compatible.

%{Comprenant bien que :

  %* Renoncement} - N'importe quelle condition ci-dessus peut être retirée si vous avez l'autorisation du détenteur des droits.
  %* Domaine public - Là où l'œuvre ou un quelconque de ses éléments est dans le domaine public selon le droit applicable, ce statut
		     %n'est en aucune façon affecté par le contrat.
  %* Autres droits - d'aucune façon ne sont affectés par le contrat les droits suivants :
    %- Vos droits de distribution honnête ou d’usage honnête ou autres exceptions et limitations au droit d’auteur applicables;
      %Les droits moraux de l'auteur;
    %- Droits qu'autrui peut avoir soit sur l'œuvre elle-même soit sur la façon dont elle est utilisée, comme la publicité
      %ou les droits à la préservation de la vie privée.

%=== Note ===

%Ceci est le résumé explicatif du Code Juridique; La version intégrale du contrat est consultable ici:
%<http://creativecommons.org/licenses/by-nc-sa/3.0/legalcode>.

\PoemTitle{Intoxication (25 Septembre 2011)}
  \begin{verse}
    C’était le soir où le nerd donnait l’opium\\
    à s’en rompre le membre caudal,\\
    la Horde suintait le rhum\\
    et les corps d’amour -- quedalle! 
  \end{verse}
  \begin{verse}
    Oh  dormir  sans  ronfler,  la  joie  des  sans-vie!\\
    Que  de  tombes  dans l’addiction, d’illusions et de défaites;\\
    Ainsi  que  le  nihiliste  déblatérant  «~À quoi  bon  vivre?  Vive  le
    Paradis!~»\\
    nous  trouvons le  matériau à  fantasme parmi  les insomnies  propres aux
    jours des gris.\footnote{Allusion à un album de Sonata Artica,\emph{The Days of 
    Grays}}
  \end{verse}
  \begin{verse}
    Mœurs, que vous êtes électriques!\\
    Dormir nuit et jour, votre nouveau joug,\\
    nouveau dédain encordeur de cous\\
    à nous, les vers frénétiques.
  \end{verse}
\PoemTitle{Vie de famille (6 Octobre 2011)}
  \begin{verse}
    Je ne me souviens que de sa bouche\\
    (ainsi ais-je perdu ses yeux)\\
    -- elle s’est renfermée tranchant le vide\\
    et la rampante a éclos;
  \end{verse}
  \begin{verse}
    Il faut plonger pour ne plus rien dire:\\
    mêmes actes aux mêmes produits,\\
    tout recommence et pourrit,\\
    ces caveaux ne vont pas me suffire.
  \end{verse}
  \begin{verse}
    Qu’est-ce qu’il y a par là?\\
    -- Le grand cri vers le monde:\\
    passent la distraction et une foule de jours las,\\
    il va se cacher et s’effondre.
  \end{verse}
  \begin{verse}
    Hé, s’il te plait,\\
    arrache moi de ce trou,\\
    hé s’il te plait,\\
    ~~~~consumes-moi,
  \end{verse}
  \begin{verse}
    mais ne laisses rien du tout.
  \end{verse}
\PoemTitle{Corbaques (10 Octobre 2011)}
  \begin{verse}
    Frappez les quatre coups,\\
    ouvrez le rideau\\
    -- ils ne s’enfuiront pas moins vite\\
    offrant le spectacle aux badauds.
  \end{verse}
  \begin{verse}
    Nos voix se perdent désharmonisées;\\
    «~Wir sind frei~»\\
    faut-il que c’la vous effraye\\
    si c’est immonde à rimer?
  \end{verse}
  \begin{verse}
    Le rationnel, héros tragique\\
    bave clairement à l’inverse d’un Clappique;\\
    les sentiments posés en offrande\\
    de coups de semonces se défendent:
  \end{verse}
  \begin{verse}
    «~Fendons nous pour briser le néant!\\
    L’air (nous en sommes sûrs) est fait\\
    pour être par nous coloré!
  \end{verse}
  \begin{verse}
    Il n’y a pas jusqu’à notre détracteur\\
    qui ne vienne vers nous pour nous adorer!~»
  \end{verse}
  \begin{verse}
    Intelligence, détruis-toi un peu,\\
    laisses les émotions souffler la réplique!\\
    (Pour qui se cherche un peu\\
    cet ordre s’applique.)
  \end{verse}
  \begin{verse}
    Tirons le rideau, ouvrons nos bras,\\
    nos inspirateurs sont ici\\
    ~~~~-- et eux ne s’enfuiront pas.
  \end{verse}
\PoemTitle{Frénésies (26 Octobre 2011)}
  \begin{verse}
    Cheminant sur les voies sanguines\\
    embrasant les vides pulmonaires\\
    se foutant de vos gueules salines\\
    avance la bouffée de colère.
  \end{verse}
  \begin{verse}
    Que plus rien ne soit\\
    et brûle intensément\\
    -- il n’y a qu’à elle que je sursois\\
    et de m’en servir comme lavement.
  \end{verse}
  \begin{verse}
    Je baise Œdipe et l’allemand aussi\\
    -- qu’on les démembre gaiement\\
    et que reste la joie -- seulement.
  \end{verse}
\PoemTitle{Séquelles inachevées (Octobre 2011)}
  \begin{verse}
    Mon crâne lourd s’accommode des échos\\
    -- c’est le soir, la rentrée des vieillots;\\
    je me perds somnolant (il pleut)\\
    fondu dans ce manteau bleu.
  \end{verse}
  \begin{verse}
    Peu avant j’étouffais\\
    comme mordu par la bête urbanisée;\\
    peu avant \emph{ça} j’étais dissous\\
    le moral en berne, la queue en dessous;
  \end{verse}
  \begin{verse}
    Plus de douleur mais suffocations d’être \emph{dans les temps},\\
    chimères en danse et pauvre cloche felée;\\
    je ne m’oublie plus -- OK.\\
    Mais où me retrouver?
  \end{verse}
\PoemTitle{L’éminence grise (25 Octobre 2011)}
  \begin{verse}
    Rien à faire d’autre que de subir\\
    -- la peine sera ruinée dans l’action;\\
    l’audace et sa saveur donnent le ton\\
    et mon spectacle sera un éclat de rire.
  \end{verse}
  \begin{verse}
    Moquons nous ailleurs s’il le faut,\\
    les briques grises s’empilent dans nos têtes;\\
    tout est bétonné dans cette étrange fête\\
    où les mâchoires appartiennent au cosmos.
  \end{verse}
  \begin{verse}
    Moins de temps, d’ennui toujours davantage.
  \end{verse}
\PoemTitle{Le chiffre (21 Octobre 2011)}
  \textbf{Exercice} -- Vous placerez les mots suivants à leur bonne place
  dans le texte:
  \textit{fibre}; \textit{immonde}; \textit{charnier}; \textit{dubitatif};
   \textit{chat}; \textit{morphéophage}; \textit{tombeau}

  \begin{verse}
    On dirait qu’aveugle hors de moi je me trouve\\
    et des sons me déchirent la \_\_\_\_\_\\
    et beuglent «~Vas-t’en!\\
    Tu n’es pas libre!~»
  \end{verse}
  \begin{verse}
    Face aux monstres du quotidien \_\_\_\_\_\_\_\\
    je mentirais\\
    -- plutôt un \_\_\_\_\_\_\_\_\\
    que tous les pères du monde;
  \end{verse}
  \begin{verse}
    Les cris du \_\_\_\_ se propagent aux murs\\
    et les miens sous le cran de sûreté\\
    par les prétextes que les politiques font\\
    bientôt détoneront.
  \end{verse}
  \begin{verse}
    Sitôt \_\_\_\_\_\_\_\_\_\_\_\_ je ne serais plus\\
    -- et seulement la nuit passée, je saurais me retrouver;\\
    le \_\_\_\_\_\_\_ restera vide et étroit:\\
    pour l’instant, le reste m’échoit.
  \end{verse}
\PoemTitle{Récursivité (Octobre 2011)}
  \begin{verse}
    Mal être de la répétition,\\
    j’étais ce que tu seras\\
    et sans qu’on s’on aperçoive de la malédiction\\
    nous marcherons sur les mêmes pas.
  \end{verse}
  \begin{verse}
    Variations sans égales\\
    mais si peu de temps\\
    -- que valent nos choix\\
    si aucun écho ne s’en ressent?
  \end{verse}
  \begin{verse}
    Partout ce qui en nous choit,\\
    parmi cela et ce que l’on sent\\
    et ce qui chez toi monte d’autant\\
    -- où trouver notre régal?
  \end{verse}
  \begin{verse}
    Galérienne de vie parallèle je t’aime,\\
    fleurs écrasées le matin revenues aux nues,\\
    particules qui se cognent à en perdre haleine\\
    s’embrassent et annoncent ta venue,\\
    le temps nu et pour nous\\
    où se perdent les lignes du Cou.
  \end{verse}
