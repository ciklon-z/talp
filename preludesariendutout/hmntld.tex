%	 *** Licence ***

% Cette œuvre est diffusée sous les termes de la license Creative Commons
% «~CC BY-NC-SA 3.0~», ce qui signifie que :

% Vous êtes libres :

%   * de reproduire, distribuer et communiquer cette création au public ;
%   * de modifier cette création.

% Selon les conditions suivantes :
    	
%  * Paternité - Vous devez citer le nom de l'auteur original de la manière indiquée par l'auteur de l'œuvre ou le
%	         titulaire des droits qui vous confère cette autorisation (mais pas d'une manière qui suggérerait
%	         qu'ils vous soutiennent ou approuvent votre utilisation de l'œuvre).
%  * Pas d’utilisation commerciale - Vous n'avez pas le droit d'utiliser cette œuvre à des fins commerciales. 
%  * Partage des conditions initiales à l'identique - Si vous transformez ou modifiez cette œuvre pour en créer une nouvelle,
% 						      vous devez la distribuer selon les termes du même contrat ou avec une
%					              licence similaire ou compatible.

% Comprenant bien que :

%  * Renoncement - N'importe quelle condition ci-dessus peut être retirée si vous avez l'autorisation du détenteur des droits.
%  * Domaine public - Là où l'œuvre ou un quelconque de ses éléments est dans le domaine public selon le droit applicable, ce statut
%		      n'est en aucune façon affecté par le contrat.
%  * Autres droits - d'aucune façon ne sont affectés par le contrat les droits suivants :
%    - Vos droits de distribution honnête ou d’usage honnête ou autres exceptions et limitations au droit d’auteur applicables;
%    - Les droits moraux de l'auteur;
%    - Les droits qu'autrui peut avoir soit sur l'œuvre elle-même soit sur la façon dont elle est utilisée, comme la publicité
%      ou les droits à la préservation de la vie privée.

% === Note ===

% Ceci est le résumé explicatif du Code Juridique; La version intégrale du contrat est consultable ici:
% <http://creativecommons.org/licenses/by-nc-sa/3.0/legalcode>.

\PoemTitle{En devenir (30 Janvier 2012)}
  \begin{verse}
    Nous sommes ainsi que des larves d’humains\\
    dans des couloirs bien étroits\\
    -- nos mains sont gluantes de solitude et d’ennui;
  \end{verse}
  \begin{verse}
    nous donnons des \textsc{Majuscules} à nos drogues\\
    pour nous persuader de les aimer\\
    -- ainsi le Sang, les chairs et l’Amour.
  \end{verse}
\PoemTitle{Vieux murs (26 Janvier 2012)}
  \begin{verse}
    Il me faut des bouches à nourrir\\
    -- à donner à d’autres\\
    et probablement autant de phrases\\
    pour tout piétiner;
  \end{verse}
  \begin{verse}
    J’enfume les rues à ma façon\\
    -- la clameur dans l’étuve\\
    l’idéal le plus acide --\\
    dissous dans la petite cuve.
  \end{verse}
  \begin{verse}
    Éclats\\
    qui déchirent toute chair\\
    (je comprend là dedans le ciel)\\
    -- ces mers brulées noires.
  \end{verse}
\PoemTitle{Interconnexion (29 Janvier 2012)}
  \begin{verse}
    Étoiles que nous sommes\\
    (certes pas à tous, mais tous brillant\\
    ainsi que nous le voulons)\\
    nous avons la cervelle enhypée :
  \end{verse}
  \begin{verse}
    Ensemble sans être sur la place,\\
    (c’est, cela dit, pour bientôt)\\
    nous n’avons jamais autant serré de mains\\
    (et si peu de corps…);
  \end{verse}
  \begin{verse}
    nous n’avons pas quitté les monastères d’où tremblent les volontés
    et les vieilleries divines pour nous contenter,\\
    un simple dieu, électroménager, \\
    autocuiseur, oubli, \textsc{télévision HD},\\
    pommes de pétrole -- le sacre de la marchandise et des marchés…
  \end{verse}
  \begin{verse}
    Hommes mégaphones, émigrés d’Oéretéheffe:\\
    on n’arrête pas les tempêtes en vitupérant\\
    -- nous sommes ailleurs (déjà!) et pas sans arguments.
  \end{verse}
\PoemTitle{Gilgamesh (2 Février 2012)}
  \begin{verse}
    Nous ne sommes plus des hyperboréens:\\
    plus de centimètre qui ne soit inconnu,\\
    des câbles électriques, une grande LAN
  \end{verse}
  \begin{verse}
    tant d’hommes, si peu de chemins,\\
    la guerre et la rage accrue\\
    serrent à la gorge comme une liane.
  \end{verse}
  \begin{verse}
    Alcools, dieux, places irréelles carbonisées\\
    -- quelque chose pour quoi vivre,\\
    où demeurer et s’éteindre
  \end{verse}
  \begin{verse}
    -- elles se sont enfuies elles qui devaient partir après nous,\\
    plus de cartes qui ne soient utiles, que pour explorer Mars ou les nébuleuses\\
    -- trou noir de la connaissance, temps sans consistance,\\
    toujours plus de vitesse qui ne veut rien dire et plus de moments pour donner.
  \end{verse}
\PoemTitle{Perruches et loups empaffés (14 Février 2012)}
  \begin{verse}
    Habitués des tramways\\
    -- yeux sur pattes angoissés\\
    ou pire, le néant dans la cavité
  \end{verse}
  \begin{verse}
    \textsc{Cela va sans dire}
  \end{verse}
  \begin{verse}
    -- évidemment que c’est le bien,\\
    ne marche plus mon gueux\\
    que vers ton office (affreux).
  \end{verse}
  \begin{verse}
    Et dans ces voies aux nouveaux détours\\
    la mystique marchande prend contour\\
    -- n’arrêtons pas de refluer\\
    -- la vie comme marée inversée…
  \end{verse}
  \begin{verse}
    \textsc{Jolie couleur du peintre-horizon}
  \end{verse}
  \begin{verse}
    (~Verdâtres de mon cœur\\
    ~~glissant comme la mousse du toit\\
    ~~dans le bain d’ammoniaque\\
    ~~au fond dégoulineront!~)
  \end{verse}
