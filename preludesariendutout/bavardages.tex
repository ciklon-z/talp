%	 *** Licence ***

% Cette œuvre est diffusée sous les termes de la license Creative Commons
% «~CC BY-NC-SA 3.0~», ce qui signifie que :

% Vous êtes libres :

%   * de reproduire, distribuer et communiquer cette création au public ;
%   * de modifier cette création.

% Selon les conditions suivantes :
    	
%  * Paternité - Vous devez citer le nom de l'auteur original de la manière indiquée par l'auteur de l'œuvre ou le
%	         titulaire des droits qui vous confère cette autorisation (mais pas d'une manière qui suggérerait
%	         qu'ils vous soutiennent ou approuvent votre utilisation de l'œuvre).
%  * Pas d’utilisation commerciale - Vous n'avez pas le droit d'utiliser cette œuvre à des fins commerciales. 
%  * Partage des conditions initiales à l'identique - Si vous transformez ou modifiez cette œuvre pour en créer une nouvelle,
% 						      vous devez la distribuer selon les termes du même contrat ou avec une
%					              licence similaire ou compatible.

% Comprenant bien que :

%  * Renoncement - N'importe quelle condition ci-dessus peut être retirée si vous avez l'autorisation du détenteur des droits.
%  * Domaine public - Là où l'œuvre ou un quelconque de ses éléments est dans le domaine public selon le droit applicable, ce statut
%		      n'est en aucune façon affecté par le contrat.
%  * Autres droits - d'aucune façon ne sont affectés par le contrat les droits suivants :
%    - Vos droits de distribution honnête ou d’usage honnête ou autres exceptions et limitations au droit d’auteur applicables;
%    - Les droits moraux de l'auteur;
%    - Les droits qu'autrui peut avoir soit sur l'œuvre elle-même soit sur la façon dont elle est utilisée, comme la publicité
%      ou les droits à la préservation de la vie privée.

% === Note ===

% Ceci est le résumé explicatif du Code Juridique; La version intégrale du contrat est consultable ici:
% <http://creativecommons.org/licenses/by-nc-sa/3.0/legalcode>.

\PoemTitle{Poème à l’ail (31 Janvier 2012)}
  \begin{verse}
    Cela m’effraye\\
    de me cailler dans la neige\\
    au teintes caillées et beiges\\
    -- aie, ça caille
  \end{verse}
  \begin{verse}
    (mon attirail ???)
  \end{verse}
  \begin{verse}
    Chaleur en moi qui est une longue chute\\
    -- parapets de glaces,\\
    Ô radiateur!
  \end{verse}
  \begin{verse}
    (Je voulais honorer de grands hommes, de grands objets\\
    -- il n’y en avait pas dans la chambrée,\\
    sauf celui-là, instrument pour le cancre boiteux)
  \end{verse}
  \begin{center}
    \textsc{Annotation sublime}
  \end{center}
\PoemTitle{Aux huiles essentielles de Polésie (29 Janvier 2012)}
  \begin{verse}
    À sommeiller au plus tard\\
    Deux, trois heures,\\
    on attrape de drôles de papillons --
  \end{verse}
  \begin{verse}
    le décor est tout sombre et il me semble\\
    (ainsi qu’à toi, petit vampire)\\
    qu’on souffle ces mots là, la tête tranchée:
  \end{verse}
  \begin{verse}
    «~Ô vénéré 60,\\
    Ô couleurs primaires~»
  \end{verse}
  \begin{verse}
    au delà, un flot de paroles que je ne saisis plus,\\
    les nerfs tendus en arquebuse\\
    -- le réveil en somme agacé.
  \end{verse}
  \begin{verse}
    Huit heures -- la lumière claire est froide,\\
    j’ai perdu un charnier d’abondance moite\\
    plein de mots à rebours et de folies\\
    -- ainsi que des soleils dans des flaques.
  \end{verse}
\PoemTitle{Orgasme capillaire (29 Janvier 2012)}
  \begin{verse}
    Parfums hydroliques qui vous prennent la chaleur dans le ventre sur le fait,\\
    le sommeil qui abonde comme l’hémoglobine en Syrie et ailleurs,\\
    tout est calme et miséreux, volupté vulgaire, disent-ils, ce n’est que 
    peu de chose,\\
    des doigts, dix et c’est agréable.
  \end{verse}
\PoemTitle{Headbanging (6 Février 2012)}
  \begin{verse}
    Plus de monde en deuil\\
    -- je veux vos cornes à vos doigts,\\
    compter si l’ambiance est là\\
    et divaguer l’œil vide;
  \end{verse}
  \begin{verse}
    quand cette terre et l’autre\\
    en cœur retentissent\footnote{Allusion à une chanson d’Iron Maiden : \emph{When Two Worlds Collide}.}\\
    -- tu n’es plus rien -- si ce n’est le nouveau monde\\
    (laisses le vieux aux «~braves~» gens\\
    qui n’ont rien à brouter)…
  \end{verse}
  \begin{verse}
    Jusqu’à l’aube encore il faudra sautiller\\
    et boire et chanter et aimer,\\
    le ronronnement des foules possédées…\\
    -- être ailleurs est déjà suspect…
  \end{verse}
\PoemTitle{Plaisante, ris! (17 Février 2012)}
  \begin{verse}
    J’ouvris la fenêtre\\
    -- des vers me tombèrent dessus en grouillant\\
    ils avaient le verbe frétillant et comique\\
    pour un propos que je voulais épidictique;
  \end{verse}
  \begin{verse}
    À baffrer sans failles je me suis perdu\\
    -- plus de plans ni de carte\\
    et l’angoisse qui marche\\
    à grandes enjambées\\
    -- au fond pour rien, on me l’a atomisée:
  \end{verse}
  \begin{verse}
    Tenons donc ici des propos normands;
  \end{verse}
  \begin{verse}
    Irina la douce rousse\\
    fou foutoir d’allitérations\\
    -- \textsc{Quand on ne sait rien on se tait}
  \end{verse}
  \begin{verse}
    (Mais qui aime suivre les ordres?)
  \end{verse}
  \begin{verse}
    En malpoli cimmérien\\
    -- je dis merci hein\footnote{Le jeu de mot n’est pas évident; ci-mmér-ien = mer-ci-hein}\\
    pour le service rendu, la compagnie\\
    et le verbe qui donne -- rien sinon le paradoxe!
  \end{verse}
\PoemTitle{Réminiscences (29 Janvier 2012)}
  \begin{verse}
    Ville de grès qui tremble\\
    aqueducs flambant ironiques\\
    -- les flammes papillons des Trop-Vieux.
  \end{verse}
  \begin{verse}
    État qui n’est rien sinon vous,\\
    sources d’où le terme viendra\\
    offrir son dû à la salvation,
  \end{verse}
  \begin{verse}
    tyrannies des cordoigts\footnote{Au départ, \textit{corps-doigts}}\\
    «~ne dis plus rien aux insectes~»\\
    quatrepatté de brumes.\footnote{Au départ, \textit{quatre-patté}}
  \end{verse}
\PoemTitle{Brebis urbaines enragées (14 Février 2012)}
  \begin{verse}
    C’est le soir et le jour est revenu\\
    -- palettes à nouveau cailloux.
  \end{verse}
  \begin{verse}
    Hé! Regarde cette rue!
  \end{verse}
  \begin{verse}
    On aurait dit un milles-pattes,\\
    mais -- disparu! Et pourquoi?\\
    Je ne vois que des culs-de-jatte là,\\
    et de la cervelle sur laquelle marcher.
  \end{verse}\\
