%	 *** Licence ***

% Cette œuvre est diffusée sous les termes de la license Creative Commons
% «~CC BY-NC-SA 3.0~», ce qui signifie que :

% Vous êtes libres :

%   * de reproduire, distribuer et communiquer cette création au public ;
%   * de modifier cette création.

% Selon les conditions suivantes :
    	
%  * Paternité - Vous devez citer le nom de l'auteur original de la manière indiquée par l'auteur de l'œuvre ou le
%	         titulaire des droits qui vous confère cette autorisation (mais pas d'une manière qui suggérerait
%	         qu'ils vous soutiennent ou approuvent votre utilisation de l'œuvre).
%  * Pas d’utilisation commerciale - Vous n'avez pas le droit d'utiliser cette œuvre à des fins commerciales. 
%  * Partage des conditions initiales à l'identique - Si vous transformez ou modifiez cette œuvre pour en créer une nouvelle,
% 						      vous devez la distribuer selon les termes du même contrat ou avec une
%					              licence similaire ou compatible.

% Comprenant bien que :

%  * Renoncement - N'importe quelle condition ci-dessus peut être retirée si vous avez l'autorisation du détenteur des droits.
%  * Domaine public - Là où l'œuvre ou un quelconque de ses éléments est dans le domaine public selon le droit applicable, ce statut
%		      n'est en aucune façon affecté par le contrat.
%  * Autres droits - d'aucune façon ne sont affectés par le contrat les droits suivants :
%    - Vos droits de distribution honnête ou d’usage honnête ou autres exceptions et limitations au droit d’auteur applicables;
%    - Les droits moraux de l'auteur;
%    - Les droits qu'autrui peut avoir soit sur l'œuvre elle-même soit sur la façon dont elle est utilisée, comme la publicité
%      ou les droits à la préservation de la vie privée.

% === Note ===

% Ceci est le résumé explicatif du Code Juridique; La version intégrale du contrat est consultable ici:
% <http://creativecommons.org/licenses/by-nc-sa/3.0/legalcode>.

\PoemTitle{Lady NénuPhare (11 Février 2012)}
  \begin{verse}
    Mon amie, connais-tu simplement\\
    le délire qui m’a pris,\\
    mis entre tes griffes si insolemment?
  \end{verse}
  \begin{verse}
    Écoute donc, mon amie --\\
    apprécie ce spectacle!
  \end{verse}
  \begin{verse}
    Une fois de plus mon amie\\
    tu m’avais nourri -- pain de mon âme,\\
    comment te rendre grâce sans te dévorer?
  \end{verse}
  \begin{verse}
    Et puis j’eus de nouveaux songes\\
    tout vers toi tournés,\\
    de claires histoires sans paroles\\
    -- tes bras m’offraient ces hyperboles;
  \end{verse}
  \begin{verse}
    Par le pouvoir du verbe qui fait\\
    et qui à toute chose rend sa part honorable,\\
    nous avions
  \end{verse}
  \begin{verse}
    ~~~~des villes fraiches au bord des Cieux,\\
    ~~~~baignées du Soleil et de sources\\
    ~~~~-- le soir nous cuisait, entre les routes\\
    ~~~~qui partaient de là poussaient des pommiers;
  \end{verse}
  \begin{verse}
    ~~~~des nuits froides et poudreuses\\
    ~~~~derrière la vitre réfugiés\\
    ~~~~nous formions une figure de neige amoureuse\\
    ~~~~toute nonchalante entre les oreillers;
  \end{verse}
  \begin{verse}
    ~~~~des lunes nouvelles pour nous disposées\\
    ~~~~et des mains unies pour donner à la bagatelle\\
    ~~~~la forme honnie d’un cliché;
  \end{verse}
  \begin{verse}
    Ainsi étions nous, alors que le poète était;\\
    amoureux incomplets, fantasmes de papier\\
    volatiles comme l’éther et libres comme jamais.
  \end{verse}
  \begin{verse}
    ~~~~J’approuvais ton squelette et son pulpeux contour\\
    ~~~~tourné comme une virgule d’un récit poli\\
    ~~~~comme un vitrail,\\
    ~~~~seuls nous n’étions que demi-teinte\\
    ~~~~-- unis, la plus terrible des couleurs;
  \end{verse}
  \begin{verse}
    ~~~~pour la faim d’un homme courant,\\
    ~~~~faut-il plus qu’une majuscule, une fin?\\
    ~~~~La mère au commencement\\
    ~~~~-- l’amante étant le point divin;
  \end{verse}
  \begin{verse}
    Le monde sans toi, mon amie,\\
    ~~~~n’est que sérail terne\\
    ~~~~-- sans joie ni plaisir ni vie,\\
    ~~~~ni couleur ni odeur;
  \end{verse}
  \begin{verse}
    ~~~~un monde de bitume ennuyé,\\
    métallique et pierreux d’où viennent les regrets;
  \end{verse}
  \begin{verse}
    où l’on ne s’entend plus respirer\\
    mais bien pour truicider à bas prix\\
    ce monde vulgaire et rééel\\
    où sévit Bull-Amesys…
  \end{verse}
  \begin{verse}
    Je veux te manger comme une poire\\
    -- même tes yeux aux iris de miel;\\
    ou bien te rire au nez\\
    à quel point je t’adore
  \end{verse}
  \begin{verse}
    ~~~~-- au delà de toute chose,\\
    ~~~~je ne vais pas énumérer.
  \end{verse}
  \begin{verse}
    Tu es ma nitroglycérine préférée:\\
    te voilà et le verbe explose -- pure Poésie;\\
    partie -- hélas je compose
      \footnote{C’est bien dans le sens «~il est dommage que je compose~». Haters Gonna Hate.}
    -- pure poésie\\
    mais ce n’est que poudre aux yeux et chiures en débris…
  \end{verse}
  Combien de temps  encore avant que ma cervelle fume  et que tout s’effondre
  et  toi  et  moi  qui  crevèrent en  silence  dans  une  mémoire  délirée
  jusqu’au bout,  un vieux prétexte  peut-être pour imploser  simplement de
  musique de  regards de babioles  babéliennes de déceptions noires  de joies
  zarbies de proches  zombies et les yeux  de mes verdâtres pour  me donner la
  force de laisser tomber, combien de  temps encore avant que tout cela n’ait
  plus de  sens dans  l’écriture et  tout dans  la vie  et combien  de temps
  encore, dis combien encore, encore, encore
  \begin{center}
    \textsc{Boum.}
  \end{center}
\PoemTitle{Le blason phagocité (16 Février 2012)}
  \begin{verse}
    Goûter écarlate aux bas comme tes lèvres\\
    et l’éther hallucinatoire qui bleuit\\
    -- ce que l’on voit autour de soi:
  \end{verse}
  \begin{verse}
    je veux dire -- appeau à satyres\\
    minuscule et si effrayant\\
    (la preuve, te voilà, je détale)
  \end{verse}
  \begin{verse}
    tes cils et tes yeux à demi fermés\\
    %FIXME souligner beauté
    soulignent intégralement le mot \underline{beauté};\\
    à leur place,\\
    que de tartares à remplir par fournées\\
    -- si petite et raffinée\\
    tu laisse même les géants attérés.
  \end{verse}
\PoemTitle{Fantasme métalleux (3 Février 2012)}
  \begin{verse}
    Les pogo bat des bras\\
    le bec ouvert, une langue à cornes;\\
    moi? Je vis; empli de tritons
  \end{verse}
  \begin{verse}
    -- je fournis des Kadaths de peu de prix,\\
    à d’autres des murailles d’éther,\\
    à ceux qui en veulent -- encore et toujours
  \end{verse}
  \begin{center}
    \textsc{Du Spectacle}
  \end{center}
  \begin{verse}
    (Immondice)
  \end{verse}
  \begin{verse}
    Le nerf de la vie est un nerf de musique,
  \end{verse}
  \begin{verse}
    \emph{flûtes ignoblement décorées} ou non\\
    -- voici la chute et l’apothéose,\\
    un océan de noir dans le crépuscule rose.
  \end{verse}
\PoemTitle{Enthousiasme d’un terrorisme ciblé (11 Février 2012)}
  \begin{center}
    \textit{
      Proche, je n’aime pas le prochain:\\
      Qu’il s’en aille, loin, et s’élève!\\
      Comment sans cela deviendrait-il mon étoile?
    }

    \textsc{Nietzsche}, \emph{Le Gai Savoir}
  \end{center}

  \begin{verse}
    Ce qui est dit se fera\\
    -- retour des choses réprimées,\\
    à nous les yeux et le tromblon\\
    pour sonner les oreilles des comissariats;
  \end{verse}
  \begin{verse}
    la lumière comme du miel coulait\\
    -- c’est bon de vivre, je n’avais jamais essayé;\\
    carrier
      \footnote{Anglicisme, rouler. Allusion au \emph{Gunpowder Plot}.}
    de la poudre jusqu’en dessous des tombes\\
    -- en vain, peut-être, tout de même essayons.
  \end{verse}
\PoemTitle{Revenir à venir (16 Février 2012)}
  \begin{verse}
    Je dis -- \textsc{Oui} à la vie!
  \end{verse}
  \begin{verse}
    (~Mon agrément n’est pas voilé --\\
    ~~il fait pourtant\\
    ~~~~~~~~~~~~~~~~~~~~~~feu de tout son bois,\\
    ~~il peut pourtant\\
    ~~~~~~~~~~~~~~~~~~~~~~~ramener des lingots!~)
  \end{verse}
  \begin{verse}
    Pourquoi battre le pavé et le verbe\\
    -- ce fils mal élevé --\\
    cette chose qui nous échappe plus encore\\
    s’il est aimé?
  \end{verse}
  \begin{verse}
    Je veux récolter les vents et ramener les morts\\
    à moi -- puis donner à foison,\\
    n’avoir plus de mémoire\\
    pour être encore ce que je fus!
  \end{verse}
  \begin{verse}
    (~Plus de temples, s’il vous plait,\\
    ~~pour délecter la faucheuse de bobards d’Amériques;\\
    ~~cendres , bois et os, triple molosse\\
    ~~encore une fois mordez!~)
  \end{verse}
