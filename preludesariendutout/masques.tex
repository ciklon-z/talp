%	 *** Licence ***

% Cette œuvre est diffusée sous les termes de la license Creative Commons
% «~CC BY-NC-SA 3.0~», ce qui signifie que :

% Vous êtes libres :

%   * de reproduire, distribuer et communiquer cette création au public ;
%   * de modifier cette création.

% Selon les conditions suivantes :
    	
%  * Paternité - Vous devez citer le nom de l'auteur original de la manière indiquée par l'auteur de l'œuvre ou le
%	         titulaire des droits qui vous confère cette autorisation (mais pas d'une manière qui suggérerait
%	         qu'ils vous soutiennent ou approuvent votre utilisation de l'œuvre).
%  * Pas d’utilisation commerciale - Vous n'avez pas le droit d'utiliser cette œuvre à des fins commerciales. 
%  * Partage des conditions initiales à l'identique - Si vous transformez ou modifiez cette œuvre pour en créer une nouvelle,
% 						      vous devez la distribuer selon les termes du même contrat ou avec une
%					              licence similaire ou compatible.

% Comprenant bien que :

%  * Renoncement - N'importe quelle condition ci-dessus peut être retirée si vous avez l'autorisation du détenteur des droits.
%  * Domaine public - Là où l'œuvre ou un quelconque de ses éléments est dans le domaine public selon le droit applicable, ce statut
%		      n'est en aucune façon affecté par le contrat.
%  * Autres droits - d'aucune façon ne sont affectés par le contrat les droits suivants :
%    - Vos droits de distribution honnête ou d’usage honnête ou autres exceptions et limitations au droit d’auteur applicables;
%    - Les droits moraux de l'auteur;
%    - Les droits qu'autrui peut avoir soit sur l'œuvre elle-même soit sur la façon dont elle est utilisée, comme la publicité
%      ou les droits à la préservation de la vie privée.

% === Note ===

% Ceci est le résumé explicatif du Code Juridique; La version intégrale du contrat est consultable ici:
% <http://creativecommons.org/licenses/by-nc-sa/3.0/legalcode>.

\PoemTitle{Rome, ou presque (4 Février 2012)}
  \begin{verse}
    Je veux parler,\\
    mais ma voix comme un ballon dégonflé\\
    s’étrangle à mesure qu’elle veut se vider;
  \end{verse}
  \begin{verse}
    je veux partir,\\
    de tous les maux du monde en écho dans mon cervelet\\
    (cacophonie spectrale
      \footnote
      {
	C’est le titre d’un recueil de \textsc{Laurent Buscail}, paru chez In Libro Veritas.
      }, dirait l’autre);
  \end{verse}
  \begin{verse}
    je veux m’emplir,\\
    pourtant à bout de souffle, rien ne s’aspire,\\
    les bras restent attachés -- je veux l’occire;
  \end{verse}
  \begin{verse}
    me vider, me combler, au rythme d’un verre --\\
    j’ai les idées nettes et les paroles flasques,\\
    ô amour, pour toi que de frasques!
  \end{verse}
\PoemTitle{Dérivées (30 Janvier 2012)}
  \begin{verse}
    Ces messieurs des sciences nous sont bien étrangers\\
    (nous avons sans doute,\\
    pourtant,\\
    un peu d’orgueil en commun);
  \end{verse}
  \begin{verse}
    Nos langues pareilles\\
    n’ont le même langage\\
    que désorientées,
  \end{verse}
  \begin{verse}
    -- je veux dire, en dehors des gangues originelles,\\
    des corridors à nous et préconçus ainsi que des poires.
  \end{verse}
\PoemTitle{La Peste Noire (14 Février 2012)}
  \begin{verse}
    Geiser -- explosion de la terre ou du monde\\
    où même la pierre réchauffe les corps;\\
    la chaleur des morts n’en finit plus de tomber dans le siècle\\
    -- \textsc{Sans doute} moins de sang\\
    ~~~~-- mais plus d’ennui et de vide,\\
    ~~~~du bois à payer pour continuer de brûler\\
    ~~~~et se carboniser pour continuer de crever…
  \end{verse}
  \begin{verse}
    Geiser sans cervelle, tu ne verras rien mourir\\
    -- tu resteras immobile à gerber en ponctuel alcoolique;
  \end{verse}
  \begin{verse}
    Que bave ta vapeur!\\
    Les mots ne seront que de la buée\\
    colorée, certes, mais infertile.
  \end{verse}
\PoemTitle{Une question de connivence (29 Janvier 2012)}
  \begin{verse}
    Carnet de frites de dés en poudre\\
    (la vieille querelle amerloquée des frenchis bavards)\\
    Souhaites-en moi des carnets et des jeux\\
    -- \textsc{en voici quatre}:
  \end{verse}
  \begin{center}
    \textsc{JE~JE~JE~JE}
  \end{center}
  \begin{verse}
    (Mais au fond, il n’y en a qu’un seul,\\
    avec toi, d’outre-lignes)
  \end{verse}
\PoemTitle{19-347 (26 Janvier 2012)}
  \begin{verse}
    Quand je pose les mains sur les murs\\
    aussi lisses et beaux soient-ils,\\
    «~c’est de la matière insaisissable~».
  \end{verse}
  \begin{verse}
    Que des petits riens et le délire,\\
    est-ce que tout cela est\\
    «~Suis-je sur du papier?~»
  \end{verse}
  \begin{verse}
    (L’un et l’autre\\
    se valent\\
    -- peut-être,\\
    ce doit dépendre des épaules)
  \end{verse}
  \begin{verse}
    Il va falloir boire encore\\
    avant de…
  \end{verse}
  \begin{verse}
    «~Alors, on a des oiseaux\\
    de proie dans le ventre?~»
  \end{verse}
\PoemTitle{Mille Sabords (15 Février 2012)}
  \begin{verse}
    C’est un rêve -- donc te voilà\\
    -- ô ma reine, tu n’es ici que moi\\
    -- quelle décapitation!
  \end{verse}
  \begin{verse}
    J’ai pour moi l’imagination et la sottise\\
    de te faire danser, pantin d’hécatombe\\
    à ma guise -- le songe tombe
  \end{verse}
  \begin{center}
    \textsc{Misère ordinaire}
  \end{center}
  \begin{verse}
    solitude de cabot à glucose,\\
    dépossession chronique du timide\\
    -- les mots que l’on a donnent le cynique que l’on est.
  \end{verse}
\PoemTitle{Ces années là (17 Février 2012)}
  \begin{verse}
    D’ici tonne la duplication\\
    des feuilles -- et d’autres choses aussi\\
    -- lieu d’études, vraiment?\\
    Soyons plus incisifs\\
    donc gueulons comme un bousier\\
    -- la mâchoire éclopée\\
    antre à humidité itérative\\
    -- là est le bouvier.
  \end{verse}
  \begin{verse}
    rames croisées et rimes démêlées\\
    dévoreuse de béton -- université,\\
    que de couleurs perdues…
  \end{verse}
\PoemTitle{Nécrophagie civilisationnelle (15 Février 2012)}
  \begin{verse}
    Cauchemar du versificateur;\\
    pire, du politicien, du casseur de noix\\
    -- «~être pour être dévoré
  \end{verse}
  \begin{verse}
    non par les vers ou la gangrène\\
    -- mais par mes semblables\\
    puis la plèbe entière!~»
  \end{verse}
  \begin{verse}
    C’est ainsi qu’arrivent clochers et parnasses,\\
    pour le simple présent qui nous lasse\\
    et même la seconde, la vie qui passe\\
    est requis trois cimetières qui s’entassent.
  \end{verse}
