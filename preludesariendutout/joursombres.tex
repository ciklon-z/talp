%	 *** Licence ***

% Cette œuvre est diffusée sous les termes de la license Creative Commons
% «~CC BY-NC-SA 3.0~», ce qui signifie que :

% Vous êtes libres :

%   * de reproduire, distribuer et communiquer cette création au public ;
%   * de modifier cette création.

% Selon les conditions suivantes :
    	
%  * Paternité - Vous devez citer le nom de l'auteur original de la manière indiquée par l'auteur de l'œuvre ou le
%	         titulaire des droits qui vous confère cette autorisation (mais pas d'une manière qui suggérerait
%	         qu'ils vous soutiennent ou approuvent votre utilisation de l'œuvre).
%  * Pas d’utilisation commerciale - Vous n'avez pas le droit d'utiliser cette œuvre à des fins commerciales. 
%  * Partage des conditions initiales à l'identique - Si vous transformez ou modifiez cette œuvre pour en créer une nouvelle,
% 						      vous devez la distribuer selon les termes du même contrat ou avec une
%					              licence similaire ou compatible.

% Comprenant bien que :

%  * Renoncement - N'importe quelle condition ci-dessus peut être retirée si vous avez l'autorisation du détenteur des droits.
%  * Domaine public - Là où l'œuvre ou un quelconque de ses éléments est dans le domaine public selon le droit applicable, ce statut
%		      n'est en aucune façon affecté par le contrat.
%  * Autres droits - d'aucune façon ne sont affectés par le contrat les droits suivants :
%    - Vos droits de distribution honnête ou d’usage honnête ou autres exceptions et limitations au droit d’auteur applicables;
%    - Les droits moraux de l'auteur;
%    - Les droits qu'autrui peut avoir soit sur l'œuvre elle-même soit sur la façon dont elle est utilisée, comme la publicité
%      ou les droits à la préservation de la vie privée.

% === Note ===

% Ceci est le résumé explicatif du Code Juridique; La version intégrale du contrat est consultable ici:
% <http://creativecommons.org/licenses/by-nc-sa/3.0/legalcode>.

\PoemTitle{Pègre de plateau (4 Février 2012)}
  \begin{verse}
    Fantômes et banshees\\
    en teintes blanches,\\
    spectres de voleurs et de voyous\\
    jamais de ta couleur à toi,
  \end{verse}
  \begin{verse}
    le veilleur de nuit chasse les idées,\\
    garde bien fermés les coffres à CD\\
    (dont on ne veut plus, mais n’allez pas lui dire);\\
  \end{verse}
  \begin{verse}
    Horreur, le Réseau manifeste\\
    -- un \textsc{Ta Gueule} prononcé comme un javelot\\
    et voilà crachats et salauds\\
    unanimement tournés vers la presse;
  \end{verse}
  \begin{verse}
    le veilleur de nuit tire sur la Liberté,\\
    le savoir ne se pèse qu’en billets,
  \end{verse}
  \begin{verse}
    (cela veut dire : soyez résignés)
  \end{verse}
  \begin{verse}
    (cela veut dire : mourrez)
  \end{verse}
\PoemTitle{Universal Mafia (26 Janvier 2012)}
  \begin{verse}
    Mon ami,\\
    ne demande jamais la permission\\
    pour jouir de la musique\\
    des mots des yeux,
  \end{verse}
  \begin{verse}
    ce sont des amitiés belles\\
    et aussi fortes que toi;
  \end{verse}
  \begin{verse}
    quand ta compagne te lâche\\
    (ainsi qu’un rondin fendu)\\
    tu peux toujours lire\\
    -- ou te balader -- chose pareille et si belle;
  \end{verse}
  \begin{verse}
    tu le sais, il y a la pègre\\
    de ceux qui voudraient
  \end{verse}
  \begin{verse}
    un péage pour chaque idée et un denier pour chaque soupir, chaque sourire
    et pour tout ce qui nous fait et jamais ne leur appartient…
  \end{verse}
  \begin{verse}
    -- Partage, mon ami,\\
    partage avec moi, avec le monde entier\\
    -- ensemble pour bâtir autre chose\\
    que leurs réserves toussant la morosité.
  \end{verse}
\PoemTitle{Le travail rend libre ? (17 Février 2012)}
  \begin{verse}
    Brousailles grillées orange\\
    étangs de plastique -- et pourquoi?\\
    -- ce n’est rien, c’est l’humain.
  \end{verse}
  \begin{verse}
    Ailleurs de ces travaux qu’ils vénérent\\
    -- je vis, moi n’est plus un autre\\
    -- fiction? Sans doute,\\
    mais tout peut changer
  \end{verse}
  \begin{center}
    \textsc{d’un coup}
  \end{center}
  \begin{quote}
    \textbf{Règle n°33}

    Quand vous ne pouvez obliger les hommes par la publicité,
    écrivez vos réclames sur des fourneaux.
  \end{quote}
  \begin{verse}
    (~Ou bien c’est l’«~extrême~»\\
    ~~prononcé avec des accents de chèvre,\\
    ~~trémolos rigolos au degré second\\
    ~~qu’on cache bien au fond~)
  \end{verse}
\PoemTitle{Crèves, ACTA! (11 Février 2012)}
  \begin{verse}
    Bréviaire des intenses moments:\\
    quand j’aime au delà de toute chose\\
    quand le métal me ramollit la fibre de joie\\
    quand Guy Fawkes rit contre eux et avec moi;
  \end{verse}
  \begin{verse}
    N’étant pas d’ici -- patrie des verdâtres\\
    aux fruits si peu récoltés -- Francie,\\
    je ne t’aime pas en putain décharnée\\
    quand tu ne sais donner -- la Liberté;
  \end{verse}
  \begin{verse}
    L’Internet, c’est chez moi et chez nous\\
    -- le seul lieu où nos échos\\
    ont une couleur vraie -- pis, uniforme
  \end{verse}
  \begin{verse}
    -- je veux dire, ces tyrans médiatiques\\
    que cela les ennuie de penser sans pyramides,
  \end{verse}
  \begin{verse}
    je veux dire que c’est pour nous une chance,\\
    enfin, de les détruire et d’aimer et de vivre.
  \end{verse}
