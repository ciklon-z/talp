%	 *** Licence ***

% Cette œuvre est diffusée sous les termes de la license Creative Commons
% «~CC BY-NC-SA 3.0~», ce qui signifie que :

% Vous êtes libres :

%   * de reproduire, distribuer et communiquer cette création au public ;
%   * de modifier cette création.

% Selon les conditions suivantes :
    	
%  * Paternité - vous devez citer le nom de l'auteur original de la manière indiquée par l'auteur de l'œuvre ou le
%	         titulaire des droits qui vous confère cette autorisation (mais pas d'une manière qui suggérerait
%	         qu'ils vous soutiennent ou approuvent votre utilisation de l'œuvre).
%  * Pas d’utilisation commerciale - Vous n'avez pas le droit d'utiliser cette œuvre à des fins commerciales. 
%  * Partage des conditions initiales à l'identique - si vous transformez ou modifiez cette œuvre pour en créer une nouvelle,
% 						      vous devez la distribuer selon les termes du même contrat ou avec une
%					              licence similaire ou compatible.

% Comprenant bien que :

%  * Renoncement - N'importe quelle condition ci-dessus peut être retirée si vous avez l'autorisation du détenteur des droits.
%  * Domaine public - Là où l'œuvre ou un quelconque de ses éléments est dans le domaine public selon le droit applicable, ce statut
%		      n'est en aucune façon affecté par le contrat.
%  * Autres droits - d'aucune façon ne sont affectés par le contrat les droits suivants :
%    - Vos droits de distribution honnête ou d’usage honnête ou autres exceptions et limitations au droit d’auteur applicables;
%    - Les droits moraux de l'auteur;
%    - Les droits qu'autrui peut avoir soit sur l'œuvre elle-même soit sur la façon dont elle est utilisée, comme la publicité
%      ou les droits à la préservation de la vie privée.

% === Note ===

% Ceci est le résumé explicatif du Code Juridique; La version intégrale du contrat est consultable ici:
% <http://creativecommons.org/licenses/by-nc-sa/3.0/legalcode>.


\PoemTitle{Amours du camp (28 Novembre 2011)}
  \begin{verse}
    Arrache moi une case\\
    je suis bien trop droit;\\
    ce n’est pas ta faute d’être courbe\\
    les lignes parallèles je n’aime pas ça.
  \end{verse}
  \begin{verse}
    Gueule du non-sens,\\
    poésie sur pattes:\\
    j’adore ta folie.
  \end{verse}
\PoemTitle{Totem (8 Novembre 2011)}
  \begin{verse}
    Il y a là tant de lagons\\
    que je veux bien me noyer\\
    et ma tête chargée comme un wagon\\
    de joie a déraillée.
  \end{verse}
  \begin{verse}
    Ils ne sont pas bleus, ni marrons\\
    je ne sais -- mais l’expression!\\
    Ces yeux sortis des tableaux\\
    suffisent à remettre sur flots.
  \end{verse}
  \begin{verse}
    Même si par delà il y a l’engeance\\
    affreuse et à venir de la distance,\\
    ici je suis quelqu’un\\
    -- je vis,\\
    ~~~~~~~~~~et je n’oublie rien.
  \end{verse}
\PoemTitle{Urgente balafre (2 Novembre 2011)}
  \begin{verse}
    La tentacule fut tendue\\
    et les poumons arrachés,\\
    la bramée s’est retenue,\\
    le sol s’est effondré.
  \end{verse}
  \begin{verse}
    Tout coule\footnote{Démocrite? À vérifier…}\\
    mais il n’y a d’eau que dans les fleuves où l’on se noie.
  \end{verse}
  \begin{verse}
    Immolation de pacotille et course à la montre.\\
    \textsc{Monsieur est à cramer.}
  \end{verse}
  \begin{verse}
    La lionne a pris le large pour son cirque particulier,\\
    faire des figures et rugir comme à la portée de tous.
  \end{verse}
\PoemTitle{Carminerie (30 Novembre 2011)}
  \begin{verse}
    Elle part en morceaux\\
    mes songes d’hiver.\\
    Certes, ce n’est pas un défaut\\
    que de me plaire
  \end{verse}
  \begin{verse}
    -- quoi que cela rapporte peu.\\
    Le silence n’est que métaphore;\\
    je suis le chien de mon petit jeu,\\
    la balle rebondit ainsi que des transports.
  \end{verse}
  \begin{verse}
    Une heure de vacarme incohérent\\
    vaut bien une seconde de silence rassasiant;\\
    oui, mais la joie, dans case?\\
    Tu peux gémir, tu ne l’auras pas d’occase.
  \end{verse}
\PoemTitle{Lac de La Source II (15 Septembre 2011 / Novembre 2011)}
  \begin{verse}
    Pas encore boulentonnerré que je songetombe\\
    fort à faire la foudre et le trou.\\
    Est-elle vipérine -- ou mieux,\\
    qu’elle nous jette des caillots de foie?
  \end{verse}
  \begin{verse}
    Grognards emplumés -- \textsc{À bas la lisière aqueuse!}\\
    ce que tu sens, ce que tu vois\\
    -- naturels molierants pulvérisants de grâce\\
    qui n’exige le glauque en trombe.

    Rien que sorte de sa toile sans glousser au fond\\
    et qui ne se berce d’onomatophores\\
    Brulants synapses \& varimaudeuses baffrant\\
    produisent des globophiles en ronde.
  \end{verse}
\PoemTitle{Yeux noir-bégonia (28 Novembre 2011)}
  \begin{verse}
    Morphée aux ailes d’encre\\
    mon ami de pacotille,\\
    tes bras détruisent tout\\
    mais ne donnent rien.
  \end{verse}
  \begin{verse}
    Guerre du nom d’être\\
    entrailles qu’il a fallu détendre\\
    trop de nœuds pour y voir clair.
  \end{verse}
\PoemTitle{Mémorial de la Flosse (Orléans, 5 Décembre 2011)}
  \begin{verse}
    Yeux Yeux Yeux, Yes, Yes, Yes.\\
    Le vide plein de grâce.\\
    Vive les livres.
  \end{verse}
  \begin{verse}
    Mélancolie, bruissement de la langue, carrefours d’audio, de stéréo polyphonique,
    verbe ici et là qui éclate dans le soir, fin de soirée emphatique du haut de toutes
    les désespérances.
  \end{verse}
  \begin{verse}
    J’adore les dentistes et l’orthopédie, et la merde, le caniveau, le temps de chien,
    les brumes où l’on y voit plus clair -- où je suis, et quoi? Arrêter? Non, non, non.
  \end{verse}
  \begin{verse}
    La Mort, en tout petit, {\tiny la mort}, on ne la voit plus; et qu’on s’en fout, de ce ver de
    terre en armes!\\
    -- moi je ne lis pas dans les crânes,\\
    dans les âmes vides\\
    mais j’adore tes yeux.
  \end{verse}
  \begin{verse}
    Lumières qui défilent et que nous sommes studieux, et que nous sommes immobiles, et la
    Loire qui pleure, et les têtes enfolâtrées, décorations de Noël à conchir, pourtant…
  \end{verse}
  \begin{verse}
    joie d’être là, joie du monde, oh la belle bleue, clair, échardes d’ennui retirées de mon
    crâne, tram,
  \end{verse}
  \begin{verse}
    Yeux Yeux Yeux, Yes, Yes, Yes,
  \end{verse}
  \begin{verse}
    la vie, enfin.
  \end{verse}
