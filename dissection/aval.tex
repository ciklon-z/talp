%	 *** Licence ***

% Cette œuvre est diffusée sous les termes de la license Creative Commons
% «~CC BY-NC-SA 3.0~», ce qui signifie que :

% Vous êtes libres :

%   * de reproduire, distribuer et communiquer cette création au public ;
%   * de modifier cette création.

% Selon les conditions suivantes :
    	
%  * Paternité - vous devez citer le nom de l'auteur original de la manière indiquée par l'auteur de l'œuvre ou le
%	         titulaire des droits qui vous confère cette autorisation (mais pas d'une manière qui suggérerait
%	         qu'ils vous soutiennent ou approuvent votre utilisation de l'œuvre).
%  * Pas d’utilisation commerciale - Vous n'avez pas le droit d'utiliser cette œuvre à des fins commerciales. 
%  * Partage des conditions initiales à l'identique - si vous transformez ou modifiez cette œuvre pour en créer une nouvelle,
% 						      vous devez la distribuer selon les termes du même contrat ou avec une
%					              licence similaire ou compatible.

% Comprenant bien que :

%  * Renoncement - N'importe quelle condition ci-dessus peut être retirée si vous avez l'autorisation du détenteur des droits.
%  * Domaine public - Là où l'œuvre ou un quelconque de ses éléments est dans le domaine public selon le droit applicable, ce statut
%		      n'est en aucune façon affecté par le contrat.
%  * Autres droits - d'aucune façon ne sont affectés par le contrat les droits suivants :
%    - Vos droits de distribution honnête ou d’usage honnête ou autres exceptions et limitations au droit d’auteur applicables;
%    - Les droits moraux de l'auteur;
%    - Les droits qu'autrui peut avoir soit sur l'œuvre elle-même soit sur la façon dont elle est utilisée, comme la publicité
%      ou les droits à la préservation de la vie privée.

% === Note ===

% Ceci est le résumé explicatif du Code Juridique; La version intégrale du contrat est consultable ici:
% <http://creativecommons.org/licenses/by-nc-sa/3.0/legalcode>.


\PoemTitle{Téloche à cloches (20 Novembre 2011)}
  \begin{verse}
    En partance pour nulle part,\\
    les vieux sont agresseurs;\\
    le long terme passe sur les mœurs\\
    -- rien de tel pour pleurer.

    Désintégration -- avales\\
    tes yeux avant de parler;\\
    le prolo ravale\\
    son envie de juger.
  \end{verse}
  \begin{verse}
    \textit{Breaking The Law},\\
    ou bien tout est faux --\\
    l’ennui passe\\
    -- c’est un ange moderne
  \end{verse}
  \begin{verse}
    Échéances de malheur,\\
    la joie dans quelle case elle va,\\
    acariens parlotteurs\\
    -- quel merdier à tout va !?
  \end{verse}
  \begin{verse}
    Soyons intègres -- tout ira bien,\\
    oui, restons fidèles aux chiens,\\
    les indigènes ont des dents trop grandes\\
    pour digérer nos offrandes.

    Quand est-ce que ça brulera,\\
    quand vomira-t-on tout\\
    toute cette bouillie de rancœur?
  \end{verse}
\PoemTitle{Handel (11 Novembre 2011)}
  \begin{verse}
    «~Achètes-toi un igloo~»\\
    «~La voilà, surchauffons tout!~»\\
    oh la surprise des extrémités sanguines;\\
    \textsc{Tout cela me rend dingue et me termine.}
  \end{verse}
  \begin{verse}
    J’ai décapité pour toi le raseur\\
    mais ses membres gigotent de partout,\\
    restons bien verts de peur\\
    -- n’osons pas avouer tout.
  \end{verse}
  \begin{verse}
    La perdre, volontiers peut-être\\
    mais plus que trembler ainsi\\
    j’aime être rassemblé\\
    ainsi que le prématuré Osiris.
  \end{verse}
  \begin{verse}
    Tant d’indécents retours\\
    à soi au lieu dit culminent.\\
    Nous nous sommes écartelés, au fond\\
    en autant d’écharpes multicolores.
  \end{verse}
  \begin{verse}
    Ingrédient spleenétique de la joie sans failles\\
    où se trouvent les riens qu’il faut\\
    et la folie qu’on exige.
  \end{verse}
\PoemTitle{Outrage de la virgule (14 Décembre 2011)}
  \begin{verse}
    J’aime ta mère et la baise aussi.
  \end{verse}
\PoemTitle{Termitière (5 Novembre 2011)}
  \begin{verse}
    Ils croyaient errer dans des couloirs interminables\\
    mais ils n’étaient qu’en cage;\\
    tout ce qu’ils gueulaient à la face des barreaux\\
    n’était rien qu’eux, que bourreaux.
  \end{verse}
  \begin{verse}
    \textsc{Oui}, je veux rire et aspirer\\
    ce que donnent les baumes aux auras\\
    mais ma parole sera tranchée\\
    par tout ce qui en moi sera
  \end{verse}
  \begin{verse}
    et ce que je peux dire, si cela a encore une efficacité\\
    atterrira en crêpe sur les planchers à cervelles,\\
    et les muscles que je ne saurais actionner\\
    seront la froide guerre de ma détente.
  \end{verse}
  \begin{verse}
    De gros et lisses becs me dévorent la tête,\\
    il faut hurler ou disparaitre,\\
    plonger de la planche pour ne plus rien être.
  \end{verse}
\PoemTitle{J’encule à sec La Rochefoucauld (30 Novembre 2011)}
  \begin{verse}
    Le Lundi n’évoque chez moi qu’un Samedi en attente;\\
    sans crier gare, je m’y retrouve\\
    poisson dans l’eau\\
  \end{verse}
  \begin{verse}
    Au reste, je hais la littérature.
  \end{verse}
  \begin{verse}
    Blaiseupascaliens, pariez si vous voulez du fond de vos culTEs,\\
    mais en orpailleurs, s’il vous plait;\\
    -- je salue votre richesse\\
    (c’est le geste approprié des voleurs).\\
  \end{verse}
  \begin{verse}
    On perd pied dans votre marécage;\\
    on le traverse, pourtant.\\
    \textsc{Il ne faut pas} respirer pour proférer\\
    «~travail~», «~activité~»
  \end{verse}
  \begin{verse}
    N’essayez pas de me faire aimer le billot.
  \end{verse}
\PoemTitle{Attributs de l’objet (30 Novembre 201)}
  \begin{quotation}
    Je t’enconcombronne la courge.
  \end{quotation}
  \begin{verse}
    Plus rien qui ne tombe\\
    sur l’un et l’autre plateau de la balance\\
    il faut une plume pour la cadence\\
    comme une arme pour mimer les défilés.
  \end{verse}
  \begin{verse}
    Merdre à la grammaire des têtes,\\
    foutaise de baiseur de père\\
    -- ou l’inverse, foutaise de même.
  \end{verse}
  \begin{verse}
    OSEF du genre \emph{animain},\\
    zoo de gorilles électroniques en cage,\\
    pétaradeurs de cieux dépressifs\\
    à colorer de feux d’artifices pour --\\
    quoi? Tenir.
  \end{verse}
  \begin{verse}
    Bouh, les petits porteurs de pannonceaux\\
    aussi cartonnés les uns que les autres;\\
    Ne mets pas bas au genre humain:\\
    ces papelards se déchirent d’une main.
  \end{verse}
\PoemTitle{Manigances Récursives Inévitables (13 Décembre 2011)}
  \begin{verse}
    L’horloge qui marche\\
    a des cils de fumée;\\
    elle azure tout du métal qui rouille.
  \end{verse}
  \begin{verse}
    Elle a des vieux os elliptiques\\
    et des bras agités proleptiques\\
    dans le cycle de ses propos\\
    tournent ses yeux droits et dispos.
  \end{verse}
  \begin{verse}
    À rebours pourtant, elle court\\
    contre les crabes furieux,\\
    cliquetant de pattes en becs\\
    aux yeux glauques d’espérances.
  \end{verse}
  \begin{verse}
    Mais elle ne parsème pas\\
    les plaines d’Avalon,\\
    ça non, il ne faut plus qu’elle sème\\
    le désespoir des bourgeons.
  \end{verse}
  \begin{verse}
    À dépasser ainsi même les dieux\\
    (auxquels je ne crois point),\\
    il semble que ses yeux\\
    bannissent tout contrepoint.
  \end{verse}
  \begin{verse}
    Et la peur et la mort qu’on cherche, qu’on désire même, et les
    monts à perchoirs statistiques,\\
    au delà de tout cela et de la fuite,\\
    l’on retrouve ce que l’on veut perdre et à quoi on renonce
    sans y croire une once,\\
    bris de mûres aux voies lactées bleues à supernovas de
    mortels.
  \end{verse}
  \begin{verse}
    Dans madame l’Horloge qui marche,\\
    oh qu’il n’y a rien,\\
    oh que des ombres
  \end{verse}
  \begin{verse}
    qu’humilient tous les regards essorés,\\
    Renoirs récursifs aux attributs folichons\\
    de Lady Nénu-Phare.
  \end{verse}
