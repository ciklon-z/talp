%	 *** Licence ***

% Cette œuvre est diffusée sous les termes de la license Creative Commons
% «~CC BY-NC-SA 3.0~», ce qui signifie que :

% Vous êtes libres :

%   * de reproduire, distribuer et communiquer cette création au public ;
%   * de modifier cette création.

% Selon les conditions suivantes :
    	
%  * Paternité - vous devez citer le nom de l'auteur original de la manière indiquée par l'auteur de l'œuvre ou le
%	         titulaire des droits qui vous confère cette autorisation (mais pas d'une manière qui suggérerait
%	         qu'ils vous soutiennent ou approuvent votre utilisation de l'œuvre).
%  * Pas d’utilisation commerciale - Vous n'avez pas le droit d'utiliser cette œuvre à des fins commerciales. 
%  * Partage des conditions initiales à l'identique - si vous transformez ou modifiez cette œuvre pour en créer une nouvelle,
% 						      vous devez la distribuer selon les termes du même contrat ou avec une
%					              licence similaire ou compatible.

% Comprenant bien que :

%  * Renoncement - N'importe quelle condition ci-dessus peut être retirée si vous avez l'autorisation du détenteur des droits.
%  * Domaine public - Là où l'œuvre ou un quelconque de ses éléments est dans le domaine public selon le droit applicable, ce statut
%		      n'est en aucune façon affecté par le contrat.
%  * Autres droits - d'aucune façon ne sont affectés par le contrat les droits suivants :
%    - Vos droits de distribution honnête ou d’usage honnête ou autres exceptions et limitations au droit d’auteur applicables;
%    - Les droits moraux de l'auteur;
%    - Les droits qu'autrui peut avoir soit sur l'œuvre elle-même soit sur la façon dont elle est utilisée, comme la publicité
%      ou les droits à la préservation de la vie privée.

% === Note ===

% Ceci est le résumé explicatif du Code Juridique; La version intégrale du contrat est consultable ici:
% <http://creativecommons.org/licenses/by-nc-sa/3.0/legalcode>.

\PoemTitle{Néolités (10 Novembre 2011)}
  \textsc{Lotcon}
  \begin{quote}
    Bonjour -- \textsc{Terme de salutation. Je vous souhaite le bonjour.}  -- je viens
    résilier le frikenplan de mon passappel.  Il me fait suésulfite quand on le
    moque au polanploi à cause de sa pachydagée.
  \end{quote}
  \textsc{Carlyle Itrée}
  \begin{quote}
    Vatendonc  l’client  lunettHELLO!  ’faut  bien  qu’soit  cavofoutu  ton
    tranteSEINSdice pou qu’on le  rironéLOLe. T’y aura donc glisser-déposer
    des lôleuquâtes pour pas cravateufaire!
  \end{quote}
  \textsc{Lotcon}
  \begin{quote}
    Qué’q ces idiofammations? Attisez pas des sottises -- \textsc{Paroles manquant de jugement} -- 
    ! Je m’en vais t’avocabatailler et t’en phagociteras des vertépamurs! Je veux un passappel 
    touchatou avec curation cidofaprovôle ou j’te vasemingue ta bicoque comme Ilonféhalatélée!
  \end{quote}
  \textsc{Carlyle Itrée}
  \begin{quote}
    Une acéparole de plus et je vous commissionne avec les grobraséku.
  \end{quote}
  \textsc{Lotcon}
  \begin{quote}
    Ah pécore, mais vous m’ironisez! Donnezenmoidonc un et je me boutevite jambes au cou…
  \end{quote}
  \textsc{Carlyle Itrée}
  \begin{quote}
    Bonbon, salez moi cette facture, emprunte et luséapprouvé zicizéla. Vous gagnez le dernier coquenplastoc
    deux-mille-onze-et-des-brouettes avec tout-plein-fonctions et le sucre de mon accrochoréplique.
    Satisfait-ou-remboursé? Au-revoir-monsieur-ce-fut-un-plaisir-monsieur-maintenant-dégages-monsieur.
  \end{quote}
  \textsc{Lotcon}
  \begin{quote}
    Que votre frikenplan  augmente sa rente et je vous  empoignerais un créveuil
    ou un percepo! Vous-aurez-été-prévenus!
  \end{quote}
\PoemTitle{Contenance du registre (27 Novembre 2011)}
  \begin{verse}
    Dans le tram à l’ennui omniprésent de limbes et toujours grelottant on chante\\
    ~~~~«~Tchutchotchutchou!~»\\
  \end{verse}
  \begin{verse}
    Un quidam foudroie à tour de bras\\
    et rend mêmes les vitres blasées\\
    ~~~~celle qui chantonne s’est tû glacée\\
    l’autre a le coin de l’œil satisfait\\
    ~~~~«~Ah, silence, que c’est bon d’être mort!~»
  \end{verse}
  \begin{verse}
    Dis-moi la couleur de la cécité\\
    -- ou sinon je te jette mais\\
    donnes moi de la poésie sans A(rt)\\
    et je m’enfuirais comme tu voudras.
  \end{verse}
\PoemTitle{Lover Man}
  \paragraph*{I (20 ou 21 Octobre 2011)}
  \begin{verse}
    Dictature de la dénomination.\\
    Prénom et Nom, petites armes primaires;\\
    survis la première année, on te nommera\\
    et désormais?
  \end{verse}
  \begin{verse}
    Comment se retrouver avec le nom de son ombre\\
    nous qui sommes tous les mêmes,\\
    comment nous trouver\\
    si les noms restent accrochés à nos figures?
  \end{verse}
  \paragraph*{II (Idem)}
  \begin{verse}
    Je vois l’humain du futur;\\
    il est déjà là…
  \end{verse}
  \begin{verse}
    Son nom n’est plus rien\\
    -- il en a trente, il en donne,\\
    sources infinies de liberté.
  \end{verse}
  \begin{verse}
    Comment le suivre\\
    et le faire mâcher démocratiquement\\
    la poussière donnée ordinairement?\\
    On ne le peut.
  \end{verse}
  \begin{verse}
    Passer sa vie à se chercher\\
    et puis une fois cela fait\\
    se retrouver dans les autres\\
    au gré des déceptions, des peines et des joies…
  \end{verse}
  \paragraph*{III (23 Novembre)}
  \begin{verse}
    Peut-être que tous en moi vous gigotez;\\
    entrelacés, lascifs de profil, ordinaires,\\
    musiques de tous les âges, mots inconnus…
  \end{verse}
  \begin{verse}
    De l’homme qui aime à celui pour qui encore une fois\\
    rien ne compte; et de l’auroch aux livres que j’ai brulé\\
    -- jeux de synapses, le monde est en moi.
  \end{verse}
  \begin{verse}
    Gravier pianoteurs et promoteurs à fous,\\
    question d’effets, se {\large gr}{\Large an}{\LARGE d}{\huge i}{\Huge r}
  \end{verse}
  \begin{center}
    {\Huge …}
  \end{center}
\PoemTitle{Le bon ordre des choses? (8 Décembre 2011)}
  \begin{verse}
    \textsc{ÉPITAPHE~DE~LA~LITTÉRATURE:}
  \end{verse}
  \begin{quotation}
    Elle a bien vécu cette salope!
  \end{quotation}
\PoemTitle{Nécessité de l’addition (27 Novembre 2011)}
  \begin{verse}
    Un joli couple réchauffe les cœurs\\
    (et d’autres choses aussi?);\\
    A chacun son âme sœur\\
    chaque bouche vaut une insomnie.
  \end{verse}
  \begin{verse}
    Doucement, l’œil amusé et taquin\\
    le premier être dit:\\
    «~Ceci est une censure~»\\
    ~~~~-- la réponse ne s’étouffe dans le bruit\\
    elle a recouvert la pulpe au teint carmin.
  \end{verse}
  \begin{verse}
    L’autre lui fait des nattes d’un air penaud\\
    -- car il faut le dire, n’en déplaise aux idiots,\\
    le bonheur est fille jolie\\
    et ce duo de dames l’a bien compris.
  \end{verse}
\PoemTitle{Éclatement de l’année (13 Décembre 2011)}
  \begin{verse}
    L’homme semble-t-il, est un aller-retour en toc\\
    poussière d’origine retournée au vieux brasier\\
    et l’amour et toi, je m’en moques;
  \end{verse}
  \begin{verse}
    La petite mélancolie\\
    du Décembre venté a dit\\
    «~Toc Toc,
  \end{verse}
  \begin{verse}
    ton corps est malade\\
    et l’esprit a pris la poudre\\
    -- elle a bonne odeur noire~»
  \end{verse}
\PoemTitle{Procrastination \& Fatalisme (17 Novembre 2011)}
  \begin{verse}
    Je crois que cette fois j’écris -- enfin!\\
    Non, rien de génial, remettons à demain.\\
    Il est l’heure du perpetuel retard\\
    -- tout sera fait au plus tard.
  \end{verse}
  \begin{verse}
    Mettre une rime là?\\
    Je ne crois pas…\\
    Ou je le fais sans le vouloir,\\
    toute action se fond au noir…
  \end{verse}
\PoemTitle{Tout ce qui grogne (27 Novembre 2011)}
  \begin{verse}
    C’est la nuit du loup-garou\\
    il faudra bien que j’écorche tout\\
    les peaux et les visages comme de l’écorce.
  \end{verse}
  \begin{verse}
    Que je prononce entre la chair rose\\
    et les os craquants le divorce\\
    -- je rendrais votre tête affreuse
  \end{verse}
  \begin{verse}
    et me baladerais en ville\\
    -- deux casques à porter, la belle vie!
  \end{verse}
\PoemTitle{PPPE (13 Décembre 2011)}
  \begin{verse}
    Attendons l’heure en grelottant\\
    -- oh que de bla bla à venir --\\
    et la vieille panique du singe policé.
  \end{verse}
  \begin{verse}
    Défendre ce qu’on peut\\
    mentir à tous les vents\\
    à ces souffles asthmatiques qui se croient ouragans\\
    et aussi à \textsc{Monsieur Moi}:
  \end{verse}
  \begin{center}
    «~Tout va bien se passer.~»
  \end{center}
\PoemTitle{La statuaire de l’examen (30 Novembre 2011)}
  \begin{verse}
    La ville quand le matin tient encore de la nuit \\
    -- ou bien c’est la nuit qui tient encore le matin dans ses bras, je ne sais plus,\\
    avec ses taches jaunâtres, vertes, des avenues de lumières jusqu’au gâchis de la cathédrale,\\
    je m’en fous.
  \end{verse}
  \begin{verse}
    Les sirènes qui font fuir,\\
    les Cronos, tramways pour abreuvés du chrono,\\
    les voix monotones au ton drôle pré-enregistré,\\
    les tableaux et les plans où se repérer
  \end{verse}
  \begin{center}
    peut-être,
  \end{center}
  \begin{verse}
    Tout cela m’est indifférent;\\
    il ne pleuviote qu’à peine dans les cages;\\
    tout tourne, perfection grise;\\
    \textsc{Je suis un poumon} -- sans-plomb-95.\\
  \end{verse}
