%	 *** Licence ***
% Cette œuvre est diffusée sous les termes de la license Creative Commons
% «~CC BY-NC-SA 3.0~», ce qui signifie que :

% Vous êtes libres :

%   * de reproduire, distribuer et communiquer cette création au public ;
%   * de modifier cette création.

% Selon les conditions suivantes :
    	
%  * Paternité - vous devez citer le nom de l'auteur original de la manière indiquée par l'auteur de l'œuvre ou le
%	         titulaire des droits qui vous confère cette autorisation (mais pas d'une manière qui suggérerait
%	         qu'ils vous soutiennent ou approuvent votre utilisation de l'œuvre).
%  * Pas d’utilisation commerciale - Vous n'avez pas le droit d'utiliser cette œuvre à des fins commerciales. 
%  * Partage des conditions initiales à l'identique - si vous transformez ou modifiez cette œuvre pour en créer une nouvelle,
% 						      vous devez la distribuer selon les termes du même contrat ou avec une
%					              licence similaire ou compatible.

% Comprenant bien que :

%  * Renoncement - N'importe quelle condition ci-dessus peut être retirée si vous avez l'autorisation du détenteur des droits.
%  * Domaine public - Là où l'œuvre ou un quelconque de ses éléments est dans le domaine public selon le droit applicable, ce statut
%		      n'est en aucune façon affecté par le contrat.
%  * Autres droits - d'aucune façon ne sont affectés par le contrat les droits suivants :
%    - Vos droits de distribution honnête ou d’usage honnête ou autres exceptions et limitations au droit d’auteur applicables;
%    - Les droits moraux de l'auteur;
%    - Droits qu'autrui peut avoir soit sur l'œuvre elle-même soit sur la façon dont elle est utilisée, comme la publicité
%      ou les droits à la préservation de la vie privée.

% === Note ===

% Ceci est le résumé explicatif du Code Juridique; La version intégrale du contrat est consultable ici:
% <http://creativecommons.org/licenses/by-nc-sa/3.0/legalcode>.

\PoemTitle{L’Interim et les jours (15 Août 2012)}
  \begin{verse}
    Dans le dictionnaire j’ai lu sol\\
    sol do mi, de la musique en profusion\\
    ma charrue bleue l’a dans l’oi\\
    nous avons le verbe en perfusion.
  \end{verse}
  \begin{verse}
    Le ver pile quand le vert se meurt:\\
    il faut le dire à la face, des vers,\\
    enfin! Le mot croissance s’est ébrêché\\
    brisons le avant les fossés.
  \end{verse}
  \begin{verse}
    Le sol je l’ai dans l’heure\\
    moins que les putes salées des mers\\
    tout tournera au faussé --\\
    c’est tout ce que laisseront les milicés.
  \end{verse}

\PoemTitle{Introduction à la catapulte (7 Juin et 2 Août 2012)}
  Tu rêves ailleurs pis que dans  le crâne d’ailleurs des longs morceaux de
  verre  écartelés en  ronde  jusqu’en  claquer comme  une  corde de  piano
  décérébrée  qui  aurait bu  une  goutte  de trop  ou  rien  de moins  que
  l’enfer, du  magma puant, bon  grog chaud  et dégoulinant de  sa mâchoire
  double coule  des insectes  frétillants, excitant le  vert d’eau,  le vert
  pomme, le ver  dans le fruit et le  ver mine qui broie le sable  de sa gueule
  fauve qui meule orangée ténèbres \hulu{Ténèbres} nulle part où aller

  collision des sens  le monde puis en pleine \hulu{Face}  des vestiges de mots
  et de syllabes,  du passé il ne  reste que les odeurs, qu’on  fixe dans la
  chair, les écrits  s’envolent, les paroles restent et dans  ma tanière de
  lotophage à  la lumière défenestrée à  force de se tamiser  en saignées
  multiples il coule  du \hulu{BLEU} et du  rouge qui colle à mes  pieds et le
  tapis aux figures animales s’anime le corbeau s’envole,

  la jungle sort entière des cordages pour rugir à la curée comme un orateur
  des  pets  planétaires ou  un  poète  ou  un \hulu{Ca}fard  choses  égales
  chaussettes  grises ou  noires  il faut  choisir avant  la  fin du  spectacle
  de  la  représentation non  car  seule  va son  chemin  de  grand enfant  la
  présen\hulu{Ta}tion du cosmos

  une génération  de plus et le  boulier n’a plus sa  mémoire pour chanter
  les  histoires de  celles qu’il  a aimé  et dont  il ne  se souvient  plus
  c’était hier  c’était dans le gris,  quelque part entre ma  tête et la
  tienne, un ange au crâne incrusté de verre qui babafouillait.

\PoemTitle{Carte verte (15 Août 2012)}
  \begin{verse}
    Vous pouvez nous dire minables\\
    on n’est pas aussi odieux que vous,\\
    on ira encore aux restos de la valve\\
    ’comme ça on serait des parasites -- et vous?
  \end{verse}
  \begin{verse}
    Malfrats, cad votards, banksters\\
    c’est le mur injurié par le lierre\\
    vitupérez tant que nous sommes maitrisés:\\
    un mur, ça porte de quoi lapider;
  \end{verse}
  \begin{verse}
    on ne se rase pas bien\\
    on nous croit jetables\\
    des lames resteront dans les coins\\
    pour neutraliser vos têtes notables.
  \end{verse}

\PoemTitle{Héritage (27 Août 2012)}
  \begin{verse}
    Le bal des trouées\\
    s’était une fois de plus\\
    recomposé
  \end{verse}
  \begin{verse}
    L’air infecté de pus,\\
    les membres pourris et viciés\\
    de la vieille assemblée
  \end{verse}
  \begin{verse}
    dansaient en symcope\\
    le bal des trouées --\\
    le passé terrible invoquait en chacun\\
    la part d’animal enfin
    recomposée
  \end{verse}
  \begin{verse}
    Deuxième acte\\
    Entre la Colère -- it’s a fact,\\
    vêtue grisâtre\\
    insecte rampant\\
    elle ouvre toujours ses conciliabules\\
    en claquant des mandibules
  \end{verse}
  \begin{verse}
    la verte morte\\
    éviscératrice adrénaline\\
    la chair forte\\
    \hulu{Qui aujourd’hui est à égorger?}
  \end{verse}
  \begin{verse}
    (~~Sinon l’humanité qui, entière et folle,\\
    ~~~regarde les ataviques…~~)
  \end{verse}
  \begin{verse}
    Et combien de veines fleuves\\
    à percer à boire à jeter encore?\\
    Le bal des trouées\\
    applaudit, enchanté.
  \end{verse}
  \begin{verse}
    Troisième acte\\
    Poussières jaunes et choléra fabuleux\\
    \hulu{À répandre à chaque repas}\\
    \hulu{Éviter le contact avec les yeux}.
  \end{verse}
  \begin{verse}
    Et nous vivrons dans la brume sanguine\\
    des étés longs sous la pourriture fine\\
    Avec gasoils, métaux aériens\\
    Avec odeurs âcres, de béton, de freins\\
    brûlés, avec la décrépitude enfin\\
    du romantisme de scalpel à chaque main.
  \end{verse}
  \begin{verse}
    Les spectateurs avides, assoiffés\\
    du bal des trouées\\
    iront voir, s’ils ont applaudi\\
    si la peste s’est répandue\\
    ainsi que nous l’avions dit\\
    ou bien qu’ils crient «~Vendus!~»\\
    le rictus mauvais et familier\\
    de ceux qui haissent bouche déformée.
  \end{verse}
  \begin{center}
    Rideau.
  \end{center}

\PoemTitle{Synonyme d’éternité (15 Août 2012)}
  \begin{verse}
    Tombant au fond du rosé\\
    si petit entre les tables des cafés\\
    ce papillon là volète par sa caverne\\
    hydratée de l’ambre brune.
  \end{verse}
  \begin{verse}
    Rues enfumées -- here come the grace\\
    des pavés modelés pile comme une face\\
    une pièce tinte jaunâtrement\\
    électrise des neurones dans un déferlement\\
    dans un coin ça s’enlace\\
    -- tout pour déplaire à l’Obscur de Grèce
  \end{verse}
  \begin{verse}
    Mais la graisse était dans la bière\\
    et regardait quand\footnote{Oui, c’est de la parodie.}\\
    elle pourrait gâcher les entrelacs de jambes\\
    -- abgreger ces fausses iambes.
  \end{verse}

\PoemTitle{L’herpes humaine (28 Août 2012)}
  \begin{verse}
    Un rire blanc\\
    me manque terriblement\\
    une bonne ambiance en enfer\\
    ça m’est sans doute \emph{nécessaire}.
  \end{verse}
  \begin{verse}
    Plus d’aventure dans ce monde\\
    Il faudrait l’anéantir\\
    dans un blanc éclat de rire\\
    ça, ce serait une belle aventure.
  \end{verse}
  \begin{verse}
    Mais ouvre les yeux\\
    quoi sinon de la merde dans les mains\\
    Pardon! Je parlais du bulletin\\
    allez prenez-le sur le tard\\
    votard, l’herpès humain
  \end{verse}
  \begin{verse}
    L’obeissance comme laisser-aller,\\
    chantre des civilités\\
    de laisser-crever:\\
    parole d’innomable\footnote{Cf. \emph{L’innomable} de \noun{Beckett}}
  \end{verse}

\PoemTitle{Ce qui reste (3 Septembre 2012)}
  \begin{verse}
    J’ai perdu ma vie à la page amour\\
    un jour si noir qu’il pleuvait des membres\\
    on courait sous les trombes\\
    écharper le premier venu\\
    la poésie j’ai peur de la poésie\\
    cette salope, immonde trangenre\\
    qui n’a pas de cœur ni de corps\\
    cette barrière inutile les mots?\\
    Des bouchers vendant ma viande\\
    mes organes, ma cervelle au plus offrant,\\
    des monstruosités sans égale\\
    oh je hais les céhodés!
  \end{verse}
  \begin{verse}
    Pourtant je crois qu’il n’y ni ciel\\
    ni espoir ni joie ni douleur\\
    que seule que je crée comme ça,\\
    d’un coup d’un seul,\\
    antéposée névrose --\\
    mais c’est un mensonge,
  \end{verse}
  \begin{verse}
    le ciel est de la couleur que tu souhaites\\
    et l’espoir, et la douleur, et la joie adorée\\
    sont tout à toi à ta palette RVB\\
    la faille ne cesse de grogner\\
    la bouche rocailleuse antimusicale\\
    cad, en rimes bien ordonnées\footnote{Troll.}\\
    égratigne et me conjure de troller\footnote{on vous l’avait dit.}
  \end{verse}
  %FIXME accent
  \begin{verse}
    où est ma page, où est ma page,\\
    dis, où est la vie?\\
    elle croit nous croissons\\
    il croit, le légume à mettre en terre\\
    dormir ça serait bien\\
    comme la mouche écrasée\\
    je n’en sais rien\\
    avant cela, il faudrait chanter
  \end{verse}
  \begin{verse}
    chanter quoi sous le torrent d’omoplates\\
    déglinguées à la fraiseuse métaphysique\\
    une unité, deux unités de blabla ironique\\
    mes excuses peut-être les plus plates\\
    avant que chante la courge dégénérée\\
    oh tuez la grammaire des inaccomplis\\
    une tête à trancher cela devrait se faire
  \end{verse}
  \begin{verse}
    Où est-elle, ma page, je l’ai perdue\\
    ma page vide adoré\\
    pire qu’un téléfilm dialysé\\
    il faut fuir en blondasse --\\
    le pont va s’écrouler --\\
    le séide ne saurait y réchapper\footnote{Référence aux Tortues Ninja. Oui, oui.} --\\
    la vie peut-être serait plus belle -- scenarisée\footnote{On dirait un sujet de dissertation de philosophie…}
  \end{verse}
  \begin{verse}
    Page noire d’encre, page blanche\\
    brouillon, pinaillages de vocabulaire;\footnote{Troll}\\
    Langage, aies pitié de mes orteils\\
    d’un pathétique avorté\footnote{Vraiment?}\\
    car je ne code qu’avec eux\\
    le seul programme qui vaille la peine d’être executé\\
    et qui pourtant en vie te laisserait\\
    te laisserait en RAD\\
    une expression, des parenthèses fermées,\footnote{Le langage avec des millions de parenthèses, ce pourrait être une définition de la poésie}\\
    des paramètres peut-être, mais oui:\\
    une seule voie, une seule sortie.
  \end{verse}
