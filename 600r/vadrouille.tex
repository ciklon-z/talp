%	 *** Licence ***
% Cette œuvre est diffusée sous les termes de la license Creative Commons
% «~CC BY-NC-SA 3.0~», ce qui signifie que :

% Vous êtes libres :

%   * de reproduire, distribuer et communiquer cette création au public ;
%   * de modifier cette création.

% Selon les conditions suivantes :
    	
%  * Paternité - vous devez citer le nom de l'auteur original de la manière indiquée par l'auteur de l'œuvre ou le
%	         titulaire des droits qui vous confère cette autorisation (mais pas d'une manière qui suggérerait
%	         qu'ils vous soutiennent ou approuvent votre utilisation de l'œuvre).
%  * Pas d’utilisation commerciale - Vous n'avez pas le droit d'utiliser cette œuvre à des fins commerciales. 
%  * Partage des conditions initiales à l'identique - si vous transformez ou modifiez cette œuvre pour en créer une nouvelle,
% 						      vous devez la distribuer selon les termes du même contrat ou avec une
%					              licence similaire ou compatible.

% Comprenant bien que :

%  * Renoncement - N'importe quelle condition ci-dessus peut être retirée si vous avez l'autorisation du détenteur des droits.
%  * Domaine public - Là où l'œuvre ou un quelconque de ses éléments est dans le domaine public selon le droit applicable, ce statut
%		      n'est en aucune façon affecté par le contrat.
%  * Autres droits - d'aucune façon ne sont affectés par le contrat les droits suivants :
%    - Vos droits de distribution honnête ou d’usage honnête ou autres exceptions et limitations au droit d’auteur applicables;
%    - Les droits moraux de l'auteur;
%    - Droits qu'autrui peut avoir soit sur l'œuvre elle-même soit sur la façon dont elle est utilisée, comme la publicité
%      ou les droits à la préservation de la vie privée.

% === Note ===

% Ceci est le résumé explicatif du Code Juridique; La version intégrale du contrat est consultable ici:
% <http://creativecommons.org/licenses/by-nc-sa/3.0/legalcode>.

\PoemTitle{«~The Word Beyond~» (19 Juillet 2012)}
  \begin{verse}
    Qu’elles sont délicieuses mes rotules\\
    celles que j’offre à ta bouche\\
    si elle n’a pas eu assez\\
    de son phosphore acidulé\\
    à la senteur mandarine,
  \end{verse}
  \begin{verse}
    les rotules de devant le paravent\\
    l’écran cachant sans lueur --\\
    le cimetière de mes littéraires horreurs,
  \end{verse}
  \begin{verse}
    ce charnier là, voudrais-tu l’aimer\\
    si empli de bras qu’il est\\
    englouti de brassées\\
    de discours asphmatiquement -- efficacement\\
    ampoulés?
  \end{verse}
  \begin{verse}
    Le grincement des rotules\\
    de derrière la scène\\
    a réveillé le kraken -- \hulu{Release the}\\
    -- tout doux, au fond des yeux\\
    on voit la dent de l’oblitéré.
  \end{verse}

\PoemTitle{Zeus, mon cul (7 Juillet 2012)}
  \begin{verse}
    Il est -- Prométhée incendiaire;\\
    mais que de volé? Rien. Il a repris son dû,\\
    dérobé par un vieil olubriu\\
    un m’as-tu-vu qui se disait son père.
  \end{verse}
  \begin{verse}
    Ce qu’il faut entendre ni dire\\
    hurler pour crever l’amer des ténèbres,
  \end{verse}
  \begin{verse}
    \hulu{MERDE} à la tendresse\\
    qui tue le vrai --\\
    les vitres fûrent brisées,\\
    dévoilant la pourriture:\\
    pas de temps pour négocier,
  \end{verse}
  \begin{verse}
    \hulu{Apportes, apportes, Prométhée}
  \end{verse}
  \begin{verse}
    Voyez comme il lèche l’animal\\
    ce feu: il saisit comme la typhoïde\\
    -- le titan pointe le tyran minable:\\
    qui voudrait de ce débris mongoloïde?
  \end{verse}

\newpage
\PoemTitle{Tout est trop droit (26 Juin 2012)}
  \begin{verse}
    De la mer? Seulement de l’écume?\\
    Plutôt les mots qui fermentent\\
    -- ils baisent -- leurs bouches qui s’y plantent\\
    j’y vois les cadavres que j’hume
  \end{verse}
  \begin{verse}
    Matérialité oh foutoir organique\\
    «~par ici~» cogne la chanson souveraine\\
    -- ça sent l’orage et le cafard Naheum
  \end{verse}
  \begin{verse}
    il rampe des écoutilles métalliques\\
    des Word Ness préhistoriques\\
    voudras-tu de ces tiques jeunes?
  \end{verse}

\PoemTitle{2012 Guy (28 Juin 2012)}
  \begin{verse}
    Nom: Je ne sais pas\\
    Nationalité: Antédiluvien
  \end{verse}
  \begin{verse}
    Occupations: Combats, larmes, rire --\\
    débordement de tous les cadres\\
    d’un point A à un point B\\
    (je n’aime que le chemin),
  \end{verse}
  \begin{verse}
    je construis des miroirs\\
    -- des éclats de ciel géants\\
    pas l’art pour l’art!\\
    Je veux pouvoir m’y regarder\\
    devant comme à travers.
  \end{verse}
  \begin{verse}
    Tout nait d’un mot\\
    -- noyau seul: radicalité;\\
    sinon, on a la maladie dans la gorge.
  \end{verse}
  \begin{verse}
    Enlevez moi la vie il reste les siècles\\
    et une guerre en trop\\
    et de l’autre bord --\\
    idem, la frénésie du monde.
  \end{verse}

\PoemTitle{Hérédité (25 Juin 2012)}
  \begin{verse}
    À force de lire\\
    je tousse la littérature --\\
    petit chantre de la léprosité.
  \end{verse}
  \begin{verse}
    Eh quoi! M’en voudrez-vous de cracher\\
    du sang aussi noir que de l’encre?\\
    Poisson des marinas où rien ne s’ancre\\
    je suis un crayon qui a la diarhée\\
  \end{verse}
  \begin{verse}
    \hulu{Coucou la Mort}
  \end{verse}
  \begin{verse}
    Tu n’as pas l’odeur que je préfère\\
    ni l’ingéniosité du teinturier\\
    au delà des cuves la sphère\\
    bulle bleue ira tout réconcilier.
  \end{verse}

\PoemTitle{Parasite? (2 Août 2012)}
  \begin{verse}
    D’abord les bras, les jambes tombent --\\
    on a la gangrène de l’estomac\\
    ça passe pas bien les paupières ouvertes
  \end{verse}
  \begin{verse}
    je suis la faim planté\\
    le rivet dans le crâne\\
    c’est la fin j’ai planté\\
    la syntaxe au pilori
  \end{verse}
  \begin{verse}
    eréirra ne tneiver\\
    le bruit des marées de chair
  \end{verse}
  \begin{verse}
    plus de jambes\\
    faim faim faim
  \end{verse}
  \begin{verse}
    serttel ed xulfer
    \hulu{RATÉ!}
  \end{verse}
  \begin{verse}
    Boire la tasse à divas\\
    je crois que c’est ailleurs\\
    que tout se joue -- je divague, \hulu{MERDE}
  \end{verse}
  \begin{verse}
    \emph{Ici} ne sneiver\\
    avant la fin \hulu{MERDE}\\
    faim faim faim.
  \end{verse}

\newpage
\PoemTitle{D.I.Y Brouhaha (2 Août 2012)}
  \begin{flushleft}
    \underline{Vous composerez quelque chose à partir de cet assortiment:}
  \end{flushleft}
  \begin{quote}
    Motos qui pétaradent; Jean Charles; le tram tue le commerce; Bâtard; Enculé;
    Ploucs; Dans trois jours; Réunion; Place du Martroi; Marre-toi; Des verts;
    Je préfère les bleus; Un café s’il vous plait; Au Chiendent; Parle moi un peu;
    De cette nouvelle copine; Trois plombes dans les chiottes; Un colon je te dis;
    Mets-toi devant la statue; le DADA; Vous devez rendre votre exemplaire;
    Odeur de pisse; Odeur de presse; Tout de suite les titres de notre édition;
    Tant pis pour demain; Sortie de crise;
  \end{quote}
  \begin{flushleft}
    \underline{Votre composition:}
  \end{flushleft}

  \begin{verse}
    «~ {\Huge BÂ} tard! \rotatebox{90}{Marre} toi des \rotatebox{90}{verts}\\
    je préfère les motos bleues qui {\Huge PÉT}aradent~»\\
    «~- {\huge Révolution!} -Vous devez rendre votre exemplaire dans {\large 3} jours~»\\
    «~\textsc{Réunion Place du Martroi}; Mets-toi devant la statue…~»\\
    «~-- le DaDa, quoi, de cette nouvelle copine~»\\
    «~Parles moi un peu de cette odeur de presse~»
  \end{verse}

  \begin{center}
    \rotatebox{-90}{{\Huge \textbf{«}}}
  \end{center}
  \begin{center}
    \textsc{Tout de suite les titres de notre édition}\\
  \end{center}
  \begin{quote}
    \textmusicalnote{} Odeur de pisse chez les ploucs \textmusicalnote{} le tram
    tue le commerce et avec nous, Jean Charles enculé
  \end{quote}
  \begin{center}
    \rotatebox{90}{{\Huge \textbf{«}}}
  \end{center}

  \begin{verse}
    «~Un ca{\Large FÉE}, s’il te plait, au \fbox{Chiendent}\\
    c’est sympa~»
  \end{verse}
  \begin{verse}
    «~Après trois plombes dans les chiottes il est -- \textsc{SORTIE}\\
    \textsc{DE~ CRISE~ CHEZ~ AMESYS} -- Ah, mais si, un vrai colon, j’te dis!~»
  \end{verse}

\newpage
\PoemTitle{: Clous de charpentacle : (26 Juillet 2012)}
  \begin{verse}
    {\Huge \textbf{«}} Astèr\\
    miracle voilà\\
    des mots trompettistes\\
    c’est la joie
  \end{verse}
  \begin{verse}
    Astèr\\
    complainte je ne sais pas\\
    une roue sur la piste\\
    deux trois tours et puis s’en va
  \end{verse}
  \begin{verse}
    Astèr\\
    minute peinture gouache\\
    tourbillons d’éther ça flashe\\
    yeux zézoreilles et gorilles femelles\\
    ô que non que non, guenon --\\
    fais des tours alors qu’ils brassent de l’air\\
    ces cons {\Huge \textbf{»}}
  \end{verse}

\PoemTitle{Sporulation philosophique (19 Juillet 2012)}
  \begin{verse}
    Des idées pétulances\\
    j’en ai autant que des couleurs à ma portée\\
    celles des livres\\
    qui ont si bien engendré
  \end{verse}
  \begin{verse}
    «~thousand suns~»\\
    elles serpentent c’est darwinien,\\
    je l’affirme elles évoluent pour étouffer\\
    l’amante et la religieuse
  \end{verse}
  \begin{verse}
    Ô les plantes carnivores\\
    cracheuses de bites\footnote{Orthographe malicieuse. Mais c’est le mot anglais \textit{bit} qu’il faut entendre.} et de feu\\
    les idées coulent l’USR Matrice\\
    réjouissant d’autres torts pilleurs.
  \end{verse}

\PoemTitle{dico renversé (26 Juin 2012)}
  \begin{verse}
    Le ciel est clair avec une autre dose\\
    la vis s’enfonce dans le poitrail;\\
    ça fait mal, c’est morose\\
    comme du travail
  \end{verse}
  \begin{verse}
    Qu’on puisse dire ce que l’on veut\\
    d’un air pleureur d’indigné\\
    la révolte passe mieux\\
    d’avance les deux mains clouées.
  \end{verse}
  \begin{verse}
    Ce moi terrible qui désire\\
    dont la bonne foi creuse le fossé\\
    ne pourra un jour plus suivre\\
    sa doctrine de financier --
  \end{verse}
  \begin{verse}
    Éclats de merveilles en papier mâché\\
    à la télévision -- Juin, de nouveau l’été\\
    les pompes du député brillent\\
    -- on va pouvoir bouffer du bachelier.
  \end{verse}

\PoemTitle{( couic! ) (2 Août 2012)}
  J’aime la poésie mais pas les poètes. Un poème est une parenthèse sur
  un cadavre. Comme avec le corps humain, c’est le derrière de l’œuvre
  qui m’intéresse -- question de pudeur. Poésie, sacre du monde,
  misère de la personne.

  Quand le poète parle, l’humain se tait -- il faut qu’un \hulu{huluberlu}
  hurle pour rendre justice au vers, à la ligne -- infinie dévoration de la
  littérature: je suis papier, Papier me charcute avant que je sois papi.

  Passé la parenthèse, passé le jet de sang, il ne reste que l’amer et la
  Dive Bouteille pour noyer le reste.
