%	 *** Licence ***
% Cette œuvre est diffusée sous les termes de la license Creative Commons
% «~CC BY-NC-SA 3.0~», ce qui signifie que :

% Vous êtes libres :

%   * de reproduire, distribuer et communiquer cette création au public ;
%   * de modifier cette création.

% Selon les conditions suivantes :
    	
%  * Paternité - vous devez citer le nom de l'auteur original de la manière indiquée par l'auteur de l'œuvre ou le
%	         titulaire des droits qui vous confère cette autorisation (mais pas d'une manière qui suggérerait
%	         qu'ils vous soutiennent ou approuvent votre utilisation de l'œuvre).
%  * Pas d’utilisation commerciale - Vous n'avez pas le droit d'utiliser cette œuvre à des fins commerciales. 
%  * Partage des conditions initiales à l'identique - si vous transformez ou modifiez cette œuvre pour en créer une nouvelle,
% 						      vous devez la distribuer selon les termes du même contrat ou avec une
%					              licence similaire ou compatible.

% Comprenant bien que :

%  * Renoncement - N'importe quelle condition ci-dessus peut être retirée si vous avez l'autorisation du détenteur des droits.
%  * Domaine public - Là où l'œuvre ou un quelconque de ses éléments est dans le domaine public selon le droit applicable, ce statut
%		      n'est en aucune façon affecté par le contrat.
%  * Autres droits - d'aucune façon ne sont affectés par le contrat les droits suivants :
%    - Vos droits de distribution honnête ou d’usage honnête ou autres exceptions et limitations au droit d’auteur applicables;
%    - Les droits moraux de l'auteur;
%    - Droits qu'autrui peut avoir soit sur l'œuvre elle-même soit sur la façon dont elle est utilisée, comme la publicité
%      ou les droits à la préservation de la vie privée.

% === Note ===

% Ceci est le résumé explicatif du Code Juridique; La version intégrale du contrat est consultable ici:
% <http://creativecommons.org/licenses/by-nc-sa/3.0/legalcode>.

\PoemTitle{Cap(s) Lock (16 Août 2012)}
  QUAND \footnote{Pour traduire ce poème, on choisira une grammaire aussi barbante que l’originale.}
  PLUSIEURS~ SUJETS~ AU~ SINGULIER (TROISIÈME PERSONNE) SONT COORDONNÉS PAR LES
  CONJONCTIONS OU, NI, LE VERBE SE MET AU PLURIEL SI L’IDÉE DE DE CONJONCTION DE
  L’ENSEMBLE L’EMPORTE ET AU SINGULIER SI LA DISJONCTION EXCLUSIVE
  et l’infinie douceur d’être vivant
  (OU L’OPPOSITION) ENTRE LES SUJETS S’IMPOSE: LA PEUR OU LA MISÈRE ONT FAIT
  COMMETTRE BIEN DES CHOSES DONT L’ACADÉMIE.
  \footnote{Tiré de \emph{Grammaire Méthodique du français} par \textsc{RIEGEL}, \textsc{PELLAT}, \textsc{RIOUL}, PUF, 2011, p1152.}

\PoemTitle{Ticquet dernier (24 Août 2012)}
  \begin{center}
    \textit{Écrit sur un ticquet de tram de 8,5 x 5,5 centimètres.}
  \end{center}
  
  La dernière œuvre de Truck est sortie. Non, vraiment, la dernière. Certes
  il est encore jeune le Truck qui a fréquenté les mots comme une monture,
  il y a encore du souffle là-dedans, mais c’est la dernière œuvre -- après
  ça, il retournera à l’éternel silence.

  À sa quiètude de parasite.

  Cherchera-t-il encore les phrases et le reste? Oui, certes oui.

  Mais c’est la dernière œuvre de Truck, sortie pour dix euros; elle exhale
  l’encre, le papier cuit comme un corps en fin de vie, envahi par la
  respiration.

\PoemTitle{I/O Error (23 Août 2012)}
  \begin{verse}
    Demain je me draperais de rouge\\
    et je vendrai mon chariot\\
    pour courir dans les champs noirs\\
    des amis à venir --
  \end{verse}
  \begin{verse}
    Ô divine pression artérielle\\
    je m’endormirai dans un lacis d’ordinaire\\
    coulant parmi les Fleuves\\
    Ô divine pression artérielle
  \end{verse}
  \begin{verse}
    Un lourd chenal achemine\\
    des ombres rouges et muettes\\
    vers où reste la lumière?
  \end{verse}
  \begin{verse}
    Ce monde clos\\
    Ce goulet d’étranglement.
  \end{verse}

\newpage
\PoemTitle{259 RDB (28 Août 2012)}
  \begin{verse}
    Bris de verre\\
    aux creux ruisselle la boue\\
    Coule l’alcool et la puanteur\\
    Enjôlante \& les saouls\\
    Géométries insolubles\\
    pavés, pavés dans la ruelle dévorée\\
    qu’est-ce que ça veut dire?\\
    Danser, peut-être, mais avec qui?\\
    Qui n’a pas pris une ride d’adulte\\
    du creux ruisselle la nuit\\
    s’étirant en longueur\\
    dépérissant dans les artères\\
    des poumons pourris par l’algèbre.
  \end{verse}
  \begin{verse}
    Qu’ont-ils à s’agiter\\
    imbéciles furieux\\
    Elle est mignonne ma nuit\\
    elle et ses petites lueurs tristes\\
    elle me donne la nausée\\
    une hémorrhagie souriante\\
    un défi de condamné, allez savoir
  \end{verse}
  \begin{verse}
    -- tout n’est peut-être rien\\
    déjà parti\\
    Coucou l’ami\\
    Dostoievski.
  \end{verse}

\PoemTitle{Eumoniste, sors de ce corps! (28 Août 2012)}
  \begin{verse}
    Entrelacs lisses\\
    du monde au balcon --\\
    j’ai mangé le pigmée\\
    brûlé le totem\\
    au nom de l’Élysée\\
    Brûlez infects repaires de chair
  \end{verse}
  \begin{verse}
    Îles flottantes\\
    île de chair enflammée\\
    la nuit a des yeux qui me hantent\\
    mais «~le poète n’est pas dupe~»\\
    ça non, tout se dévisse\\
    part en sucette et chûte
  \end{verse}
  \begin{verse}
    Percussions soigneuses\\
    aller de l’avant\\
    plus encore\\
    (c’est un memento)\\
    conchier l’avent
    plus encore
  \end{verse}
  \begin{verse}
    Les mythes au kérosène!\\
    Ça n’intéresse plus personne.
  \end{verse}
  \begin{verse}
    Orléans, crèves\\
    de la peste, dans ta merde\\
    dans mon dégoût dans n’importe quoi
  \end{verse}
  \begin{verse}
    Les saintes mourant dès l’évangile,\\
    il faudrait les manger pour ne pas perdre la viande;\\
    Orléans, je rirais quand ta schizo’ pourriture\\
    me donnera à manger\\
    et de l’huile pour mes fritures --\\
    tous tes habitants pour un anglais!
  \end{verse}

\PoemTitle{L’Atlantéen ( 2012)}
  \begin{quote}
    «~
    We have lingered in the chambers of the see\\
    By sea-girls wreathed with seaweed red and brown\\
    Till human voices wake us, and we drown
    ~»\footnote{
    \begin{verse}
    Nous avons erré dans les chambres de la mer\\
    avec des sirènes coifées d’algues rouges et brunes\\
    mais des voix humaines nous reveillent et nous coulons
  \end{verse}
    }
  \end{quote}
  \begin{flushright}
    \noun{T.S.Eliot}, \emph{The love song of J.Alfred Prufrock}
  \end{flushright}

  J’ai erré des éons dans d’étroites  ruelles humides aux murs ternes et
  gris, presque lisses,  jamais uniformes, sableux; un jour,  j’ai cueilli la
  lumière -- l’orbe  résonnait en moi, un vrombissement  semblable au bruit
  d’un cœur entêté, perdu en dessous la claie d’un jardin sans couleurs.
  Les pierres autour  d’elle rétrécissaient puis revenaient  à leur taille
  initiale comme victimes d’une torsion élastique de la réalité.

  Alors je connus la chaleur mais je ne sus me brûler. Le vent courait sous le
  soleil  jeune, jaune,  d’une  force  limpide et  décidée;  les cités  en
  cascade se  succèdaient mais  malgré leur  continuel élargissement,  je ne
  croisais nulle silhouette,  nul être, nul semblable, même  à la dérobée.
  À la terrasse envahie de fleurs, je  songeais à une Babylone de rêve alors
  que déjà, et pour des siècles, le couchant montrait ses parures de rubis.

  À ma table, je  me suis ennuyé plus encore; je  descendis des escaliers qui
  imperceptiblement  contrarièrent la  logique  et m’amenèrent  à la  cime
  des  arbres  grandioses. C’est  là  que  naissent  les cités  et  d’où
  l’univers  tire sa  sève noire:  nul n’a  posé vraiment  ses pieds,  ni
  parcouru la  totalité d’une  seule de  leurs branches;  car elles  sont de
  dimension  cosmique et  un seul  de leurs  fruits pendant  noierait de  sucre
  plusieurs galaxies. Seuls quelques navigateurs  ont glissé près de ces bras
  stellaires: mais  les onironautes sont  peu de  chose! Il faudrait  pour cela
  l’immortalité  ou  bien  être  du  côté  de  l’ombree  du  parti  des
  mégalopoles inconnues, de  même nature que les immenses  fruitiers, et cela
  nous est impossible.

  Autrefois, j’ai  connu un enfant  dont l’aisance était  surprenante mais
  qui n’arborait que  des yeux tristes -- puis il  comparut devant les arbres
  et  des  êtres  imprécis;  il  n’apparut plus  à  personne  mais  depuis
  l’écorce des songes est mortellement vigoureuse.

  Ce  fut après  cela que  de mes  yeux parut  l’Autre. Déjà  des millions
  d’années de  vagues murmures en  langues inconnues  avaient tenu à  me le
  présenter sans y parvenir. Ce fut  loin des murs gris, des voiles mandarines
  et des cités alchimiques que je trouvais Isabelle.

  Isabelle, comme la  plupart des humains, n’avait pas de  corps. Il arrivait
  que ses  cheveux virent  à l’évocation  de la  marée noire  et qu’elle
  m’étrangle  avec  ses filaments  de  trous  noirs.  Parfois les  rides  se
  mêlaient à  une étonnante jeunesse;  souvent un extrême chassait  son ami
  dans les  décombres extérieures. De  son visage, si elle  en a eu  un, rien
  ne  subsiste.  Elle allait  et  venait,  ne  se  présentait pas.  Nous  nous
  connaissions depuis les temps ammiotiques,  où tout n’était que taches de
  couleurs vagues et vagues de chaleur à l’orée du réveil.

  Tout était  succession d’images; nous  bougions peu dans notre  demeure de
  bois; sous la charpente, l’or pleuvait  à son gré, finissait l’été et
  l’hiver.  Au  primptemps,  durant  l’automne  bleuâtre,  de  véritables
  bourrasques chatouillaient nos peaux; tout était succession d’images, nous
  étions à l’abri, nous étions si bien.

  Cela ne dura pas.

\PoemTitle{Cathartroll ( 2012)}
  Cet huluberlu qui est là, quelque part entre les mots et la main-plume (main
  porte quoi?), je me demande à quoi il ressemble.

  Ou plutôt, je le sais pertinement, mais cela me fait horreur, j’aime mieux
  le laisser  hurler ça et  là, démolir mes petits  papiers d’enfant-vers,
  d’enfant-limace, escargot. Ah si j’étais un escargot!…

  …je serais  consommé, alors  oublier cela. Vite.  Très vite.  Encore plus
  vite. N’y plus penser. \hulu{Jamais}. Tiens, l’huluberlu est passé.

  L’huluberlu  n’est  pas poète  --  du  moins,  on  ne l’a  pas  encore
  condamné.  Mis à  part  pour vagabondage,  il y  a  longtemps, mais  est-ce
  vraiment lié?

  Il a asile  dans tous les asiles,  sa sœur Zazie au nom  ridicule y éventre
  des poulets  -- pour l’humour,  dit-elle. Il  n’est jamais chez  lui plus
  d’une heure; le reste,  il le passe à coucher avec  un crieur de légumes,
  ou un  crieur de  nouvelles, ou  un crieur  de poèmes,  le plus  mauvais des
  trois. Au  lit, cela  se dispute terriblement;  deux politiques,  deux amours
  opposés: l’huluberlu ne  peut souffrir les angles, aussi  à chaque virée
  nocturne emporte-t-il un pistolet et un marteau.

  Mais oui, l’huluberlu a un visage, il marche dans la foule, tranquille tapir,
  tapi tranquille… Il vous dévisage dans le tramway, assis dans le métro’ ou dans 
  sa cahute de péage autoroutier.

  Quelle eau  qui dort quand  il ne vous  a pas remarqué!  C’est oppressant,
  vous noterez. Et ce n’est pas le pire. Le pire, je vous le laisse.

  Meilleurs vœux de bonheur, meilleurs souhaits pour vous.
