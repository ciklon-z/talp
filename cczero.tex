\addcontentsline{toc}{part}{Résumé explicatif de la license Creative Commons Zero}

\section*{Licence}

\begin{center}
  \includegraphics[scale=0.5]{cczero}

  \textbf{CC0 1.0 universel (CC0 1.0)}

  \textit{Transfert dans le Domaine Public}
\end{center}



\paragraph{Pas de droit d’auteur}

  \begin{quote}
    La  personne  qui  a  associé  une œuvre  à  cet  acte  a  dédié
    l’œuvre au  domaine public en  renonçant dans le monde  entier à
    ses droits  sur l’œuvre  selon les lois  sur le  droit d’auteur,
    droit voisin et connexes, dans la mesure permise par la loi.
  \end{quote}

  \begin{quote}
    Vous pouvez copier, modifier,  distribuer et représenter l’œuvre,
    même  à  des  fins  commerciales, sans  avoir  besoin  de  demander
    l’autorisation. Voir d’autres informations ci-dessous.
  \end{quote}

\paragraph{Autres informations}

  \begin{itemize}
    \item Les brevets et droits de marque commerciale qui peuvent être 
      détenus par autrui ne sont en aucune façon affectés par CC0, de 
      même pour les droits que pourraient détenir d’autres personnes sur 
      l’œuvre ou sur la façon dont elle est utilisée, comme le droit à 
      l’image ou à la vie privée.
    \item À moins d’une mention expresse contraire, la personne qui a 
      identifié une œuvre à cette notice ne concède aucune garantie sur 
      l’œuvre et décline toute responsabilité de toute utilisation de 
      l’œuvre, dans la mesure permise par la loi.
    \item Quand vous utilisez ou citez l’œuvre, vous ne devez pas 
      sous-entendre le soutien de l’auteur ou de la personne qui affirme.
  \end{itemize}

\paragraph{Note}

  Ceci est le \emph{résumé explicatif} du Code Juridique; 
  La version intégrale du contrat est consultable 
  \href{http://creativecommons.org/publicdomain/zero/1.0/legalcode}{ici}\footnote{Note à disposition des éditions papier : \texttt{http://creativecommons.org/publicdomain/zero/1.0/legalcode}}.


