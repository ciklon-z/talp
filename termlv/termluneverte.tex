\PoemTitle{Lune Verte (22 Septembre 2010)}
  \begin{verse}
    Voyez! L'obscurité a fui\\
    -- drainant avec elle l'ennui moribond, rejeton d'un fantasme lacunaire.\\
    Buvez! La journée est finie\\
    -- plus de chefaillons, de partis et de rengaines hémophiles.

    Au delà, et plus, sûrement,\\
    du ciel le bras tendu vers lui,statue de la liberté brouillonne et figée,\\
    élance sans hargne son dard à la Lune\\
    au travers de l'ébriété vide et verte.
  \end{verse}
  \begin{verse}
    Quoi, quatre, cinq bouteilles,\\
    n'est-ce pas adresser assez d'offrandes aux cieux estomaqués?\\
    Qu'importe, soyons autres et démultipliés là haut sur l'île.

    Quoi, faut-il dans la nuit un peu de désordre\\
    pour crever l'abcès saisissant des ténèbres,\\
    et les lueurs urbaines comme autant d'astres glauques?
  \end{verse}
\PoemTitle{Tumulus (22 Septembre 2010)}
  \begin{verse}
    Je me suis réveillé, avec un grand trou\\
    creusé dans la maigre bidoche ;\\
    Dans une fosse neuronale aussi\\
    on aurait pu me chercher.
  \end{verse}
  \begin{verse}
    Un corps gras dans la soupe soupape...
  \end{verse}
  \begin{verse}
    un tumulus de granit, pourtant en éclats,\\
    feldspaths de souvenirs et crâne parsemé de mica,\\
    vaguement assemblé sans architecte sobre ;\\
    un mont incertain se rendort puant l'opprobre...
  \end{verse}
  \begin{verse}
    ...public !
  \end{verse}
  \begin{verse}
    Mais à nouveau il dort et s'en fout,\\
    aimant sans honte ni regrets\\
    la nuit dont il a déjà tout oublié.
  \end{verse}
\PoemTitle{Les garnisons d'amadou verdoyant (16 Septembre 2010)}
  Les lignes de feu se sont éteintes  un soir, au clair du bois transversal et
  rond, comme  les pelages  d'agneaux noirs et  échevelés. Ils  ruminent, ils
  grognent pourtant rien n'est à eux, ni est à leur donner.

  Ils vont  tristes et  tous semblables,  à l'étable,  parqués d'un  azur trop
  clairvoyant et passé, intoxiqué à l'entretien et au maintien, leurs mains ne
  tiennent plus rien, pas plus que leurs cœurs en miettes.

  Parfois, le Soleil !S'il brille, pourtant,  c'est toujours trop peu, pas assez,
  pas assez  longtemps – et l'aurore,  lui dicte-t-on sa mort,  alors que faire
  des  naissances de  cieux ?Alors  les congénères  rêvent, pensent  au dessus
  d'eux, les murs se déchirent à  leur conscience, projetée au delà comme les
  gazs  des  spray anti-odeur,  plaqués  dans  l'éthéré  comme des  clous  de
  guimauve sur du bois de cacao. Dans le délire subsiste la résignation du seul
  noir qui au final l'emportera.

  C'est le  jeu des alliances  et des  plaines funestes pour  pas un sou  de sang
  craché, le fleuve des amas verbeux qui  colle au lit et dépose les paquets et
  les plis de couverture au delà des barges du ciel plein de lumière comme s'il
  suffisait de luminaires pour  voir clair dans la mer d'yeux  et de fantasmes en
  cohorte défilant sans se soucier de rien.

  Bonheur, d'être  ici et là, vide  de sens et  plein de viande, cheveu  sur la
  soupe froide d'iceberg, fondant inversement avec une peau de plus en plus dure,
  rêche, coquille d'hérissons en devenir,  le croque-mort, le croque-mitaine et
  le monsieur  noir en robe, exhibitionniste  sans montrer jamais le  bout de son
  nez bien que le Maître le saura un jour ou l'autre.

  Dormir, dormir,  dormir encore, rêver, ne  plus songer, voilà qui  est mieux,
  les trous de mur  n'en finissent pas de donner du  néant aux avertisseuses. Ta
  passoire d'où  tu fais gicler tes  membres, ils ne pourront  pas te l'acheter,
  même là où tout se vend, il reste encore des cygnes noirs et des architectes
  alambiqués. Ils prostitueront leurs principes à grande vitesse de combustion,
  mais leur antre  mitigé et recouvert de voiles obscurs,  s'opposera à tant de
  muselières retirées, carnassières mangeuses d'hommes.

  Profite, profite, des chairs et des mots et des chaires ;\\
  Ici rien n'est vrai, tout est réel et sans limite. Sans limite.
\PoemTitle{Apostrophe de l'agoraphobe (24 Septembre 2010)}
  Grands souillons de matelots, faut-il que l'amer se substitue à l'intérêt ?

  Les foules,  comme des  océans d'yeux et  de salive immonde  et opaque  – si
  noire qu'on s'y perd si on ne prend garde d'être aspiré au fond du trou.

  Espaces immortels  et infinis, je vous  déteste quand vous mugissez  ;la ville
  n'est pas  un monolithe, mais  un vers géant qui  s'ébroue avec le  froid qui
  tombe sur son cadavre régulier.
\PoemTitle{La fourmi aquaphile (24 Septembre 2010)}
  La pluie a  d'orgasmique le mur inverse de nos corps avares.  Quand ses traits
  tombent sur  tout à  la ronde, on  croit que chutent  des cohortes  de démons
  germeurs de zizanie.

  Quoi que ce ne  soit pas en grande pompe qu'ils s'annoncent,  ni avec le froid,
  ni la faim, ni la drogue ;les  éclats électriques du ciel ont perdu largement
  de leur crédibilité, et les bêtes de la pluie sont malheureuses.

  Pourtant,  le  chaos  est  là,  et  immense.  Un  masque  morne  s'empare  des
  zygomatiques des  singes avancés, qui  ne font  alors qu'une longue  marche à
  grand peine,  l'œil vitreux contre leur  propre glace, terne et  réaliste, et
  qui  s'éparpille  au  sol comme  autant  de  débris  de  miroirs –  Ô  les
  redresseuses de torts de la pluie ont une équité qui me plaît.

  Car  quand  j'avance  trempé  sous  la   pluie,  perdu  au  milieu  d'une  mer
  d'immeubles,  fourmi  craignant  simplement  de   se  noyer,  elles  sont  là,
  nombreuses et  valeureuses, à venir s'échouer  sur le récif vulgaire  de mes
  pieds, et d'agoniser : « Courage ! ».
%\PoemTitle{Poésie du tram (Perdu)}
\PoemTitle{La Liane Accrochée (26 Septembre)}
  \begin{verse}
    C'était un soir, dimanche houleux\\
    en attendant, bête obstiné,\\
    le tram, – lent à se déplacer,\\
    empli de passager fiévreux, –\\
    que je tombais sur l'infini\\
    le plus intense à la ronde,\\
    qui transformait les mètres en mondes.
  \end{verse}
  \begin{verse}
    Il faut partir, trouver sa voie encombrée\\
    dans un train de ville au contenu englué,\\
    batailler comme à une révolution\\
    pour prendre place dans l'amas brouillon.
  \end{verse}
  \begin{verse}
    Mais la nuit se couvre de monstres en troupeau\\
    harassés, fatigués, lourds d'ennui,\\
    quand l'adorée regardée\\
    reste, résignée, au bord du quai.
  \end{verse}
\PoemTitle{Tréfonds farceurs}
  \begin{verse}
    Des murs épais et gris\\
    comme des arbres ont poussé dans son cervelet,\\
    longue suite imposée d'histoires qu'il suit sans comprendre.
  \end{verse}
  \begin{verse}
    Les pièges se succèdent, et les farces plus élaborées,\\
    les réveils se font faux,\\
    Oniros règne et enfonce ses doigts dans son cerveau\\
    pour le divertir et le maîtriser,\\
    et invoque, trompeur, l'oubli.
  \end{verse}
  \begin{verse}
    Les pages deviennent rares,\\
    la lucidité moins évidente,\\
    les arbres plongent leurs racines dans l'Hadès,\\
    mais l'avidité reste pourtant au bord de la barque\\
    pour aspirer tout le fleuve.
  \end{verse}
\PoemTitle{Inutilité et expansion (2010)}
  Eh  quoi,  diriez-vous  que  la  nuit   ajoute  une  touche  de  morosité  aux
  lieux,  communs  en  tous  points,  car  obscurs  ?  Ils  partagent  cela  avec
  l'encéphalogramme détraqué des hommes.

  Jamais  contents, jamais  repus.  Ne pas  se poser,  ne  pas ralentir,  courir,
  toujours  courir, quand  bien même  les bords  sont sains  et la  route unique
  parsemée de  serpents. Mais  c'est la  nuit, et  on n'en  sait rien,  et toute
  seule, la Lune se reflète dans le lac, invention comme une autre.

  Quand il y a quelques lueurs qui nous parviennent, c'est déjà trop tard, tels
  des satellites  irrémédiablement distants,  nous n'y comprenons  rien. Humeur
  quelconque,  sommet  du doute,  trouble  inconséquent  ?Voilà. La  fatigue  a
  gagné, et la musique s'oublie.

  Il y a des années, vous auriez eu des visions à la pelle ;maintenant… …il
  n'y  a plus  rien,  juste le  néant,  une sublime  inutilité  est restée  en
  expansion, et c'est très bien ainsi.
\PoemTitle{Réalité organique (2010)}
  L'horloge aux milliers  de gonds d'argent pulse tranquillement  les instants du
  monde. Organe orgiaque de l'univers, pas  un seul globule minable n'échappe à
  son appétit assimilateur.

  L'heure a  sonné pour certains  ;et elle sonnera  pour tous, pour  tout être,
  afin qu'il voit le vide à moitié polygonal ou l'œuf fendillé par le milieu.
  Il y en a  qui n'ont pas attendu, et qui ont ouverts  leurs cœurs enfantins à
  cet  ogre sans  descendance,  sinon d'autres  égéries fabuleusement  inutiles
  telles que  la mélancolie, la tristesse  et l'art – Pouah  !Ces trois choses
  là sentent  le coprolite  et le  sapin, soyons heureux  qu'ils ne  sachent pas
  parler, hormis la dernière, peut-être, qui parvient tout juste à baragouiner
  des pelletées de foutaises.
\PoemTitle{Entrée dans l'arène (5 Octobre)}
  \begin{verse}
    La nuit,\\
    ~~~~je ne comprends rien à la nuit.\\
    ~~~~Elle se diffuse à l'aise\\
    ~~~~dans l'air par les hommes délaissés.
  \end{verse}
  \begin{verse}
    Les mots,\\
    ~~~~je ne comprends rien aux mots.\\
    ~~~~Ils coulent sans s'en douter\\
    ~~~~tâches obscures dans le jour d'une page blanche.
  \end{verse}
  \begin{verse}
    La vie,\\
    ~~~~je ne comprends rien à la vie.\\
    ~~~~Là, ici, partout sans qu'on le sache, elle est là,\\
    ~~~~étirant ses pattes multiformes, gazon de consciences,
  \end{verse}
  \begin{verse}
    Elle explose en hurlant,\\
    ivre d'elle même,\\
    et cherche sans entrain son propre sens.
  \end{verse}
\PoemTitle{L'amitié géographe (5 Octobre)}
  \begin{verse}
    Qu'il est beau de voir des visages familiers\\
    quand les excroissances urbaines se réunissant en une mâchoire\\
    beuglent et broient frénétiques la fraternité !
  \end{verse}
  \begin{verse}
    Et combien la valeur des amis\\
    comme leur présence devient chère à nos yeux !
  \end{verse}
  \begin{verse}
    Vous étiez perdu tout à fait\\
    quand un fantôme vous a retrouvé.
  \end{verse}
